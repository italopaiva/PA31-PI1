	Este tópico baseia-se no estudo de softwares existentes no mercado que possuem o intuito de testar a qualidade da água,
	para assim dar a segurança nescessária de qualidade ao consumo humano, de acordo a constituição vigente, e as normas
	de qualidade de saúde determinadas pela Agência Nacional de Águas (\citeauthor{anaGov}).
	
	Estes softwares poderão ser aplicados no sistema de armazenamento da água coletada através do ar, uma vez que as
	partículas de água podem conter substâncias químicas, o que torna a água inapropriada ao consumo. Devem ser utilizados
	com o auxilio de sensores para coleta dos dados. \\

% \begin{center}
\noindent
\textbf{Water Quality Analyser} (Fonte: \citeauthor{eWater})
% \end{center}	

	Com uma interface de usuário simples e direta possuindo foco na visualização de entradas e saídas de dados, este
	software ajuda a identificar a qualidade de água e simplificar o caminho para uma breve avaliação, através de
	ferramentas, que comparam a qualidade da água com os termos legais previamente definidos.
	
	Utiliza um gerenciamento de dados, para validação, vizualização e apresentação de relatórios, além de fornecer
	estatísticas de mudanças e aleatoriedade na qualidade da aguá, além de outros dados temporais.
	
	Este software foi desenvolvido pela colaboração, entre a empresa eWater CRC e o Departamento de Meio Ambiente
	e Gestão de Recursos de Queensland (QDERM), na Australia.
	
	A ferramenta é paga custando \$299 dollares, equivalente á R\$ 892,89 reais (cotação de 20/04/2015 - preço do
	dollar 2,98) pelo licenciamento de um ano, possuindo a possibilidade de teste gratuito pelo período de 30 dias.

	
\textbf{Prós:}
	
\begin{itemize}
  \item Muitas ferramentas;
  \item Interface simples;
  \item Recomendado para utilização em escala industrial;
 \end{itemize}
 
\textbf{Contras:}
	
\begin{itemize}
  \item Disponível apenas em inglês;
  \item Ferramenta paga;\\
 \end{itemize}

 
% \begin{center}
\noindent
\textbf{Logger Pro} (Fonte: \citeauthor{loggerPro})
% \end{center}	

	Utiliza coleta de dados em tempo real, suportando mais de 80 sensores e dispositivos diferentes, apresenta os dados
	em interface aceitando os sistemas operacionais Windows ou MAC.
	
	Custa \$339, equivalente à R\$1.012,34 (cotação de 20/04/2015 - preço do dollar 2,98). Possuindo uma versão Lite,
	gratuita e com menos funcionalidades.
	
\textbf{Prós:}
	
\begin{itemize}
  \item Multiuso, com muitas ferramentas;
  \item Coleta em tempo real;
  \item Captura vídeos;
  \item Desenha previsões utilizando gráficos de dados coletados anteriormente;
  \item Realiza análise estática dos dados;
  \item Possui gráficos avançados com ajuste de função;
  \item Transmição sem fio para dispositivos móveis;
 \end{itemize}
 
\textbf{Contras:}
	
\begin{itemize}
  \item Preço dos equipamentos;
  \item Feito para nível estudantil;\\
 \end{itemize}
 

% \begin{center}
\noindent
\textbf{AquaChem} (Fonte: \citeauthor{aquaChem})
% \end{center}	

	Ferramenta para análise cobrindo uma ampla gama de funções e cálculos utilizados para análise, interpretação 
	e comparação de dados da qualidade da aguá. Possuindo estatísticas e funções mais complexas, como matrizes de
	correlação e cálculos análiticos.
	
	Possui preço em dólares variando entre 1.355,75(R\$ 4.048,61) até 2.390,50 (R\$ 7.138,63).
	
\textbf{Prós:}
	
\begin{itemize}
  \item Cálculos geoquímicos;
  \item Análise simples e fácil de dados de qualidade da água;
  \item Cálculos estatísticos;
  \item Criado para escala comercial;
\end{itemize}
 
\textbf{Contras:}
	
\begin{itemize}
  \item Alto custo;
\end{itemize}