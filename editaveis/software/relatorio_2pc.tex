\documentclass[12pt,openright,oneside,a4paper,brazil]{abntex2}
\usepackage[utf8]{inputenc}
\counterwithout{section}{section}
\counterwithout{figure}{chapter}
\counterwithout{table}{chapter}
\setlength{\parindent}{1.3cm}
\usepackage{indentfirst}
\setlength{\parskip}{0.2cm}
\usepackage[bottom=2cm,top=3cm,left=3cm,right=2cm]{geometry}
\usepackage{graphicx}
\graphicspath{{figuras/}}
\usepackage{placeins}
\usepackage{longtable}
 
 %opening
% \title{}
% \author{}
 
\begin{document}
\textual
\begin{center}
 {\large Planta de abastecimento da água através da umidade do ar}\\[0.2cm]
 {Atividades responsáveis pela frente de Software para o Ponto de Controle 2}\\
\end{center}
 
 \section*{Introdução}
 Neste relatório estão contidas todas as atividades que compõem o projeto da disciplina que estavam sobre a responsabilidade pela frente de pesquisa de \emph{Software} da equipe, os objetivos que se desejavam alcançar com a resolução destas atividades que que valor agregado estas trariam para o projeto.
 \section*{Atividade}
As atividades que ficaram responsáveis pela frente de pesquisa de \emph{Software}, assim como as das outras frentes de pesquisa, foram divididas em \emph{Sprints} devido a metodologia de desenvolvimento que foi adaptada ao projeto chamada \emph{Scrum}.
Algumas atividades que não foram cumpridas na \emph{Sprint} atual foram alocadas para a próxima \emph{Sprint}. Neste relatório, as atividades estão sendo mostradas de acordo com o planejamento final das Sprints, já com as atividades alocadas na \emph{Sprint} em que ela foi realizada.
As atividades são as seguintes:

\textbf{\emph{Sprint 1:}}
\begin{itemize}
\item Analisar a viabilidade dos Softwares de monitoramento da qualidade da água que foram encontrados.
\item Descrever a análise dos Softwares.
\item Levantar os dados monitorados pelos Softwares e verificar se condiz com os parâmetros do Índice de \item Qualidade da Água (IQA) estabelecidos pela frente de pesquisa de eletrônica.
\end{itemize}

\textbf{\emph{Sprint 2:}}
\begin{itemize}
\item Estabelecer requisitos funcionais e não funcionais baseando-se nos parâmetros de qualidade que a equipe necessita estabelecer no projeto e em dados analisados nos softwares similares.
\item Elaborar documento de visão de um sistema de controle da qualidade da água.
\end{itemize}

\textbf{\emph{Sprint 3:}}
\begin{itemize}
\item Elaborar o diagrama de caso de uso do sistema de controle da qualidade da água.
\item Estabelecer um protótipo do sistema de controle da qualidade da água.
\end{itemize}
 
\section*{Objetivos}
Com a execução dessas atividades, a equipe objetivava a criação de um protótipo de um sistema de controle de qualidade da água que é captada pelas turbinas. Não podemos abastecer a água em qualquer estado, pois nada se garante que a mesma esteja própria para o consumo humano.

\section*{Conclusão}
Ao final das três Sprints, a equipe responsável pela frente de pesquisa de Software conseguiu cumprir com êxito todas as atividades que lhe foram atribuídas, o que é de devida importância para o desenvolvimento do projeto, pois evita atrasos nas demais atividades das outras frentes de pesquisa. 

\end{document} 

