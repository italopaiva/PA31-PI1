Para analisar a qualidade da água são necessários sensores, os dados obtidos por esses sensores são transmitidos para
unidades com capacidade de processamento, para computadores. Os dados obtidos por esses sensores individualmente devem
ser analisados via software. Para efetuar a análise da qualidade da água, é necessária uma análise dos dados obtidos
pelo conjunto desses sensores, pois cada um indica uma ou algumas características específicas da água e não a qualidade
como um todo.

O software a ser utilizado depende basicamente do circuito eletrônico escolhido, tanto do computador, como dos tipos
e especificações dos sensores. Para medir a qualidade da água serão utilizados sensores capazes de avaliar basicamente
o nível de PH, oxigênio dissolvido, temperatura, íons dissolvidos, turbidez, resíduo total da água.

No software será definido o intervalo em que cada um dos indicadores de qualidade é aceitável para consumo humano de
acordo com a análise feita pelo grupo.  Em algumas situações, não seguiremos estritamente os padrões estipulados pelas 
agências reguladoras, desde que a escolha esteja dendro do estipulado por elas e traga algum benefício aos usuários. 
Por exemplo, o valor ideal de PH da água para consumo humano é de 6 a 9, porém a Agência Nacional de Vigilância Sanitária (ANVISA)
não estabelece nenhuma resolução indicando um valor mínimo para o pH.

Outro aspecto importante do software é que ele impeça a passagem da água, caso ela esteja fora dos padrões de qualidade
estipulados, até que o tratamento ocorra e seja finalizado. Para que isso ocorra,  o software fará uma uma comparação
entre cada um dos índices com os valores adequados e, caso a água não esteja dentro do padrão, o hardware enviará um 
sinal elétrico, comandando um mecanismo eletromecânico capaz de bloquear a passagem de água.

Além da análise da qualidade da água, será necessário sensores para analisar o nível da água tanto no reservatório
de distribuição de água, como no reservatório de armazenamento de água para tratamento. Os dados relacionados ao nível 
e qualidade da água serão transmitidos para um outro sistema que possibilita uma interação com o usuário, composto por
um software diferente.

Pretende-se utilizar a linguagem C/C++ para a elaboração do software relacionado ao controle da água. Estas linguagens
pode ser compiladas e utilizadas para uma grande parte dos sistemas embarcados. Para o desenvolvimento, será feito uma
análise mais profunda do problema, serão levantados todos os requisitos funcionais e não funcionais do software, a 
arquitetura, além disso, serão feitos testes ao longo do processo de desenvolvimento.

Além desse software, será necessário o desenvolvimento de outros, que estão mais relacionados à interação com o ser
humano, apresentando para o usuário os índices de qualidade da água. Esses softwares serão executados por hardwares
mais potentes e,  portanto, não haverá tantas limitações de processamento e memória. Dessa forma, não será necessário
fazer uma análise profunda da interação entre a eletrônica e o software, embora seja necessário uma análise dos 
sistemas operacionais, linguagens que possam ser interpretadas ou compiladas para ele.