  
  O presente trabalho não irá fornecer um sistema completo com todas as funcionalidades implementadas,
  apenas um protótipo da interface com o usuário. Para tal, é necessário o entendimento de alguns conceitos de
  Interação Humano-Computador.\\
  
  \noindent
  \textbf{Usabilidade}
  
  No processo de produção de software, uma parte importante é a produção de protótipos. Esses protótipos sevem
  como base para a validação de requisitos, além de serem usados para a avaliação de usabilidade do software.
  Sendo assim, os softwares que serão desenvolvidos devem seguir os seis princípios de usabilidade, segundo \cite{preece02}, que são:
 
 \begin{itemize}
  \item Eficácia: O software deverá fazer bem o que lhe foi destinado a fazer.
  \item Eficiência: O software deverá fazer suas funções de modo rápido e fácil.
  \item Segurança: O software deverá prevenir o usuários de cometer erros que prejudiquem a realização do que o usuário deseja.
  \item Utilidade: O software deverá ser útil.
  \item Capacidade de aprendizagem: O software deverá ser fácil de ser utilizado.
  \item Capacidade de memorização: Após a primeira vez de uso, o usuário deverá ser capaz de utilizar o software novamente com mais facilidade do que na primeira vez.
 \end{itemize}
 
	Para a construção dos protótipos, serão utilizados ferramentas de mockup, abaixo estão listadas algumas no mercado:
     
\begin{center}
\textbf{Balsamiq}
\end{center}	

	A ferramenta Balsamiq Mockup permite a criação de protótipos e modelos de baixa e alta fidelidade,
	para sistemas desktop, web ou mobile.
	
\textbf{Prós:}
	
\begin{itemize}
  \item Interface Intuitiva;
  \item Possui versão para desktop;
  \item Versão trial de 30 dias;
  \item Preço baixo para a licença;
 \end{itemize}
 
\textbf{Contras:}
	
\begin{itemize}
  \item Não oferece equipe de suporte;
 \end{itemize}
     
     
     
\begin{center}
\textbf{Mockingbird}
\end{center}	

	É uma ferramenta online que permite a criação e compartilhamento de protótipos.
	
\textbf{Prós:}
	
\begin{itemize}
  \item Interface Intuitiva;
  \item Possui equipe de suporte;
  \item Ferramenta gratuita;
 \end{itemize}
 
\textbf{Contras:}
	
\begin{itemize}
  \item Não poussi aplicação offline;
 \end{itemize}
      
     
     
\begin{center}
\textbf{MockFlow}
\end{center}	

	Conjunto de wireframing profissional para projetar interfaces.
	
\textbf{Prós:}
	
\begin{itemize}
  \item Interface Intuitiva;
  \item Versão online ou desktop;
  \item Ferramentas para trabalhar em equipe;
\end{itemize}
 
\textbf{Contras:}
	
\begin{itemize}
  \item Versão free limitada a 4 páginas;
  \item Não possui a opção de pagamento mensal;
  \item Baseado em Flash;
\end{itemize}