\chapter[Considerações finais]{Consideracões finais}

  \section{Ponto de Controle 1}

    Durante a fase inicial do projeto a equipe teve dificuldadades para definir o local da implantação do projeto,
    o que gastou uma quantidade significativa de tempo em pesquisas. Inicialmente haviam várias propostas de locais,
    que foram sendo descartadas ao longo do tempo até que ficaram duas propostas de local: na cidade de Acarí no Rio
    Grande do Norte e em plataformas de petróleo em alto mar. Para escolher entre as duas propostas finais, o grupo
    também levou bastante tempo pesquisando a viabilidade, prós e contras de cada uma das opções. Essa dificuldadade
    em definir o local foi um dos maiores problemas ocorridos durante essa fase do projeto, pois gastou um tempo de
    produção muito alto. Após o grupo ter acordado o local no qual seria focado os esforços do projeto, cidade de Acarí-RN,
    passamos para a fase de definir a tecnologia que seria utilizada e o local específico dentro da cidade no qual seria
    implantada a solução. Como dispúnhamos de bastantes opções de tecnologia, um tempo considerável de pesquisa foi gasto
    para levantar a viabilidade das mesmas e onde exatamente colocá-las dentro da cidade. 

    Um dos pontos que fizeram com que a equipe se atrasasse em relação ao primeiro ponto de controle foi o fato da
    presença dos planos de gerenciamento,  que tomou grande parte do tempo de todos os integrantes para a realização
    destes. O atraso se deu pelo fato de não se ter claro a necessidade de tais planos no primeiro ponto de controle,
    mobilizando grande demanda de atenção e trabalho, que ao final poderia ter sido concentrado em pesquisas e temas 
    que precisaríamos realmente ter abordado para o objetivo em questão.

    Com o primeiro ponto de controle abrindo a avaliação e monitoramento oficial do projeto, algumas pendências já
    foram destacadas como objetivo para o segundo ponto de controle. O projeto para o segundo ponto de controle deve
    ter um caráter mais preliminar que o apresentado inicialmente, ou seja, será apresentado novamente o escopo, porém
    desta vez de forma mais breve e objetiva, contando com todos os avanços realizados ao longo do desenvolvimento do 
    projeto. Este avanço deve ter neste segundo momento uma característica mais técnica e específica, sem estimativas
    e sim contendo números reais, dados, contas e informações de cunho científico, baseados em artigos e referências válidas.
    Também deverá possuir já a solução preliminar específica para o problema em questão que será criada e projetada,
    assim como deverá conter as informações técnicas passo a passo de seu funcionamento e também de seus efeitos quando
    em atividade.

    Devido ao desvio de curso causado pela elaboração dos planos, a equipe somente conseguiu elaborar o referencial
    teórico, os planos de gerenciamento e os demais documentos, ficando lacunas para o avanço da solução 
    para nosso projeto. Outro problema encontrado foi na dificuldade de utilização da ferramenta proposta LaTeX e 
    quanto à organização e gerenciamento de um grupo relativamente grande, aumentando assim a lista de pendências 
    que serão sanadas ao longo do desenvolvimento da segunda iteração. Devido aos problemas com a ferramenta, e com a perda
    de tempo inerente, alguns conteúdos do documento que já estão prontos não foram adicionados ao mesmo.
