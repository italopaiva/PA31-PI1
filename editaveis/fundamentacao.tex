\chapter[Fundamentação]{Fundamentação}
\addcontentsline{toc}{chapter}{Fundamentação}

  \section{Metodologia de desenvolvimento}
  
  Descrever aqui a metodologia utilizada para o desenvolvimento do projeto, NÃO a da gerência.
  
  \section{Referencial Teórico}
  
    Esse tópico aborda..
    
    PROCUREM POR TEXTOS DE BASE CIENTÍFICA PARA ESSES TÓPICOS SEGUINTES, E LEMBREM DE REFERENCIAR.
    
  \subsection{Sistema de captação da umidade do ar}
  
  Descrever aqui, o mais detalhado possível, TODAS as tecnologias que pesquisamos.
  
  Descrever, detalhadamente, possíveis materiais que serão utilizados no equipamento.
  
  \subsection{Matriz energética}
  
  Descrever as fontes energéticas pesquisadas e fazer relação com as tecnologias.
  
  Técnicas e métodos de conversão e armazenamento que podem ser utilizados (Descrever cada um).
  
  Descrever como medir a eficiência energética e autonomia do sistema, descrever conceitos o máximo possível.
  
  \subsection{Sistema de monitoramento e controle da qualidade da água}
    
    \subsubsection{Parâmetros de qualidade da água}
    
    Descrever e conceituar parâmetros de qualidade da água
    (Pesquisar na ANA (Agência Nacional de Águas), ADASA (Agência Reguladora de Águas, Energia e Saneamento Básico do DF), CAESB...).
    Pesquisar modelos matemáticos inerentes ao assunto.
  
    \subsubsection{Projeto eletrônico e de controle da planta}
    
    Estudo dos possíveis componentes eletrônicos (descrever detalhadamente) para o monitoramento da qualidade da água.
      Sensores, Microcontroladores, etc..
      
    Descrever métodos de processamento de sinais.
    
    Descrever unidades central de processamento.
    
    \subsubsection{Interface do sistema de monitoramento da qualidade da água}
    
    Descrever algoritmos que possam ser utilizados.
    
    Pesquisar sistemas de monitoramento de água já existentes e descrevê-los.
    Descrever os dados que precisam ser acompanhados pelo sistema.
    
    Descrever conceitos de IHC (usabilidade, protótipos, etc...).
    
    Pesquisar ferramentas de MockUp e descrevê-las.
    
    
    
    
    
    