\chapter[Fundamentação]{Fundamentação}
\addcontentsline{toc}{chapter}{Fundamentação}
  
  Neste capítulo serão abordados tópicos como a metodologia de desenvolvimento do projeto e o referencial teórico levantado
  para cada área específica do projeto.
  
  \section{Metodologia de desenvolvimento}
  
  A metodologia de desenvolvimento adotada neste projeto é uma adaptação de metodologias gerais de desenvolvimento de
  produtos e serviços. Para a construção deste modelo de desenvolvimento foram selecionadas algumas técnicas de projeto
  de produtos e serviços expostos em algumas obras de Slack (\citeyear{slack99}), Krajewski (\citeyear{krajewski96}) e
  Ramaswamy (\citeyear{ramaswamy96}). Tais trabalhos descrevem formas de estabelecer etapas bem definidas para o desenvolvimento do projeto.
  
  Como estrutura básica do projeto será adotada a seguinte metodologia, dividida em seis etapas:
  
  \begin{figure}[h]
  \begin{center}

  \includegraphics[scale=0.3]{editaveis/figuras/metodologia_de_desenvolvimento}
  \label{Metodologia de desenvolvimento}
  \caption{Metodologia de desenvolvimento}
  \end{center}
  \end{figure}
  \FloatBarrier
  
  A descrição de todas as etapas acima é relatada a seguir.
  
  \begin{enumerate}
   \parindent=1.25cm
   
   \item Geração de Conceito\\
      
      \noindent
      Divide-se em três:
      
      \noindent
      \subitem \textbf{Geração de ideias}
      
      Foi proposto ao grupo o desenvolvimento de um projeto preliminar de uma planta de abastecimento de água potável a partir 
      da umidade do ar. O projeto a ser realizado deveria ser inovador, de utilidade para seus consumidores finais, viável e
      que utilizasse fontes alternativas de energia para a solução do problema a ser idealizado pelo grupo.
      
      O método utilizado nesta etapa do projeto consiste em produzir reuniões onde todo o grupo estará presente, tais componentes 
      devem fornecer ideias que possam ser debatidas e analisadas por todos, sem descriminação.
      
      \noindent
      \subitem \textbf{Análise de tecnologia}
      
      Nesta fase devem ser realizadas vigorosas pesquisas externas a fim de investigar a existência de tecnologias que realizam o
      trabalho proposto ou que realizam processos similares. 
	    
      A proposta desta etapa é investigar os prós e contras das várias tecnologias pesquisadas e a viabilidade da tecnologia 
      a ser adotada.
      
      \noindent
      \subitem \textbf{Especificação de localização}
      
      Análise de requisitos sobre localização e avaliação das características do problema, a fim de identificar os possíveis 
      consumidores e sua localidade. 
      
      Os potenciais consumidores deste projeto são as famílias nordestinas, que tem suas vidas diariamente sendo ameaçadas 
      devido à falta de reserva de água.
   
  \item Seleção de Conceitos
  
    Esta etapa é responsável por avaliar quais conceitos gerados na fase anterior são relevantes e pela construção do
    embasamento teórico sobre os elementos determinados na etapa anterior. 
  
  \item Projeto Preliminar
  
    Esta etapa busca especificar o serviço e o produto com todos os seus componentes e processos necessários ao funcionamento.
    
    Esta fase é uma das mais importantes de todo o projeto, pois são tomadas decisões que serão o suporte de todo o
    desenvolvimento deste. No projeto preliminar deste projeto foram estabelecidas as seguintes informações principais: 
    
    \begin{itemize}
     \item A região de implementação do projeto será no Município de Acari, situado mais especificamente na região do Seridó, 
	na Mesorregião Central Potiguar, no estado do Rio Grande do Norte, no Brasil.
     
     \item A tecnologia adotada será uma turbina eólica capaz de transformar a umidade do ar em água.
     
     \item Serão adotados critérios quanto a qualidade da água, eficiência da turbina e 
	monitoramento de todo o sistema de captação, armazenamento e distribuição.     
    \end{itemize}
    
  \item Projeto Detalhado e Protótipo
  
    O projeto detalhado se diferencia do projeto preliminar devido às alterações e correções implementadas.
    
    O protótipo do projeto será feito na plataforma de desenho Catia v5\_R19, este processo de construção deve ter como 
    base o projeto detalhado e as dimensões reais do produto, para ser feito em menor escala.

  \item Avaliação e Melhoria
  
    Antes do término do projeto e lançamento do serviço, algumas modificações devem ser feitas em busca de melhorias. 
    Dentre as técnicas utilizadas neste processo, serão recomendadas o Desdobramento da Função Qualidade
    (ou QFD, Quality Function Deployment) e da Análise de Valor.
    
    O Desdobramento da Função Qualidade (QFD) tem como foco principal às necessidades dos clientes, oferecendo as
    alternativas capazes de satisfazê-los. Ou seja, dentro da matriz QFD os requisitos do consumidor (o quê) relacionam-se
    com as características do serviço (como), para que possamos prover mudanças que aumentem a satisfação do cliente.
    
    A Análise de Valor tem como objetivo aumentar o valor relativo de cada componente do serviço prestado, o que pode 
    ser feito através da redução de custos ou através do aumento do nível do serviço. Deve-se, em uma primeira etapa,
    distinguir as funções básicas das secundárias, para em seguida identificar tudo o que possa oferecer diminuição de custos,
    principalmente em funções secundárias; e aumento do valor, em funções básicas.
    
  \item Projeto Final
  
    O projeto final deve conter as alterações propostas por meio das técnicas de avaliação e melhoria, em modelo semelhante
    ao do projeto preliminar, porém de forma extremamente mais completa.
    
  \end{enumerate}

  
  \section{Referencial Teórico}
  
    Essa seção aborda alguns conceitos chave referentes ao trabalho.
    
    \subsection{Sistema de captação da umidade do ar}
    
      % \documentclass[12pt,openright,oneside,a4paper,brazil]{abntex2}
% \usepackage[utf8]{inputenc}
% \counterwithout{section}{section}
% \counterwithout{figure}{chapter}
% \counterwithout{table}{chapter}
% \setlength{\parindent}{1.3cm}
% \usepackage{indentfirst}
% \setlength{\parskip}{0.2cm}
% \usepackage[bottom=2cm,top=3cm,left=3cm,right=2cm]{geometry}
% \usepackage{graphicx}
% \graphicspath{{figuras/}}
% \usepackage{placeins}
% \usepackage{cite}
% \usepackage{url}
% \usepackage{breakurl}
% \include{bibliografia}
% 
% \makeatletter
% \setlength{\@fptop}{0pt}
% \makeatother
% 
% \begin{document}
% 
% \textual
% \begin{center}
%  {\large Captação de água e materiais estruturais}\\[0.2cm]
%  \end{center}
 
Dentre as inspirações de tecnologia a serem aplicadas no projeto, as que se destacaram foram a Eolewater, Maxwater, Skywater
e Warkawater. As três primeiras se baseiam no cooling compression, porém com custos diferentes. A quarta tem custo  
extremamente baixo se comparado com as outras três tecnologias, por causa da simplicidade dos materiais que a constituem.
Contudo, produz valores significantemente menores de água, além de não gerar energia para os processos de controle da qualidade
da água. Por não gerar energia, também descartamos o Skywater.

As duas primeiras se baseiam em turbinas eólicas autossuficientes que retiram a umidade do ar, condensam,
purificam e distribuem. Devido a geometria dos rotores do Max water que não é tão eficiente como os rotores tradicionais,
de turbinas eólicas comuns, foi escolhido como base O sistema da Eolewater, que está descrito na Figura ~\ref{max_water_turbina}.

\begin{figure}[!ht]
\centering
\includegraphics[scale=0.6]{editaveis/figuras/max_water}
\caption[Max Water]{Max Water. Note como seus rotores são paralelos e verticais, tal configuração é ineficiente do ponto
	de vista aerodinâmico por gerar turbulência sobre os rotores vizinhos e gerar torques contrários com uma mesma 
	direção de fluxo de ar.\footnotemark}

\label{max_water_turbina}
\end{figure}
\footnotetext{Disponível em: http://peswiki.com/index.php/Image:Max-water.jpg}

\begin{figure}[!htbp]
\centering
\includegraphics[scale=0.6]{editaveis/figuras/Componentes}
\caption[Componentes de uma turbina Eolewater.]{Componentes de uma turbina Eolewater.\footnotemark}
\FloatBarrier
\label{Eole_Water}
\end{figure}

\footnotetext{Fonte: \cite{renewable} }

As componentes de uma turbina eólica pouco mudam para essa acima,pois a Eolewater além de gerar energia para seu próprio
funcionamento gera água, enquanto que a turbina eólica gera apenas energia.Na maioria das tecnologias de sucesso que 
pesquisamos, A Obtenção de água é feita pela condensação  a frio (cooling condensation), que é feita com o contato de 
ar com uma superfície fria. Para gerar essa superfície fria, um compressor comprime um fluido refrigerante, elevando sua
temperatura. Esse fluido em alta temperatura passa por um trocador de calor e depois é expandido, o que causa uma queda 
ainda maior na temperatura do fluido. O fluido sob baixa temperatura circula por um condensador, por onde passa o ar 
atmosférico coletado. Esse condensador faz com que a temperatura do ar caia até o ponto de orvalho, temperatura na qual a
água presente no ar se condensa em pequenas gotículas devido a saturação da quantidade de água no ar. A água proveniente
da condensação é coletada e passa por tratamentos em UV e carvão ativado para que seja descontaminada e esteja pronta para
consumo.



No momento, o levantamento de materiais será apenas estrutural e se baseará em uma turbina eólica.

\begin{figure}[!htbp]
\centering
\includegraphics[scale=0.80]{editaveis/figuras/turbina}
\caption[Seção de uma turbina eólica]{Seção de uma turbina eólica típica conectada à rede.\footnotemark}
\FloatBarrier
\label{secao_turbina_eolica}
\end{figure}
\footnotetext{Fonte: USP, 2005 }

Dentro da turbina eólica temos os seguintes subconjuntos: torre, rotor, nacele, caixa de multiplicação (transmissão),
gerador, mecanismos de controle, anemômetro, pás de rotor, biruta (sensor de direção). A torre é o elemento que sustenta
o rotor e a nacele na altura adequada para o funcionamento da turbina eólica. Esse item é de elevada contribuição no custo 
inicial do sistema. O rotor é a componente onde as pás são conectadas e que realiza a transformação de energia cinética dos
ventos em energia mecânica de rotação \cite{rossiEtAl}.

Nacele é um compartimento localizado no alto da torre que abrigam mecanismos do gerador (freios, caixa multiplicadora,
embreagens, sistemas hidráulico, etc). Usaremos a Nacele para também abrigar os mecanismos de obtenção de água. 
	
Caixa multiplicadora (transmissão) é o mecanismo que transmite a energia mecânica do eixo do rotar ao eixo do gerador.
Gerador é o converterá energia mecânica do eixo em energia elétrica. Mecanismos de controle são os que supervisionam a 
velocidade média nominal que ocorre com maior frequência durante um determinado período. Anemômetro tem a função de medir
a intensidade e a velocidade dos ventos. Pás do rotor captam o vento e converte sua potência ao centro do rotor.
Biruta é um conjunto de sensores que captam a direção do vento \cite{rossiEtAl}.
	
Uma das componentes que se tem muito estudo é a pá rotativa. Ela pode ser feita com os seguintes materiais: 
madeira, aço, alumínio, fibra de vidro com resina poliéster, fibra de vidro com fibra de carbono, madeira com epóxi,
fibra de carbono. A escolha do material vai depender da escolha do perfil aerodinâmico, que será estudado posteriormente.
\cite{portalEnergia}.

\begin{figure}[!htbp]
\centering
\includegraphics[scale=0.80]{editaveis/figuras/pa}
\caption[Seção transversal de uma pá feita de fibra de vidro]{Seção transversal de uma pá feita de fibra de vidro\footnotemark}
\FloatBarrier
\label{secao_transversal_pa}
\end{figure}
\footnotetext{Fonte: \cite{usp}}

Como pode ser visto as fibras são colocadas estruturalmente nas principais direções de propagação das tensões, quando 
em operação. A fibra de carbono e ou Kevlar são atualmente os compostos mais avançados que podem ser utilizados em áreas
críticas (longarina da pá), mas tal material possui preços muito elevados \cite{barrosVarela}.  

Em relação ao suporte estrutural, ou torre, nas turbinas eólicas elas podem ser do tipo treliçadas, tubular e estaiada,
no entanto, para a Eolewater as estruturas mais comuns são as duas últimas. As tores são constituídas de concreto e aço,
tendo o peso em torno de 40 toneladas e 50 metros de comprimento \cite{usp}.

\begin{figure}[!htb]
\centering
\includegraphics[scale=0.80]{editaveis/figuras/torre}
\caption[Tipos de torre]{Tipos de torre. Da esquerda para a direita: Treliçada, Tubular e Estaiada \footnotemark}
\FloatBarrier
\label{torre}
\end{figure}
\footnotetext{Fonte: \cite{usp}}

O modelo do dispositivo da Eolewater de gerar água por meio da energia eólica possui uma turbina WMS1000, de potência de30kW.
O tempo de vida proposto para esse mecanismo é de 20 anos, dependendo das condições em que o motor é submetido ele pode gerar
até 1200 litros de água por dia (mais informações na tabela abaixo). Como o dispositivo não necessita de quaisquer outros 
recursos para operar há um impacto mínimo sobre o meio em que é colocado \cite{renewable}.

\begin{figure}[!htbp]
\centering
\includegraphics[scale=0.3]{editaveis/figuras/condicoes}
\caption[Tabela de condições de umidade e temperatura para o rendimento de água]{Tabela de condições de umidade e temperatura para o rendimento de água \footnotemark}
\FloatBarrier
\label{condicoes}
\end{figure}
\footnotetext{Fonte: \cite{renewable}}

Essa tecnologia possui um controle de pitch centrífuga para regular a velocidade do motor, tem um sistema de travagem
rotor mecânica e elétrica, o qual evita danos nas lâminas giratórias (pás), ainda, contém um mastro de inclinação que 
integra a ação dos cilindros telescópicos com capacidade de empuxo de 115 toneladas. Deve-se destacar que os componentes 
que entram em contato com a água são feitos de uma liga de aço inoxidável especial que operará sem risco de corrosão 
\cite{eole}.

Uma tecnologia como essa segundo o site da Indústria Eólica uma turbina de vento abaixo de 100 kW vai custa por volta
de US \$ 3.000 a US \$ 5.000  por quilowatt de capacidade. Portanto, levando em conta as especificações técnicas do Turbine
WMS1000 abaixo a tecnologia é eficiente, mas cara \cite{renewable}.
	
\begin{figure}[!htbp]
\centering
\includegraphics[scale=0.8]{editaveis/figuras/especificacao}
\caption[Especificação Técnica do Turbine WMS1000]{Especificação Técnica do Turbine WMS1000 Vento \footnotemark}
\FloatBarrier
\label{Especificacoes}
\end{figure}
\footnotetext{Fonte: \cite{renewable}}
 
A outra tecnologia, Warawater, por sua vez é uma tecnologia muito barata se comparada com a mencionada anterior. 
Essa custa cerca de US\$ 500 e pode ser construída em menos de uma semana com uma equipe de quatro pessoas e materiais 
existente localmente \cite{warkawater}.

Os materiais necessários para a sua construção de sua estrutura são: recipiente de coleta, bambu e um revestimento
interno de plástico reciclado (rede). Sua torre possui em média 10 metros de altura, com 60 Kg e pode suprir até 100 litros
de água por dia \cite{warkawater2}.

\begin{figure}[!htbp]
\centering
\includegraphics[scale=0.3]{editaveis/figuras/warkawater}
\caption[Utilização da tecnologia warkawater por uma população carente]{Utilização da tecnologia warkawater por uma população carente  \footnotemark}
\FloatBarrier
\label{Especificacoes}
\end{figure}
\footnotetext{Fonte: \cite{warkawater}}

% \bibliographystyle{abnt-alf}
% \bibliography{bibliografia}

% \end{document}
    
    \pagebreak
    \subsection{Matriz energética}
    
      Na contemporaneidade, quando se fala de geração de energia, em qualquer local do mundo a primeira questão a ser levantada
é a de maior distribuição possível juntamente com a maior viabilidade econômica envolvida. Estes foram dois parâmetros 
primordiais para se escolher a energia renovável para ser implantada dentro do projeto.

A Matriz Energética do Projeto foi divida em três áreas apresentadas a seguir.

  \subsubsection{Fontes Energéticas Referenciais}
    
    No decorrer do período de elaboração do escopo do projeto, várias possíveis soluções foram levantadas e, em seguida, 
    discutidas com o intuito de se chegar em um sistema que atendesse da melhor formar os requisitos iniciais do projeto.
    Dentre todas as opções disponíveis, foram pré-selecionadas três que, em princípio, se destacaram. São elas:
    
    \begin{itemize}
      \item \textbf{Eole Water}: uma turbina eólica autossuficiente que capta a água a partir de um sistema de refrigeração
	que faz com q a água no ar se condense ao entrar em contato com suas pás.
     
      \item \textbf{Maxwater}: um moinho de vento vertical, com um sistema muito semelhante ao primeiro, mas que possui custo,
	produção diferenciados.
      
      \item \textbf{Warkawater}:torre feita de bambu e que é forrada por dentro com uma malha plástica, que retém gotículas
	de orvalho. Esse sistema é passivo e não necessita de energia para produção de água já que funciona de forma passiva.
    \end{itemize}
  
    Dois requisitos muitos importantes para a escolha do sistema foram a necessidade do uso de uma fonte energética renovável
    e de que a água fosse analisada em tempo real. Observando o segundo requisito se percebe que existe a necessidade da
    implementação de sensores e de um sistema que envie todos os dados recolhidos por tais sensores. Nesse contexto pensou-se 
    primeiramente no uso de painéis solares fotovoltáicos, em um segundo momento, levando em consideração a região propícia à 
    implementação de um parque eólico e escolha do sistema Eolewater, decidiu-se por utilizar o excedente de energia gerado 
    pela turbina eólica para suprir as necessidades dos sistemas embarcados.
  
    A premissa do uso do potencial dos ventos para geração de trabalho data de milhares de anos atrás, onde essas tecnologias
    eram usadas principalmente para o bombeamento de água e para moagem de grãos. As primeiras tentativas do uso da energia
    eólica para geração de eletricidade foram no século XIX, ma só na década de 1970 é que essa tecnologia foi aplicada em
    escala comercial.
    
    A avaliação do potencial eólico de uma região requer trabalhos sistemáticos de coleta e análise de dados sobre a velocidade
    e o regime de ventos. Geralmente, uma avaliação rigorosa requer levantamentos específicos, mas dados coletados em 
    aeroportos, estações meteorológicas e outras aplicações similares podem fornecer uma primeira estimativa do potencial
    bruto ou teórico de aproveitamento da energia eólica.
    
    Embora ainda haja divergências entre especialistas e instituições na estimativa do potencial eólico brasileiro, 
    vários estudos indicam valores extremamente consideráveis. Até poucos anos, as estimativas eram da ordem de 20.000 MW.
    Hoje a maioria dos estudos indica valores maiores que 60.000 MW.
    
    As primeiras turbinas eólicas de uso comercial tinham a capacidade de produção elétrica entre 10kW e 50kW, já
    as máquinas de grande porte atuais tem uma potência superior a 1Mw. 
    
  \subsubsection{Técnicas e métodos de conversão e armazenamento}
    
    Um dos desafios do projeto foi estabelecer a demanda de água para a região selecionada e a partir disso dimensionar a
    estrutura física necessária de acordo com os requisitos. Dentre os modelos analisados pela equipe, a turbina de vento 
    mais favorável foi a Eolewater, modelo WMS100 Wind Turbine da empresa EoleWater, uma turbina de eixo horizontal que
    apresenta resultados satisfatórios à proposta e servirá de base de estudo para a implementação do projeto na área 
    energética.
    
    Apesar dos avanços nessa área, a energia eólica não possui uma capacidade de produção muito grande, por isso deve-se
    focar no requisito eficiência de conversão. Algumas formas de aperfeiçoar a produção estão relacionadas à dimensão do 
    gerador, aerodinâmica, materiais utilizados, projeção da estrutura e logística.
    
    Partindo dessa necessidade, o gerador exerce função primordial no funcionamento da turbina, pois é ele quem converte a
    energia mecânica em energia elétrica que alimenta todo o sistema. Dessa forma, temos o seguinte esquema de funcionamento:
    Os ventos fazem com que as pás do rotor girem, consequentemente girando o rotor. Este, por sua vez, converte a energia 
    cinética dos ventos em energia mecânica de rotação. Esse conjunto conectado a um eixo transmite essa rotação para o
    gerador. O gerador finalmente converte a energia mecânica em energia elétrica que alimenta todos os outros componentes
    eletrônicos (controle) e mecânicos da turbina.
    
    A produção de energia elétrica estimada da turbina é de aproximadamente 30kW (a produção real depende do diâmetro do rotor,
    rendimento do sistema, velocidade dos ventos, condições climáticas da região). Essa energia produzida alimentará os
    componentes da turbina e a energia remanescente será utilizada pela estação de controle e armazenada em baterias para
    que seja utilizada em situações emergenciais.
    
    Dessa forma, temos a seguinte distribuição energética:
    
    \begin{figure}[!ht]
    \centering
    \includegraphics[scale=0.45]{editaveis/figuras/distribuicao_energetica}
    \caption{Distribuição energética}
    \label{distribuicao_energetica}
    \end{figure}
    
    \begin{enumerate}
     \item Geração de eletricidade (aproximadamente 30kW);
     \item Alimentação do compressor e componentes eletrônicos da turbina;
     \item Direcionamento a torre de controle;
     \item Distribuição no quadro de controle;
     \item Alimentação elétrica do maquinário e estrutura básica;
     \item Direcionamento de energia remanescente para as baterias emergenciais.
    \end{enumerate}
    
  \subsubsection{Eficiência Energética}
  
    A turbina eólica, ou aerogerador, é uma máquina capaz de absorver a potência cinética do vento por meio de um rotor
    aerodinâmico, convertendo este movimento em potência mecânica de eixo (torque vs rotação),  que é transformada em
    potência elétrica (tensão \textit{vs} corrente) por intermédio de um gerador elétrico.
    
    A parte estrutural de geração de energia de uma turbina é constituída por um rotor e pela torre que a sustenta,
    pela transmissão/multiplicação e pelo conversor. Ela somente consegue extrair energia através da energia cinética do 
    ar que passa pelo interior da área interceptada pelas pás rotativas.  
    
    Energia eólica provém da radiação solar. Se considerarmos que, aproximadamente, 2\% da energia solar que a Terra absorve,
    é convertida em energia cinética dos ventos, teremos uma estimativa da energia total disponível dos ventos ao redor
    do planeta.
    
    Os ventos (massas de ar em movimento),são influenciados por diferentes aspectos dentre os quais se destacam a rugosidade
    do solo, os obstáculos e o relevo da região, e possuem energia cinética que pode ser aproveitada com o uso de aerogeradores.
    
    Dessa forma, a energia cinética $E_C$, contida em uma amostra de volume de ar, $A$ x $\delta x$, com densidade do ar $\rho$, 
    movendo-se com uma velocidade, $v$,  onde $A$ é uma unidade de área perpendicular à direção dos ventos e $\delta x$ é paralelo 
    à direção dos  ventos, é dada por:
    
    $$ E_C = \frac{Mv^2}{2} = \frac{\rho A {\delta x} v^2}{2} $$
    
    A primeira vista, imagina-se que a máxima energia retirada dos ventos por uma turbina eólica é a energia cinética dos
    ventos que atravessam um círculo formado pela área das pás. Contudo, o próprio vento possui energia cinética na esteira
    do rotor, fazendo com que nem toda energia seja retirada. Segundo uma teoria criada por Betz* (para um modelo ideal), 
    a eficiência aerodinâmica do rotos estaria limitada a 59,3\% da energia presente nos ventos.
    
    O rotor é o primeiro estágio de conversão da energia do vento em eletricidade. Os estágios seguintes são a transmissão
    e o próprio gerador, que adéqua às velocidades de rotação e converte a energia mecânica em energia elétrica, 
    respectivamente.
    
    Em média, a eficiência de conversão dos modernos aerogeradores está dividida como ilustra a Figura ~\ref{eficiencia_conversao_aerogeradores}.
    
    \begin{figure}[!ht]
    \centering
    \includegraphics[scale=0.7]{editaveis/figuras/eficiencia_conversao_aerogeradores}
    \caption{Eficiência de conversão dos aerogeradores modernos}
    \label{eficiencia_conversao_aerogeradores}
    \end{figure}
    
    Na contemporaneidade, os parâmetros de rotores utilizados nos aerogeradores modernos são de duas ou três pás.
    Isso graças à relação de potência extraída por área de varredura do rotor (muito superior ao rotor multipás) para 
    velocidades mais elevadas. Tais características são aceitáveis em sistemas de geração de eletricidade, porém se tornam
    inviáveis em sistemas que requeiram altos momentos de força e/ou carga variável.
    
    Rotores modernos, com mais de três pás, são usados somente quando há necessidade de um grande torque de partida, ou seja, 
    basicamente, bombeamento mecânico de água. Aerodinamicamente, no entanto, grande número de pás e alto torque de partida,
    diminuem a eficiência do sistema. 

    Sendo assim, o desenvolvimento de pás para aerogeradores deve ser resultante da integração entre estes fatores.
    Com o estágio atual da tecnologia, a dificuldade de fabricação não reside na aerodinâmica, mas sim na construção
    e resistência dos materiais que compõem as pás, que devem responder a diferentes exigências da máquina eólica,
    além de ser necessário que sejam resistentes, rígidos, leves e de baixo custo.
    
    As perdas de transmissão relacionam-se diretamente ao atrito que existe entre as engrenagens.
    Em velocidades de giro fixas, as perdas variam pouco, podendo-se assumir que são uma porcentagem fixa
    da potência nominal. Esta porcentagem real depende da qualidade da transmissão, mas um valor razoável
    pode ser em torno de 2\% da potência em cada etapa de engrenamento. Como a transmissão consome certa quantidade de
    energia, as perdas podem ser consideráveis em baixas potências, já que o rendimento nestes casos é menor.
    
    Para que a geração de eletricidade a partir do movimento do ar seja plausível, técnica e economicamente, alguns fatores
    ganham relevância. A velocidade dos ventos é o fator mais crítico na determinação da energia que será obtida de um
    aerogerador, e também seu custo. Outros fatores seriam:
    \begin{itemize}
     \item \textbf{Topografia}: o ar é mais frio durante a noite e tende a ocupar regiões próximas ao solo, além de produzir
	pouca quantidade de vento. Por isso devem ser escolhidas áreas mais elevadas. Para a escolha dessas áreas devem ser
	observadas também: facilidade de locomoção até a instalação, proximidade ao ponto de consumo, espaço necessário
	para manutenções e evitar áreas muito frias, a fim de não danificar o aerogerador.
	
     \item \textbf{Barreiras Naturais}: prédios, árvores, plantações e construções elevadas que podem diminuir
	a velocidade do vento e turbulência, danificando o equipamento.
      
     \item \textbf{Superfície}: quanto mais acidentado o terreno (maior rugosidade), com plantações, construções, árvores, entre outros,
	mais alta a torre deve ser.
      
     \textit{Obs.:} Quando não especificada a altura na qual ocorreu a medição da velocidade dos ventos,
	consideramos a altura padrão internacional de 10 metros acima do solo, ou a altura em que cada gerador está operando. 
     
    \end{itemize}
    
    \pagebreak
    \subsection{Sistema eletrônico de monitoramento e controle da qualidade da água}
      
      \subsubsection{Parâmetros de qualidade da água}
        
        A poluição e uso irracional e inadequado da água comprometem a disponibilidade de água em padrões de qualidade apropriados 
ao uso da geração atual e das que estão por vir. Dessa maneira, faz-se necessário o monitoramento das águas que serão
destinadas ao abastecimento da população.

Os indicadores ambientais surgiram a partir da crescente preocupação socioambiental referente ao desenvolvimento.
Os indicadores tornaram-se fundamentais no processo decisório das políticas públicas e no acompanhamento de seus efeitos.
Esta dupla vertente apresenta-se como um desafio permanente de gerar indicadores e índices que tratem um número cada vez 
maior de informações, de forma sistemática e acessível, para os tomadores de decisão \cite{cetesbIndiceQualidadeH2O}.

Existe um indicador chamado Índice de Qualidade das Águas (IQA), o qual foi desenvolvido pela 
Companhia de Tecnologia de Saneamento Ambiental do Estado de São Paulo (CETESB) por meio da consulta à especialistas 
em qualidade da água, os quais apontaram parâmetros a serem avaliados acompanhados de seus respectivos pesos e a 
condição com que se apresenta cada parâmetro, de acordo com uma lista de valores. Desses parâmetros, os 9 (nove) 
principais foram selecionados. Para esses, foram estabelecidas curvas de variação de qualidade das águas de acordo 
com o estado ou a condição de cada parâmetro. Essas curvas são apresentadas na Figura ~\ref{graficos_parametros}.

A maior parte dos parâmetros levados em consideração no cálculo do IQA são indicadores de contaminação propiciada
pela emissão de esgotos domésticos.

O IQA é o principal índice de qualidade da água utilizado em todo o país. A finalidade de tal índice é transformar 
diversos dados relacionados à qualidade da água em um único número, que representa o nível de qualidade da água.

De acordo com a Agência Nacional de Águas \cite{anaGovIndicadores}, como desvantagens do uso do IQA, deve-se citar o fato de
o índice não contemplar vários parâmetros importantes para o abastecimento público, tais como:

\begin{itemize}
 \item Substâncias tóxicas (ex: metais pesados, pesticidas, compostos orgânicos);
 \item Protozoários patogênicos;
 \item Substâncias que interferem nas propriedades organolépticas da água.
\end{itemize}

Os nove parâmetros com seus respectivos pesos são apresentados a seguir:

\begin{figure}[!h]
\centering
\includegraphics[scale=0.8]{editaveis/figuras/tabela_parametros_unidade}
\FloatBarrier
\label{tabela_parametros_unidade}
\caption[Parâmetros de qualidade da água do índice IQA e seus respectivos pesos]
  {Parâmetros de qualidade da água do índice IQA e seus respectivos pesos. \footnotemark}
\end{figure}
\footnotetext{Fonte: \cite{anaGovIndicadores}; \cite{cetesbIndiceQualidadeH2O}}

Além de seu peso ($w$), cada parâmetro possui um valor de qualidade ($q$), o qual é obtido do respectivo gráfico de qualidade
em função de sua concentração ou medida. Observe a figura abaixo:

\begin{figure}[!h]
\centering
\includegraphics[scale=0.5]{editaveis/figuras/graficos_parametros}
\label{graficos_parametros}
\caption[Curvas de variação de qualidade das águas]{Curvas de variação de qualidade das águas.\footnotemark}
\end{figure}
\FloatBarrier
\footnotetext{Fonte: \cite{cetesbIndiceQualidadeH2O}}

O cálculo do IQA é dado da seguinte forma:

$$ IQA = \prod_{i = 0}^{n} {q_i}^{w_i} $$

\begin{center}
Fonte: \cite{anaGovIndicadores}
\end{center}

Onde:

\begin{itemize}
 \item $IQA$: Índice de Qualidade das Águas, representado por um número de 0 a 100;
 \item $q_i$: qualidade do i-ésimo parâmetro, representado por um número entre 0 e 100, obtido do respectivo gráfico de qualidade, em função de sua concentração ou medida;
 \item $w_i$: peso correspondente ao i-ésimo parâmetro fixado em função da sua importância para a conformação global da qualidade, isto é,  um número de 0 a 1, de forma que:
 
    $$ \sum_{i=1}^n w_i = 1 $$
    Sendo $n$ o número de parâmetros que entram no cálculo do IQA.
\end{itemize}

A tabela a seguir apresenta a classificação da água de acordo com o valor do IQA:

\begin{figure}[!h]
\centering
\includegraphics[scale=0.7]{editaveis/figuras/tabela_classificacao_IQA}
\label{tabela_classificacao_IQA}
\caption[Classificação da água de acordo com o IQA]{Classificação da água de acordo com o IQA.\footnotemark}
\end{figure}
\FloatBarrier
\footnotetext{Fonte: \cite{cetesbIndiceQualidadeH2O}}

Segundo a \cite{anaGovIndicadores}, a descrição dos parâmetros que compõe o IQA pode ser efetuada da seguinte maneira:

\begin{itemize}
 \item \textbf{Oxigênio dissolvido (OD)}: é um fator limitante para a manutenção da vida aquática e de processos de 
    autodepuração em sistemas aquáticos naturais e estações de tratamento de esgotos \cite{cetesbOxigenioDissolvido}.
    Águas limpas têm concentração de oxigênio, na maioria das vezes, maior ou igual a 5 mg/L. No que se refere aos processos que
    contribuem para a introdução do oxigênio na água, pode-se citar, além da fotossíntese, determinados processos físicos 
    os quais dependem das características hidráulicas dos corpos d’água, por exemplo: velocidade da água.
    
 \item \textbf{Coliformes termotolerantes}: São bactérias não patogênicas que ocorrem no trato intestinal de animais de sangue
    quente e são indicadoras de poluição da água causada pelos esgotos domésticos. Quando há grande quantidade dessas bactérias
    na água, há possibilidade de existir microorganismos capazes de transmitir doenças de veiculação hídrica.
    
 \item \textbf{Potencial Hidrogeniônico (pH)}: O pH (medida que determina a acidez ou basicidade de uma mistura) é capaz de
    alterar o metabolismo de diversas espécies aquáticas. Dessa forma, o CONAMA estabelece que o pH deve estar entre 6 e 9 
    para a proteção da vida aquática. O pH também afeta a intensidade do efeito das substâncias tóxicas nos seres aquáticos,
    tais quais os metais pesados.
    
 \item \textbf{Demanda Bioquímica de Oxigênio (DBO5,20)}: representa a quantidade necessária de oxigênio para oxidar a matéria
    orgânica aquática por meio da decomposição microbiana aeróbia. A ocorrência de altos valores de DBO5,20 acarreta na diminuição
    dos valores de oxigênio dissolvido (OD). A sigla DBO5,20 equivale à quantidade de oxigênio consumido em 5 dias a uma temperatura de 20C.
    A causa de altos valores de DBO5,20 numa amostra d’água é geralmente devido à emissão de dejetos de origem orgânica,
    principalmente esgotos domésticos.
    
 \item \textbf{Temperatura da água}: influencia em diversos parâmetros físico-químicos da água (ex: tensão superficial e a
    viscosidade) além de afetar o crescimento e reprodução de espécies aquáticas.
    
 \item \textbf{Nitrogênio total}: O nitrogênio pode ocorrer nos corpos d’água em diversas formas (orgânica, amoniacal, nitrito 
    e nitrato). Os nitratos são tóxicos aos humanos.
    Além disso, como os compostos de nitrogênio nutrem os processos biológicos, a grande concentração de nitrogênio 
    (juntamente com outros nutrientes) em corpos d’água causa a eutrofização, que prejudica o abastecimento público e
    a preservação da vida aquática.
    A principal fonte de nitrogênio em corpos d’água é o lançamento de esgotos sanitários e efluentes industriais.
    Em áreas agrícolas, o escoamento da água das chuvas em solos que receberam fertilizantes também é uma fonte de nitrogênio,
    assim como a drenagem de águas pluviais em áreas urbanas.
    
 \item \textbf{Fósforo total}: Assim como o nitrogênio, o fósforo (em excesso) causa a eutrofização das águas.
    As principais fontes de fósforo são os esgotos domésticos, pelo fato de eles terem a presença de detergentes
    superfostadados e das próprias fezes. Além disso, a drenagem pluvial de áreas agrícolas e urbanas também é uma fonte 
    significativa. Entre os efluentes industriais destacam-se os das indústrias de fertilizantes, alimentícias, laticínios, 
    frigoríficos e abatedouros.
    
 \item \textbf{Turbidez}: indica o grau de atenuação que um feixe de luz sofre ao atravessar a água. Tal atenuação é devida à
    absorção e espalhamento de luz causada pelos sólidos em suspensão. 
    A principal fonte de turbidez é a erosão dos solos (chuvas trazem os corpos sólidos).
    Porém, além desta, é possível citar as atividades de mineração, bem como o lançamento de esgotos e de efluentes industriais,
    como causadoras da turbidez nas águas.
    
 \item \textbf{Resíduo total}: é a matéria a qual permanece após a evaporação, secagem ou calcinação da amostra de água durante um 
    determinado tempo e temperatura.    
\end{itemize}

Existe uma empresa chamada Clean Environment Brasil \cite{cleanEnvironmentBrasil}, que fornece equipamentos de monitoramento e controle da água à ANA 
(Agência Nacional de Águas). Eles produzem várias sondas multiparamétricas (conseguem monitorar vários parâmetros em uma sósonda).

Seria interessante utilizar a sonda YSI EXO, que coleta dados de 6 sensores substituíveis, pois ela também é usada pela ANA para determinar a qualidade da água.

As opções de sensores para esse tipo de sonda são:

\begin{itemize}
 \item Oxigênio dissolvido;
 \item Matéria Orgânica Dissolvida (fluorescência);
 \item pH ou pH/ORP;
 \item Profundidade (que já é integrado);
 \item Algas totais (canal duplo para clorofila e algas azuis/verdes);
 \item Turbidez.
\end{itemize}

Assim, com apenas um equipamento relativamente leve (3,6 kg com baterias e sensores instalados) de pequenas
dimensões (phi = 7,62cm e comprimento = 71,10cm) é possível monitorar a maioria dos parâmetros do índice IQA \cite{cleanEnvironmentBrasil}.







        
      \subsubsection{Projeto eletrônico e de controle da planta}
        
        \documentclass[12pt,openright,oneside,a4paper,brazil]{abntex2}
\usepackage[utf8]{inputenc}
\counterwithout{section}{section}
\counterwithout{figure}{chapter}
\counterwithout{table}{chapter}
\setlength{\parindent}{1.3cm}
\usepackage{indentfirst}
\setlength{\parskip}{0.2cm}
\usepackage[bottom=2cm,top=3cm,left=3cm,right=2cm]{geometry}
\usepackage{graphicx}
\graphicspath{{figuras/}}
\usepackage{placeins}
\usepackage{cite}
\usepackage{url}
\usepackage{breakurl}
\include{bibliografia}

\makeatletter
\setlength{\@fptop}{0pt}
\makeatother

\begin{document}

\textual
\begin{center}
 {\large Anemômetroultrassônico}\\[0.2cm]
 \end{center}
 
Existem vários tipos de anemômetros, mas entre eles o mais vantajoso por ter uma rápida resposta, boa exatidão nos dados e é possível ser implementado em ambientes com extremas temperaturas e pressão é o ultrassónico. Este sensor basicamente utiliza o som que é desencadeado pelo movimento das partículas que o compõe, estimando assim a velocidade do vento. 

O processo de recolhimento dos dados de velocidade e direção do vento é o seguinte,primeiramente com o sinal de excitação, transmite-o para um transistor-transmissor que posteriormente repassa para um transistor-transmissor um sinal ultrassônico, neste processo de transição o sinal sofre alterações, desse modo no transistor-transmissor o sinal é processado e analisado para entregar as informações requeridas. O arranjo do equipamento é essencial para o recolhimento dos dados, por causa de seu formato, que contémum tetraedro com três transdutores em cada extremidade, sendo possível detectar a direção do vento e assim sua velocidade. 

Abaixo especificações técnicas de um anemômetro ultrassônico (Vaisala WINDCAP® WMT700):

{\large Velocidade do vento}\\
Faixa de medição	- 

701	0...40 m/s

702	0...65 m/s

703	0...75 m/s

Precisão	-	+/- 0,2 m/s ou 3 \% da leitura, o que for maior

Limite inicial - 0,01 m/s

Resolução - 0,01 m/s\\

{\large Direção do vento}

Faixa de medição - 0 ... $300\,^{\circ}\mathrm{C}$

Precisão - +/- $2\,^{\circ}\mathrm{C}$

Limite inicial - 0,1 m/s

Resolução - $1\,^{\circ}\mathrm{C}$,

{\large Saídas}

Taxa de transmissão - 300, 1200, 2400, 4800, 9600, 19200, 38400, 57600, 115200

Médias disponíveis -  máx. 3600 s

Intervalo de atualização de leitura - máx. 4 Hz

Unidades

Saídas digitais - m/s, nós, mph, km/h

Saídas analógicas - V, mA, Hz

Temperatura virtual - graus Celsius

{\large Geral}

Temperatura de operação	- -10 ... +60 ou -40 ... +60 ou -55 ... $+70\,^{\circ}\mathrm{C}$,
\newpage

\begin{center}
 {\large Sensor de pH}\\[0.2cm]
 \end{center}
 
{\large Especificação do sensor de pH}\\
 O sensor de pH da água é de extrema importância para o projeto da “planta de abastecimento de água potável através da umidade do ar” pois monitora um dos índices de qualidade da água especificados pela ANA(agência nacional de águas). 
 
	Para aplicações industriais, o método de medição de pH mais empregado é o eletrodo de vidro(Solé, 1979).Os eletrodos de pHpossuem basicamente o mesmo funcionamento que as baterias: transferem uma tensão mínima que poderá ser detectada por um medidor ou um regulador de pH. A diferençaé que os eletrodos de pH não produzem tensão de forma contínua, a não ser quando são introduzidos num líquido

{\large Calibração de sensores de pH}\\
O período de calibração de um sensor de pH depende do contexto em que o sensor será aplicado e do tipo de sensor que será utilizado. O tipo de sensor a ser utilizado pode variar de acordo com parâmetros como temperatura. O sensor de pH pode vir acoplado a um sensor de condutividade, entre outros. Apesar de existirem vários tipos, todo eletrodo de pH requer calibração periódica. Uma calibração em dois pontos caracteriza um eletrodo com um medidor de pH específico.

{\large Sensor ORP}\\
O sensor ORP é similar ao sensor pH quanto ao seu funcionamento, porém, ao invés de seu eletrodo ser envolvido por vidro, é geralmente envolvido por platina ou ouro, devido ao fato de esses metais não interferirem nas reações químicas. Nos sensores ORP, o gel interno recebe a corrente elétrica provinda do meio e a transmite ao interior do sensor. Posteriormente, o fio e prata pura transmite a corrente positiva ao para o cabo de conexão, que leva o sinal recebido ao controlador.
\newpage

\begin{center}
 {\large Métodos de Processamento de Sinais}\\
 \end{center}
 O processamento de sinais será realizado por meio de um sistema de sensoriamento do local onde se encontrará a planta de abastecimento. Os sensores farão a varredura do local, coletando dados sobre a qualidade da água já tratada provinda da umidade do ar. Os dados a serem coletados estão de acordo com o Índice de Qualidade da Água(IQA). Para cada fator a ser monitorado de acordo com o IQA existirá um sensor relacionado. 
 
	Como foi dito anteriormente, almeja-se utilizar a sonda YSI EXO, que coleta dados de 6 sensores substituíveis e que estão de acordo com o Índice de Qualidade da Água(IQA). A sonda YSI EXO é comandada por um software embarcado. Além disso, a própria sonda, além de possuir um sistema de sensoriamento, processa suas próprias informações, enviando-as à uma central computadorizada.
\newpage
\begin{center}
 {\large Especificações do Sensor de Umidade Relativa e Temperatura do Ar}\\
 \end{center}
{\large Características:}\\
 \begin{itemize}
\item Processamento digital de sinal
\item Umidade relativa e temperatura do ar em um único sensor
\item Alta acurácia de leitura e linearização
\item Sinais de saída condicionados
\item Excelente estabilidade de longo termo
\item Baixo tempo de resposta
\item 100 \% intercambialidade
 \end{itemize}
{\large Construção:}\\
O invólucro do sensor e do circuito eletrônico é moldado em plástico injetado e estabilizado para U.V. O invólucro do circuito é selado, sendo à prova de respingos e poeira. Todas as conexões elétricas são feitas através de um conector selado na base do sensor. O sensor opera de 0 a 100\% umidade relativa. Os transdutores internos não sofrem danos mesmo com condensação. O circuito digital do sensor realiza a compensação de temperatura e a linearização do sinal de saída. O armazenamento dos dados de calibração do sensor em memória interna não volátil fornece uma maior acurácia nas leituras.
\begin{figure}
\centering
\includegraphics[scale=1]{umidade}
\end{figure}
\FloatBarrier
\newpage
\begin{center}
 {\large Sensores de Processamento de Sinais para Umidade do Ar}\\
 \end{center}
 O uso de equipamentos que usa o processamento de sinais é de suma importância para extrair informações do clima para a aplicação de retirada de água a partir da umidade do ar. O uso de um sensor de umidade relativa e temperatura do ar combinam sensores de umidade e temperatura de alta precisão em um único instrumento. Ambos os sensores fornecem sinais de saída diretamente proporcionais à umidade relativa e temperatura. O  processamento digital de sinais embutido no sensor garante uma precisa linearização dos sinais de saída em toda a faixa de operação e total intercambialidade do sensor. 
 
Operação: Para melhor desempenho, ele deve ser posicionado em local com boa circulação de ar e longe de grandes construções, como  edifícios ou muros. A membrana protetora do sensor de umidade relativa deve ser regularmente limpa. Sempre que possível, a limpeza deve ser feita sem a remoção do filtro. Em aplicações que envolvem contínuo contato com água ou poeira grossa, um filtro de espuma  opcional esta disponível. A calibração do sensor de temperatura não é necessária. Para o sensor de umidade, a calibração pode ser feita a cada 6 a 12 meses para garantia de máximo desempenho.
 Características:
 \begin{itemize}
\item Processamento digital de sinal
\item Umidade relativa e temperatura do ar em um único sensor
\item Alta acurácia de leitura e linearização
\item Sinais de saída condicionados
\item Excelente estabilidade de longo termo
\item Baixo tempo de resposta
\item 100 \% intercambialidade
 \end{itemize}
 \newpage
\begin{center}
 {\large Parâmetros da qualidade da água}\\
 \end{center}
Existe um indicador chamado Índice de Qualidade das Águas (IQA), que é o principal índice de qualidade da água utilizado em todo o país. A finalidade do desenvolvimento de tal índice é analisar a qualidade da água após o tratamento da mesma.

Como desvantagens do uso do IQA, deve-se citar o fato de o índice não contemplar vários parâmetros importantes para o abastecimento público, tais como:
\begin{itemize}
\item Substâncias tóxicas (ex: metais pesados, pesticidas, compostos orgânicos);
\item Protozoários patogênicos
\item Substâncias que interferem nas propriedades organolépticas da água.

\end{itemize}

O IQA é composto por nove parâmetros, que têm seus respectivos pesos, os quais foram fixados de acordo com sua importância global na determinação da qualidade da água. Veja na tabela a seguir:
\begin{figure}[h]
\centering
\includegraphics[scale=0.8]{tabela}
\end{figure}
\FloatBarrier
Além de seu peso (w), cada parâmetro possui um valor de qualidade (q), obtido do respectivo gráfico de qualidade em função de sua concentração ou medida. Observe a figura abaixo:
\begin{figure}[h]
\centering
\includegraphics[scale=0.8]{grafico}
\end{figure}
\FloatBarrier

O cálculo do IQA é dado da seguinte forma:

\begin{figure}[h]
\centering
\includegraphics[scale=1]{eq1}
\end{figure}

Onde:
\begin{itemize}

\item IQA: Índice de Qualidade das Águas. Um número de 0 a 100;
\item qi: qualidade do i-ésimo parâmetro. Um número entre 0 e 100, obtido do respectivo gráfico de qualidade, em função de sua concentração ou medida (resultado da análise).
\item wi: peso correspondente ao i-ésimo parâmetro fixado em função da sua importância para a conformação global da qualidade, isto é,  um número de 0 a 1, de forma que:
\begin{figure}[h]
\centering
\includegraphics[scale=01]{eq2}
\end{figure}
\FloatBarrier
\end{itemize}

Sendo n o número de parâmetros que entram no cálculo do IQA.
A descrição dos parâmetros que compõe o IQA será efetuada a seguir.
\begin{itemize}
\item Oxigênio dissolvido (OD): é um fator limitante para a manutenção da vida aquática e de processos de autodepuração em sistemas aquáticos naturais e estações de tratamento de esgotos [2]. Águas limpas têm concentração de oxigênio, na maioria das vezes, maior ou igual a 5 mg/L. 
No que se refere aos processos que contribuem para a introdução do oxigênio na água, pode-se citar, além da fotossíntese, determinados processos físicos os quais dependem das características hidráulicas dos corpos d’água, por exemplo: velocidade da água.
\item Coliformes termotolerantes: São bactérias não patogênicas que ocorrem no trato intestinal de animais de sangue quente e são indicadoras de poluição da água causada pelos esgotos domésticos. Quando há grande quantidade dessas bactérias na água, há a possibilidade de existir microorganismos capazes de transmitir doenças de veiculação hídrica[1].
\item Potencial Hidrogeniônico (pH): O pH (medida que determina a acidez ou basicidade de uma mistura) é capaz de alterar o metabolismo de diversas espécies aquáticas.Dessa forma, o CONAMA estabelece que o pH deve estar entre 6 e 9 para a proteção da vida aquática.
\item Demanda Bioquímica de Oxigênio (DBO5,20): representa a quantidade necessária de oxigênio para oxidar a matéria orgânica aquática por meio da decomposição microbiana aeróbia. A ocorrência de altos valores de DBO5,20 acarreta na diminuição   dos valores de oxigênio dissolvido (OD). A sigla DBO5,20 equivale à quantidade de oxigênio consumido em 5 dias a uma temperatura de 20C.
\item Temperatura da água: influencia em diversos parâmetros físico-químicos da água (ex: tensão superficial e a viscosidade) além de afetar o crescimento e reprodução de espécies aquáticas.
\item Nitrogênio total: O nitrogênio pode ocorrer nos corpos d’água em diversas formas (orgânica, amoniacal, nitrito e nitrato). Os nitratos são tóxicos aos humanos.
Além disso, como os compostos de nitrogênio nutrem os processos biológicos, a grande concentração de nitrogênio (juntamente com outros nutrientes) em corpos d’água causa a eutrofização, que prejudica o abastecimento público e a preservação da vida aquática. A principal fonte de nitrogênio em corpos d’água é o lançamento de esgotos sanitários e efluentes industriais.Em áreas agrícolas, o escoamento da água das chuvas em solos que receberam fertilizantes também é uma fonte de nitrogênio, assim como a drenagem de águas pluviais em áreas urbanas[1].
\item Fósforo total: Assim como o nitrogênio, o fósforo (em excesso) causa a eutrofização das águas. As principais fontes de fósforo são os esgotos domésticos, pelo fato de eles terem a presença de detergentes superfostadados e das próprias fezes. Além disso, a drenagem pluvial de áreas agrícolas e urbanas também é uma fonte significativa. Entre os efluentes industriais destacam-se os das indústrias de fertilizantes, alimentícias, laticínios, frigoríficos e abatedouros [1].
\item Turbidez: indica o grau de atenuação que um feixe de luz sofre ao atravessar a água. Tal atenuação é devida à absorção e espalhamento de luz causada pelos sólidos em suspensão. A principal fonte de turbidez é a erosão dos solos (chuvas trazem os corpos sólidos).
\item Resíduo total: é a matéria a qual permanece após a evaporação, secagem ou calcinação da amostra de água durante um determinado tempo e temperatura.

\end{itemize}
Existe uma empresa chamada Clean Environment Brasil, que fornece equipamentos de monitoramento e controle da água à ANA (Agência Nacional de Águas). Eles produzem várias sondas multiparamétricas (conseguem monitorar vários parâmetros em uma só sonda). 
Seria interessante utilizar a sonda YSI EXO, que coleta dados de 6 sensores substituíveis, pois ela também é usada pela ANA para determinar a qualidade da água.

As opções de sensores para esse tipo de sonda são:
\begin{itemize}
\item Condutividade e temperatura;
\item Oxigênio dissolvido;
\item Matéria Orgânica Dissolvida (fluorescência);
\item pH ou pH/ORP;
\item Profundidade (que já é integrado);
\item Algas totais (canal duplo para clorofila e algas azuis/verdes);
\item Turbidez [3].
\end{itemize}
Assim, com apenas um equipamento relativamente leve (3,6 kg com baterias e sensores instalados) de pequenas dimensões ($\Phi$ = 7,62cm e comprimento = 71,10cm) é possível monitorar a maioria dos parâmetros do índice IQA.
\newpage
Referências

[1] - http://portalpnqa.ana.gov.br/indicadores-indice-aguas.aspx

[2] - http://www.cetesb.sp.gov.br/mortandade/causas\_oxigenio.php

[3] - http://www.clean.com.br/site/ysi-exo-2/

[4] - http://www.inicepg.univap.br/cd/INIC\_2006/inic/inic/07/INIC0000904.ok.pdf

[5] - http://br.vaisala.com/br/products/windsensors/Pages/WMT700.aspx

\end{document}
    
    \pagebreak  
    \subsection{Sistema de Gestão da Informação do monitoramento da qualidade da água}
	
	
\subsubsection{Softwares de monitoramento de água}

  	Este tópico baseia-se no estudo de softwares existentes no mercado que possuem o intuito de testar a qualidade da água,
	para assim dar a segurança nescessária de qualidade ao consumo humano, de acordo a constituição vigente, e as normas
	de qualidade de saúde determinadas pela Agência Nacional de Águas (\citeauthor{anaGov}).
	
	Estes softwares poderão ser aplicados no sistema de armazenamento da água coletada através do ar, uma vez que as
	partículas de água podem conter substâncias químicas, o que torna a água inapropriada ao consumo. Devem ser utilizados
	com o auxilio de sensores para coleta dos dados. \\

% \begin{center}
\noindent
\textbf{Water Quality Analyser} (Fonte: \citeauthor{eWater})
% \end{center}	

	Com uma interface de usuário simples e direta possuindo foco na visualização de entradas e saídas de dados, este
	software ajuda a identificar a qualidade de água e simplificar o caminho para uma breve avaliação, através de
	ferramentas, que comparam a qualidade da água com os termos legais previamente definidos.
	
	Utiliza um gerenciamento de dados, para validação, vizualização e apresentação de relatórios, além de fornecer
	estatísticas de mudanças e aleatoriedade na qualidade da aguá, além de outros dados temporais.
	
	Este software foi desenvolvido pela colaboração, entre a empresa eWater CRC e o Departamento de Meio Ambiente
	e Gestão de Recursos de Queensland (QDERM), na Australia.
	
	A ferramenta é paga custando \$299 dollares, equivalente á R\$ 892,89 reais (cotação de 20/04/2015 - preço do
	dollar 2,98) pelo licenciamento de um ano, possuindo a possibilidade de teste gratuito pelo período de 30 dias.

	
\textbf{Prós:}
	
\begin{itemize}
  \item Muitas ferramentas;
  \item Interface simples;
  \item Recomendado para utilização em escala industrial;
 \end{itemize}
 
\textbf{Contras:}
	
\begin{itemize}
  \item Disponível apenas em inglês;
  \item Ferramenta paga;\\
 \end{itemize}

 
% \begin{center}
\noindent
\textbf{Logger Pro} (Fonte: \citeauthor{loggerPro})
% \end{center}	

	Utiliza coleta de dados em tempo real, suportando mais de 80 sensores e dispositivos diferentes, apresenta os dados
	em interface aceitando os sistemas operacionais Windows ou MAC.
	
	Custa \$339, equivalente à R\$1.012,34 (cotação de 20/04/2015 - preço do dollar 2,98). Possuindo uma versão Lite,
	gratuita e com menos funcionalidades.
	
\textbf{Prós:}
	
\begin{itemize}
  \item Multiuso, com muitas ferramentas;
  \item Coleta em tempo real;
  \item Captura vídeos;
  \item Desenha previsões utilizando gráficos de dados coletados anteriormente;
  \item Realiza análise estática dos dados;
  \item Possui gráficos avançados com ajuste de função;
  \item Transmição sem fio para dispositivos móveis;
 \end{itemize}
 
\textbf{Contras:}
	
\begin{itemize}
  \item Preço dos equipamentos;
  \item Feito para nível estudantil;\\
 \end{itemize}
 

% \begin{center}
\noindent
\textbf{AquaChem} (Fonte: \citeauthor{aquaChem})
% \end{center}	

	Ferramenta para análise cobrindo uma ampla gama de funções e cálculos utilizados para análise, interpretação 
	e comparação de dados da qualidade da aguá. Possuindo estatísticas e funções mais complexas, como matrizes de
	correlação e cálculos análiticos.
	
	Possui preço em dólares variando entre 1.355,75(R\$ 4.048,61) até 2.390,50 (R\$ 7.138,63).
	
\textbf{Prós:}
	
\begin{itemize}
  \item Cálculos geoquímicos;
  \item Análise simples e fácil de dados de qualidade da água;
  \item Cálculos estatísticos;
  \item Criado para escala comercial;
\end{itemize}
 
\textbf{Contras:}
	
\begin{itemize}
  \item Alto custo;
\end{itemize}  
  
% \subsubsection{Integração com o sistema eletrônico}
% 
%   Para analisar a qualidade da água são necessários sensores, os dados obtidos por esses sensores são transmitidos para
unidades com capacidade de processamento, para computadores. Os dados obtidos por esses sensores individualmente devem
ser analisados via software. Para efetuar a análise da qualidade da água, é necessária uma análise dos dados obtidos
pelo conjunto desses sensores, pois cada um indica uma ou algumas características específicas da água e não a qualidade
como um todo.

O software a ser utilizado depende basicamente do circuito eletrônico escolhido, tanto do computador, como dos tipos
e especificações dos sensores. Para medir a qualidade da água serão utilizados sensores capazes de avaliar basicamente
o nível de PH, oxigênio dissolvido, temperatura, íons dissolvidos, turbidez, resíduo total da água.

No software será definido o intervalo em que cada um dos indicadores de qualidade é aceitável para consumo humano de
acordo com a análise feita pelo grupo.  Em algumas situações, não seguiremos estritamente os padrões estipulados pelas 
agências reguladoras, desde que a escolha esteja dendro do estipulado por elas e traga algum benefício aos usuários. 
Por exemplo, o valor ideal de PH da água para consumo humano é de 6 a 9, porém a Agência Nacional de Vigilância Sanitária (ANVISA)
não estabelece nenhuma resolução indicando um valor mínimo para o pH.

Outro aspecto importante do software é que ele impeça a passagem da água, caso ela esteja fora dos padrões de qualidade
estipulados, até que o tratamento ocorra e seja finalizado. Para que isso ocorra,  o software fará uma uma comparação
entre cada um dos índices com os valores adequados e, caso a água não esteja dentro do padrão, o hardware enviará um 
sinal elétrico, comandando um mecanismo eletromecânico capaz de bloquear a passagem de água.

Além da análise da qualidade da água, será necessário sensores para analisar o nível da água tanto no reservatório
de distribuição de água, como no reservatório de armazenamento de água para tratamento. Os dados relacionados ao nível 
e qualidade da água serão transmitidos para um outro sistema que possibilita uma interação com o usuário, composto por
um software diferente.

Pretende-se utilizar a linguagem C/C++ para a elaboração do software relacionado ao controle da água. Estas linguagens
pode ser compiladas e utilizadas para uma grande parte dos sistemas embarcados. Para o desenvolvimento, será feito uma
análise mais profunda do problema, serão levantados todos os requisitos funcionais e não funcionais do software, a 
arquitetura, além disso, serão feitos testes ao longo do processo de desenvolvimento.

Além desse software, será necessário o desenvolvimento de outros, que estão mais relacionados à interação com o ser
humano, apresentando para o usuário os índices de qualidade da água. Esses softwares serão executados por hardwares
mais potentes e,  portanto, não haverá tantas limitações de processamento e memória. Dessa forma, não será necessário
fazer uma análise profunda da interação entre a eletrônica e o software, embora seja necessário uma análise dos 
sistemas operacionais, linguagens que possam ser interpretadas ou compiladas para ele.
  
\subsubsection{Interface homem-máquina}

    
  O presente trabalho não irá fornecer um sistema completo com todas as funcionalidades implementadas,
  apenas um protótipo da interface com o usuário. Para tal, é necessário o entendimento de alguns conceitos de
  Interação Humano-Computador.\\
  
  \noindent
  \textbf{Usabilidade}
  
  No processo de produção de software, uma parte importante é a produção de protótipos. Esses protótipos sevem
  como base para a validação de requisitos, além de serem usados para a avaliação de usabilidade do software.
  Sendo assim, os softwares que serão desenvolvidos devem seguir os seis princípios de usabilidade, segundo \cite{preece02}, que são:
 
 \begin{itemize}
  \item Eficácia: O software deverá fazer bem o que lhe foi destinado a fazer.
  \item Eficiência: O software deverá fazer suas funções de modo rápido e fácil.
  \item Segurança: O software deverá prevenir o usuários de cometer erros que prejudiquem a realização do que o usuário deseja.
  \item Utilidade: O software deverá ser útil.
  \item Capacidade de aprendizagem: O software deverá ser fácil de ser utilizado.
  \item Capacidade de memorização: Após a primeira vez de uso, o usuário deverá ser capaz de utilizar o software novamente com mais facilidade do que na primeira vez.
 \end{itemize}
 
	Para a construção dos protótipos, serão utilizados ferramentas de mockup, abaixo estão listadas algumas no mercado:
     
\begin{center}
\textbf{Balsamiq}
\end{center}	

	A ferramenta Balsamiq Mockup permite a criação de protótipos e modelos de baixa e alta fidelidade,
	para sistemas desktop, web ou mobile.
	
\textbf{Prós:}
	
\begin{itemize}
  \item Interface Intuitiva;
  \item Possui versão para desktop;
  \item Versão trial de 30 dias;
  \item Preço baixo para a licença;
 \end{itemize}
 
\textbf{Contras:}
	
\begin{itemize}
  \item Não oferece equipe de suporte;
 \end{itemize}
     
     
     
\begin{center}
\textbf{Mockingbird}
\end{center}	

	É uma ferramenta online que permite a criação e compartilhamento de protótipos.
	
\textbf{Prós:}
	
\begin{itemize}
  \item Interface Intuitiva;
  \item Possui equipe de suporte;
  \item Ferramenta gratuita;
 \end{itemize}
 
\textbf{Contras:}
	
\begin{itemize}
  \item Não poussi aplicação offline;
 \end{itemize}
      
     
     
\begin{center}
\textbf{MockFlow}
\end{center}	

	Conjunto de wireframing profissional para projetar interfaces.
	
\textbf{Prós:}
	
\begin{itemize}
  \item Interface Intuitiva;
  \item Versão online ou desktop;
  \item Ferramentas para trabalhar em equipe;
\end{itemize}
 
\textbf{Contras:}
	
\begin{itemize}
  \item Versão free limitada a 4 páginas;
  \item Não possui a opção de pagamento mensal;
  \item Baseado em Flash;
\end{itemize}
      
  \section{Requisitos do sistema}
  
      O projeto possui 4 frentes de requisitos. São elas:
      
      \begin{itemize}
	\item Requisitos do projeto estrutural mecânico do sistema de captação da água e do transporte para a central de armazenamento;\\
	 
	 \textbf{Requisitos funcionais}
	  \begin{itemize}
	   \item Retirar água da umidade do ar;
	   \item Bombear água para o reservatório;
	  \end{itemize}
	  
	  \textbf{Requisitos não-funcionais}
	  \begin{itemize}
	   \item O sistema deve atender a uma demanda de água diária;
	   \item O sistema deve bombear toda água por dia produzida pro reservatório; 
       \item O sistema deve ser composto por materiais resistentes à água;
       \item O sistema deve possuir um reservatório de capacidade de até 3 dias de água;
       \item O sistema deve operar com incidências de ventos de 7 a 50 m/s, umidade a partir de 40\% e 	temperatura à partir de $26\,^{\circ}\mathrm{C}$\cite{eole}
       \item O sistema deve aproveitar o relevo da região para o transporte da água;

	  \end{itemize}
	  
	\item Requisitos do projeto dos circuitos eletrônicos que irão compor o sistema de monitoramento e controle da qualidade da água.\\
	 
	 \textbf{Requisitos funcionais}
	  \begin{itemize}
	   \item Atuar como um sistema de controle dos elementos do sistema de modo a produzir a saída desejada (manter a água própria para o consumo humano);
	   \item Obter informações climáticas da região;
	   \item Obter dados do estado reservatório;
	   \item Ler parâmetros que definem a qualidade da água;
	   \item Efetuar conversão analógica/digital dos sinais filtrados;
	   \item Processar o sinal convertido de modo que os dados possam ser transmitidos ao usuário;
	   \item Exibir dados obtidos ao usuário;
	  \end{itemize}
	  
	  \textbf{Requisitos não-funcionais}
	  \begin{itemize}
	   \item Utilizar sensores para obtenção dos dados;
	   \item Exibir dados dos sensores ao usuário em tempo real;
	   \item Utilizar filtros analógicos para retirar eventuais ruídos que a saída do sensor possa gerar;
	  \end{itemize}
	  
	\item Requisitos do projeto do sistema de Gestão da Informação do monitoramento da qualidade da água;\\
	 
	 \textbf{Requisitos funcionais}
	  \begin{itemize}
	   \item Apenas o moderador poderá, modificar os dados.
	   \item O sistema registrará os dados de qualidade da agua.
	   \item O sistema deve emitir um alerta caso um parâmetro de qualidade não esteja aceitável.
	   \item O sistema deve permitir consultas dos dados armazenados de datas anteriores.
	   \item O sistema deve possuir uma interface pare exibir os dados.
	   \item O sistema deve possuir uma página de \textit{login} antes de entrar no sistema.
	   \item O sistema deve possuir um mecanismo de impressão dos dados.
	   \item O sistema possuirá um mecanismo para exportar os dados.
	  \end{itemize}
	  
	  \textbf{Requisitos não-funcionais}
	  \begin{itemize}
	   \item O sistema deve ser fácil de usar, evitando excesso de digitação, de modo a dar agilidade ao processo.
	   \item O sistema deve possuir uma interface simples.
	   \item O sistema deve funcionar no sistema operacional Windows.
	   \item O sitema deve monitorar as amostras de água a cada 30 mimutos.
	  \end{itemize}
	
	\item Requisitos do projeto da matriz energética que dará o suporte para o sistema de captação de água e o sistema de monitoramento da qualidade da água;\\
	
	 \textbf{Requisitos funcionais}
	  \begin{itemize}
	   \item Produzir energia elétrica através da energia eólica;
	   \item Converter energia cinética em energia elétrica;
	   \item Armazenar a energia elétrica gerada;
	   \item Fornecer energia para os componentes eletrônicos, de controle e monitoramento do produto final;
	   \item Fornecer energia para o bombeamento mecânico de água.
	  \end{itemize}
	  
	  \textbf{Requisitos não-funcionais}
	  \begin{itemize}
	   \item Utilizar uma fonte renovável de energia;
	   \item Ser autossuficiente no quesito energia gerada-consumida;
	   \item Possuir eficiência energética aceitável;
	   \item Ser estável energeticamente;
	  \end{itemize}
	  
      \end{itemize}
    
    
    
    