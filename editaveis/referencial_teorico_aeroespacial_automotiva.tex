% \documentclass[12pt,openright,oneside,a4paper,brazil]{abntex2}
% \usepackage[utf8]{inputenc}
% \counterwithout{section}{section}
% \counterwithout{figure}{chapter}
% \counterwithout{table}{chapter}
% \setlength{\parindent}{1.3cm}
% \usepackage{indentfirst}
% \setlength{\parskip}{0.2cm}
% \usepackage[bottom=2cm,top=3cm,left=3cm,right=2cm]{geometry}
% \usepackage{graphicx}
% \graphicspath{{figuras/}}
% \usepackage{placeins}
% \usepackage{cite}
% \usepackage{url}
% \usepackage{breakurl}
% \include{bibliografia}
% 
% \makeatletter
% \setlength{\@fptop}{0pt}
% \makeatother
% 
% \begin{document}
% 
% \textual
% \begin{center}
%  {\large Captação de água e materiais estruturais}\\[0.2cm]
%  \end{center}
 
Dentre as inspirações de tecnologia a serem aplicadas no projeto, as que se destacaram foram a Eolewater, Maxwater, Skywater
e Warkawater. As três primeiras se baseiam no cooling compression, porém com custos diferentes. A quarta tem custo  
extremamente baixo se comparado com as outras três tecnologias, por causa da simplicidade dos materiais que a constituem.
Contudo, produz valores significantemente menores de água, além de não gerar energia para os processos de controle da qualidade
da água. Por não gerar energia, também descartamos o Skywater.

As duas primeiras se baseiam em turbinas eólicas autossuficientes que retiram a umidade do ar, condensam,
purificam e distribuem. Devido a geometria dos rotores do Max water que não é tão eficiente como os rotores tradicionais,
de turbinas eólicas comuns, foi escolhido como base O sistema da Eolewater, que está descrito na Figura ~\ref{max_water_turbina}.

\begin{figure}[!ht]
\centering
\includegraphics[scale=0.6]{editaveis/figuras/max_water}
\caption[Max Water]{Max Water. Note como seus rotores são paralelos e verticais, tal configuração é ineficiente do ponto
	de vista aerodinâmico por gerar turbulência sobre os rotores vizinhos e gerar torques contrários com uma mesma 
	direção de fluxo de ar.\footnotemark}

\label{max_water_turbina}
\end{figure}
\footnotetext{Disponível em: http://peswiki.com/index.php/Image:Max-water.jpg}

\begin{figure}[!htbp]
\centering
\includegraphics[scale=0.6]{editaveis/figuras/Componentes}
\caption[Componentes de uma turbina Eolewater.]{Componentes de uma turbina Eolewater.\footnotemark}
\FloatBarrier
\label{Eole_Water}
\end{figure}

\footnotetext{Fonte: \cite{renewable} }

As componentes de uma turbina eólica pouco mudam para essa acima,pois a Eolewater além de gerar energia para seu próprio
funcionamento gera água, enquanto que a turbina eólica gera apenas energia.Na maioria das tecnologias de sucesso que 
pesquisamos, A Obtenção de água é feita pela condensação  a frio (cooling condensation), que é feita com o contato de 
ar com uma superfície fria. Para gerar essa superfície fria, um compressor comprime um fluido refrigerante, elevando sua
temperatura. Esse fluido em alta temperatura passa por um trocador de calor e depois é expandido, o que causa uma queda 
ainda maior na temperatura do fluido. O fluido sob baixa temperatura circula por um condensador, por onde passa o ar 
atmosférico coletado. Esse condensador faz com que a temperatura do ar caia até o ponto de orvalho, temperatura na qual a
água presente no ar se condensa em pequenas gotículas devido a saturação da quantidade de água no ar. A água proveniente
da condensação é coletada e passa por tratamentos em UV e carvão ativado para que seja descontaminada e esteja pronta para
consumo.



No momento, o levantamento de materiais será apenas estrutural e se baseará em uma turbina eólica.

\begin{figure}[!htbp]
\centering
\includegraphics[scale=0.80]{editaveis/figuras/turbina}
\caption[Seção de uma turbina eólica]{Seção de uma turbina eólica típica conectada à rede.\footnotemark}
\FloatBarrier
\label{secao_turbina_eolica}
\end{figure}
\footnotetext{Fonte: USP, 2005 }

Dentro da turbina eólica temos os seguintes subconjuntos: torre, rotor, nacele, caixa de multiplicação (transmissão),
gerador, mecanismos de controle, anemômetro, pás de rotor, biruta (sensor de direção). A torre é o elemento que sustenta
o rotor e a nacele na altura adequada para o funcionamento da turbina eólica. Esse item é de elevada contribuição no custo 
inicial do sistema. O rotor é a componente onde as pás são conectadas e que realiza a transformação de energia cinética dos
ventos em energia mecânica de rotação \cite{rossiEtAl}.

Nacele é um compartimento localizado no alto da torre que abrigam mecanismos do gerador (freios, caixa multiplicadora,
embreagens, sistemas hidráulico, etc). Usaremos a Nacele para também abrigar os mecanismos de obtenção de água. 
	
Caixa multiplicadora (transmissão) é o mecanismo que transmite a energia mecânica do eixo do rotar ao eixo do gerador.
Gerador é o converterá energia mecânica do eixo em energia elétrica. Mecanismos de controle são os que supervisionam a 
velocidade média nominal que ocorre com maior frequência durante um determinado período. Anemômetro tem a função de medir
a intensidade e a velocidade dos ventos. Pás do rotor captam o vento e converte sua potência ao centro do rotor.
Biruta é um conjunto de sensores que captam a direção do vento \cite{rossiEtAl}.
	
Uma das componentes que se tem muito estudo é a pá rotativa. Ela pode ser feita com os seguintes materiais: 
madeira, aço, alumínio, fibra de vidro com resina poliéster, fibra de vidro com fibra de carbono, madeira com epóxi,
fibra de carbono. A escolha do material vai depender da escolha do perfil aerodinâmico, que será estudado posteriormente.
\cite{portalEnergia}.

\begin{figure}[!htbp]
\centering
\includegraphics[scale=0.80]{editaveis/figuras/pa}
\caption[Seção transversal de uma pá feita de fibra de vidro]{Seção transversal de uma pá feita de fibra de vidro\footnotemark}
\FloatBarrier
\label{secao_transversal_pa}
\end{figure}
\footnotetext{Fonte: \cite{usp}}

Como pode ser visto as fibras são colocadas estruturalmente nas principais direções de propagação das tensões, quando 
em operação. A fibra de carbono e ou Kevlar são atualmente os compostos mais avançados que podem ser utilizados em áreas
críticas (longarina da pá), mas tal material possui preços muito elevados \cite{barrosVarela}.  

Em relação ao suporte estrutural, ou torre, nas turbinas eólicas elas podem ser do tipo treliçadas, tubular e estaiada,
no entanto, para a Eolewater as estruturas mais comuns são as duas últimas. As tores são constituídas de concreto e aço,
tendo o peso em torno de 40 toneladas e 50 metros de comprimento \cite{usp}.

\begin{figure}[!htb]
\centering
\includegraphics[scale=0.80]{editaveis/figuras/torre}
\caption[Tipos de torre]{Tipos de torre. Da esquerda para a direita: Treliçada, Tubular e Estaiada \footnotemark}
\FloatBarrier
\label{torre}
\end{figure}
\footnotetext{Fonte: \cite{usp}}

O modelo do dispositivo da Eolewater de gerar água por meio da energia eólica possui uma turbina WMS1000, de potência de30kW.
O tempo de vida proposto para esse mecanismo é de 20 anos, dependendo das condições em que o motor é submetido ele pode gerar
até 1200 litros de água por dia (mais informações na tabela abaixo). Como o dispositivo não necessita de quaisquer outros 
recursos para operar há um impacto mínimo sobre o meio em que é colocado \cite{renewable}.

\begin{figure}[!htbp]
\centering
\includegraphics[scale=0.3]{editaveis/figuras/condicoes}
\caption[Tabela de condições de umidade e temperatura para o rendimento de água]{Tabela de condições de umidade e temperatura para o rendimento de água \footnotemark}
\FloatBarrier
\label{condicoes}
\end{figure}
\footnotetext{Fonte: \cite{renewable}}

Essa tecnologia possui um controle de pitch centrífuga para regular a velocidade do motor, tem um sistema de travagem
rotor mecânica e elétrica, o qual evita danos nas lâminas giratórias (pás), ainda, contém um mastro de inclinação que 
integra a ação dos cilindros telescópicos com capacidade de empuxo de 115 toneladas. Deve-se destacar que os componentes 
que entram em contato com a água são feitos de uma liga de aço inoxidável especial que operará sem risco de corrosão 
\cite{eole}.

Uma tecnologia como essa segundo o site da Indústria Eólica uma turbina de vento abaixo de 100 kW vai custa por volta
de US \$ 3.000 a US \$ 5.000  por quilowatt de capacidade. Portanto, levando em conta as especificações técnicas do Turbine
WMS1000 abaixo a tecnologia é eficiente, mas cara \cite{renewable}.
	
\begin{figure}[!htbp]
\centering
\includegraphics[scale=0.8]{editaveis/figuras/especificacao}
\caption[Especificação Técnica do Turbine WMS1000]{Especificação Técnica do Turbine WMS1000 Vento \footnotemark}
\FloatBarrier
\label{Especificacoes}
\end{figure}
\footnotetext{Fonte: \cite{renewable}}
 
A outra tecnologia, Warawater, por sua vez é uma tecnologia muito barata se comparada com a mencionada anterior. 
Essa custa cerca de US\$ 500 e pode ser construída em menos de uma semana com uma equipe de quatro pessoas e materiais 
existente localmente \cite{warkawater}.

Os materiais necessários para a sua construção de sua estrutura são: recipiente de coleta, bambu e um revestimento
interno de plástico reciclado (rede). Sua torre possui em média 10 metros de altura, com 60 Kg e pode suprir até 100 litros
de água por dia \cite{warkawater2}.

\begin{figure}[!htbp]
\centering
\includegraphics[scale=0.3]{editaveis/figuras/warkawater}
\caption[Utilização da tecnologia warkawater por uma população carente]{Utilização da tecnologia warkawater por uma população carente  \footnotemark}
\FloatBarrier
\label{Especificacoes}
\end{figure}
\footnotetext{Fonte: \cite{warkawater}}

% \bibliographystyle{abnt-alf}
% \bibliography{bibliografia}

% \end{document}