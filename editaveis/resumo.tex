\begin{resumo}
 A escassez de água é um problema comum em diversas regiões do Brasil e do mundo. A seca acarreta uma série de dificuldades
 para a região, dentre as quais, se destaca a social. Com a falta  de água, a agricultura e a pecuária ficam comprometidas,
 diminuindo a produção de alimento. Somado a isso, a população busca  por qualquer fonte de água que encontrarem, e, devido
 a sede,  não se preocupam se a água é suja ou contaminada. Com uma alimentação precária e consumo de água de baixa qualidade,
 os habitantes dessas regiões acabam vítimas de muitas doenças. Para solucionar parte desse problema social, foi proposta uma
 nova tecnologia capaz de amenizar o impacto da falta de água potável. O presente trabalho propõem uma tecnologia de captação
 de água através da umidade do ar. Pretende-se fazer isso por meio do uso de um condensador de umidade embutido em uma turbina
 eólica. Dessa forma, será obtido um sistema autossustentável, que não necessita de energia vinda de uma fonte externa, e capaz
 de beneficiar os habitantes da região, podendo ser considerado, também, um projeto de engenharia social. Para a consolidação
 dessa ideia, foram discutidos diversos assuntos como, por exemplo, em qual local a tecnologia melhor se adequaria.
 Dentre as regiões analisadas, o município de Acarí, localizado no Rio Grande do Norte, se destacou. Portanto, o trabalho
 proposto será adequado às características da região selecionada, buscando a melhor eficiência possível.

 \vspace{\onelineskip}
    
 \noindent
 \textbf{Palavras-chaves}: Umidade do ar. Água. Acarí.
\end{resumo}
