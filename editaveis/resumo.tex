\begin{resumo}
 
 A escassez de água é um problema comum em diversas localidades do globo, incluindo algumas regiões do Brasil. 
 A seca acarreta uma série de dificuldades para o desenvolvimento dessas regiões, dentre os problemas encontrados,
 os principais estão relacionados ao âmbito social. Com a falta  de água, a agricultura e a pecuária ficam comprometidas,
 diminuindo a produção de alimentos. Somado a isso, a busca desesperada por água faz com que a população não se preocupe
 se a água é suja ou contaminada. Com uma alimentação precária e consumo de água de baixa qualidade, os habitantes dessas
 regiões acabam vítimas de muitas doenças. Para solucionar parte desse problema social, foi proposta uma nova tecnologia
 capaz de amenizar o impacto da falta de água potável. Apesar de o problema de falta de água não poder ser sanado por completo,
 será possível reduzir, consideravelmente, a sede dos moradores próximos à região, e evitar a contaminação deles pela ingestão
 de água de baixa qualidade. O presente trabalho propõem uma tecnologia de captação de água através da umidade do ar.
 Pretende-se fazer isso por meio do uso de um condensador de umidade embutido em uma turbina eólica. Dessa forma, será
 obtido um sistema autossustentável, que não necessita de energia vinda de uma fonte externa, e capaz de beneficiar os 
 habitantes da região, podendo ser considerado, também, um projeto de engenharia social. Para a consolidação dessa ideia,
 foram discutidos diversos assuntos como, por exemplo, em qual local a tecnologia melhor se adequaria. Dentre as regiões 
 analisadas, o município de Acari, localizado no Rio Grande do Norte, se destacou. Portanto, o trabalho proposto será adequado
 às características da região selecionada, buscando a melhor eficiência possível.

 \vspace{\onelineskip}
    
 \noindent
 \textbf{Palavras-chaves}: Umidade do ar. Água. Acarí.
\end{resumo}
