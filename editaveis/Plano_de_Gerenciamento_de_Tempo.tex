\documentclass[12pt,openright,oneside,a4paper,brazil]{abntex2}
\usepackage[utf8]{inputenc}
\counterwithout{section}{section}
\counterwithout{figure}{chapter}
\counterwithout{table}{chapter}
\setlength{\parindent}{1.3cm}
\usepackage{indentfirst}
\setlength{\parskip}{0.2cm}
\usepackage[bottom=2cm,top=3cm,left=3cm,right=2cm]{geometry}
\usepackage{graphicx}
\graphicspath{{figuras/}}
\usepackage{placeins}

%opening
\title{}
\author{}

\begin{document}


\textual
\begin{center}
 {\large Plano de gerenciamento de Tempo}\\[0.2cm]
 {Planta de abastecimento de água potável a partir da umidade do ar}\\
 \end{center}
 
 \section{Histórico de Alterações}
\begin{table}[h]
\centering
\begin{tabular}{|c|c|p{6cm}|p{5cm}|}

Data & Versão & Descrição & Responsável\\
\hline                               
22/04/2015 & 0.0 & Criação do plano de gerenciamento de tempo & Hugo Martins Ferreira\\
\hline
\end{tabular}
\end{table}

\section{Objetivo}
O objetivo desse plano é estabelecer um processo de gerenciamento do tempo do projeto.

Importância do plano de gerenciamento de tempo para o projeto:
O gerenciamento do tempo, quando feito de forma adequada, é fundamental para o desenvolvimento de qualquer projeto. Tendo isso em vista, esse plano de gerenciamento tem como objetivo garantir com que cada atividade relacionada ao desenvolvimento do projeto da Planta de Abastecimento de Água através da Umidade do Ar seja executada dentro de um intervalo de tempo determinado para cada uma delas para que assim seja assegurado o desenvolvimento do projeto sem  que haja atrasos.
	
\section{Descrição dos processos de gerenciamento do tempo}
Para assegurar o desenvolvimento do projeto dentro do prazo predeterminado foram definidos processos de gerenciamento que serão seguidos: gerenciar, desenvolver e controlar o cronograma, definição das atividades, sequenciar as atividades, estimar as durações das atividades e estimar os recursos das atividades.
\begin{itemize}
\item Gerenciar, desenvolver e controlar o cronograma:\\
Esses processos relacionados ao cronograma envolvem a sua criação definindo as atividades que estão envolvidas com o desenvolvimento do projeto. Cada uma dessas atividades possui uma ordem de execução, pois algumas atividades estão ligadas à outras e por isso precisam seguir o seu tempo de execução para não atrasar nenhum passo no desenvolvimento do projeto. Para assegurar que esses processos estão sendo seguidos da maneira que deve ser, existe um membro (ou um grupo de membros) da equipe do projeto que é responsável pelo controle do cronograma.
\item Definição das atividades:\\
Esse processo envolve a identificação das atividades para o desenvolvimento do projeto e, além de identificar as atividades, esse processo delega responsáveis por cada uma delas. Cada uma dessas atividades e os seus devidos responsáveis devem ser registrados no cronograma.
\item Sequenciar as atividades:\\
Esse processo determina as datas e a ordem das entregas de cada uma das atividades. Como já foi dito, essa ordem é importante, pois algumas atividades dependem de outras.
\item Estimar as durações das atividades:\\
Esse processo determina o tempo de execução de cada atividade, pois é necessário para seguir a ordem da sequência das entregas das atividades para que não haja atraso no cronograma.
\item Estimar os recursos das atividades:\\
Esse processo determina quais são os recursos necessários para o desenvolvimento de cada atividade. Esses recursos podem ser recursos humanos, materiais, equipamentos, suprimentos ou informações de outras atividades.
\end{itemize}

\section{Priorização das mudanças nos prazos}
Para que não haja muita mudança no planejamento de execução do projeto, dificilmente haverá mudança nos prazos de entrega das atividades. Só poderá haver mudanças caso ocorra um aumento no prazo de entrega do projeto. De resto, o máximo que pode acontecer é uma antecipação no início de uma atividade, fazendo com isso que a equipe responsável pela mesma tenha um prazo maior na execução da mesma. Para isso acontecer, a atividade anterior, caso ela seja insumo para esta atividade, precisa ter sido concluída antes do seu período máximo de execução.

\section{Sistema de Controle de Mudanças de Prazos (SCMP)}
As mudanças de prazos no projeto deverão ser previamente autorizadas pela equipe de gerência. As mudanças deverão ser comunicadas previamente e deverão seguir o prazo limite de tempo que será estabelecimento anteriormente a cada atividade. Haverá um registro de mudanças que será atualizado sempre que houver alguma mudança no prazo de entrega de atividades. O registro de mudanças no projeto deverá ser entregue a todos os interessados no projeto, pois pode influenciar diretamente no tempo, nos custos e nos riscos do mesmo. Desta forma, as aprovações das mudanças de prazo deverão ser revisadas pelas equipes de gerenciamento de custo, tempo e riscos. As mudanças poderão ser realizadas sempre que houver uma reserva de tempo. Esta reserva deve ser utilizada para o término de atividades inacabadas ou para atividades acrescentadas que seguirem às especificações das equipes de gerenciamento.

\section{Mecanismo adotado para o conciliamento de recursos}
Sempre que uma atividade for definida, dever-se-á realizar uma alocação de recursos designados para suprir a demanda daquela atividade. Caso haja reserva de tempo ou custos, estes recursos serão alocados para alguma atividade. O mecanismo de escolha desta atividade será seguir a ordem de importância da mesma para o projeto, ou seja, a atividade mais importante para o desenvolvimento do projeto alocará os recursos extras. 

\section{Reserva de tempo do projeto}
O projeto contará com uma reserva de tempo sempre que for extremamente necessário. Porém, esta reserva de tempo somente existirá quando aprovada pelo conselheiro do grupo de projeto. A reserva de tempo também pode existir devido á alguma mudança na alocação de tempo para o desenvolvimento das atividades.

\section{Frequência de avaliação dos prazos do projeto}
A avaliação dos prazos do projeto será realizada semanalmente.

\section{Alocação financeira para o gerenciamento do tempo}
A alocação de recursos financeiros pode ser modificada por alguma mudança realizada pela equipe de gerenciamento de tempo. Desta forma, a equipe de gerenciamento de custos deve estar sempre informada das mudanças realizadas pela equipe de gerenciamento do tempo.

\section{Administração do plano de gerenciamento do tempo}
\begin{enumerate}

\item Responsável pelo plano:
\begin{itemize}
\item Hugo Martins Ferreira
\item Ana Paula Chavier Rodrigues
\end{itemize}
\item Frequência de atualização do plano de gerenciamento do tempo

A atualização do plano de gerenciamento do projeto será realizada mensalmente.
\end{enumerate}
\section{Assinaturas}
\begin{center}
Data: \rule{0.5cm}{0.1mm}/\rule{0.5cm}{0.1mm}/\rule{1cm}{0.1mm}     \\
\rule{13cm}{0.1mm}\\
ADRIANNY VIANA DE ARAÚJO AMORIM – GERENTE DE PROJETO\\
\rule{13cm}{0.1mm}\\
HUGO MARTINS FERREIRA- GERENTE DE TEMPO

\end{center}
\end{document}