 \section*{\centerline{Planta de abastecimento da água através da umidade do ar.}}
 \centerline{ \textbf{Sistema de monitoramento do processo de captação e distribuição da água.}}
 \centerline{\textbf{Documento de visão técnica.}}

 
 \textbf{HISTÓRICO}
 \begin{longtable}{|m{2.8cm}|m{1.5cm}|m{4.2cm}|m{3.5cm}|}
  \hline
\textbf{Data} & \textbf{Versão} & \textbf{Descrição} & \textbf{Autor}\\
  
  \hline
  18/05/2015 & 0.1 & Criação do documento & Hugo Martins
  \hline
  18/05/2015 & 0.2 & Definição da descrição do problema e de alguns requisitos do sistema & Alexandre Torres,  Hugo Martins, Italo Paiva, Jonnatas Lennon e Karine Valença
  \hline
  18/05/2015 & 0.3 & Edição no documento geral & Hugo Martins e Italo Paiva
  \hline
  21/05/2015 & 0.4 & Criação dos requisitos do sistema & Alexandre Torres, Hugo Martins, Italo Paiva, Jonnatas Lennon e Karine Valença
  \hline
  21/05/2015 & 0.5 & Edição no documento geral & Alexandre Torres
  \hline
  23/05/2015 & 0.6 & Edição no documento geral & Hugo Martins
  \hline
  25/05/2015 & 0.7 & Criação dos requisitos não funcionais & Alexandre Torres,  Jonnatas Lennon, Karine Valença
  \hline
  26/05/2015 & 0.8 & Modificação dos requisitos funcionais & Alexandre Torres
  \hline
  27/05/2015 & 0.9 & Alterações gerais  e finalização do documento & Alexandre Torres
  \hline
  16/06/2015 & 1.0 & Reestruturação do documento & Hugo Martins
    \hline
  20/06/2015 & 1.0 & Incrementando novos requisitos funcionais derivados dos três subsistemas definidos & Hugo Martins, Alexandre Torres e Jonnatas Lennon
  \hline
 \end{longtable}
 
  \section{INTRODUÇÃO}
  Como tentativa de amenizar os problemas gerados pela falta de água potável na cidade de Acarí - RN, foi proposto um sistema 
  capaz de gerar água de boa qualidade a partir da umidade do ar. Para o correto funcionamento desse sistema é necessário um 
  software capaz de permitir o monitoramento da água.
  
  O objetivo primário do software proposto será o monitoramento da qualidade e disponibilidade da água. Além disso, ele 
  disponibilizará outras informações relacionadas a tecnologia proposta referentes, por exemplo, à turbina, à casa de 
  controle, ao monitoramento da adição de sais.
  
  \section{DESCRIÇÃO DO PROBLEMA}
  Encontra-se abaixo a tabela com a descrição do problema identificado:
  
   \begin{longtable}{|m{5.0cm}|m{11.2cm}|}
  \hline
\textbf{O problema da:} & Falta de monitoramento do processo de captação e distribuição da água. 
  \hline
\textbf{Que afeta:} & Além dos cidadãos da cidade de Acarí - RN, também afeta os envolvidos no desenvolvimento do projeto da planta de abastecimento da água através da umidade do ar.
  \hline
\textbf{Cujo o Impacto é:} & A não garantia do correto funcionamento do sistema, podendo causar problemas à saúde humana e a inviabilidade do projeto.
  \hline
\textbf{Uma solução seria:} & A implementação de um software para o monitoramento do processo de captação e distribuição da água.
  \hline
 \end{longtable}
   
  \section{DESCRIÇÕES DOS ENVOLVIDOS E USUÁRIOS}
  Os stakeholders envolvidos no projeto de software são a equipe de desenvolvimento do projeto, mais diretamente a equipe 
  formada pelos integrantes da engenharia de software, os professores orientadores do projeto, os técnicos que serão usuários 
  do sistema e a população que vai se beneficiar do mesmo.
  
  \subsection{Resumo dos Envolvidos}
  A equipe composta por 27 alunos do grupo 1 da disciplina Projeto Integrador I da Universidade de Brasília, Campus Gama 
  no 1º semestre de 2015, técnicos que ficarão responsáveis pelo acompanhamento do sistema, os professores orientadores e 
  da população da cidade de Acarí – RN, mais especificamente no bairro Vereador Tarcísio Bezerra Galvão.
  
  \subsection{Resumo dos Usuários}
  Os técnicos responsáveis pelo acompanhamento do sistema, irão monitorar os dados referentes ao  processo de captação e 
  distribuição da água gerados pelo sistema.
  
  \subsection{Ambiente dos Usuários}
  O sistema será implantado na casa  de Controle da planta de abastecimento da água. Esta casa de controle objetiva manter que 
  todo o sistema esteja funcionando da maneira esperada.
  
  \subsection{Ambiente das Principais Necessidades dos Envolvidos ou Usuários}
  A equipe deve assegurar que a água extraída pela planta esteja em um nível considerável para consumo. Caso não esteja, o 
  problema causador deve ser mitigado evitando que ocorra novamente. O técnico também abastecerá o reservatório de sais 
  utilizados na mistura com a água caso esteja com uma pequena quantidade.
  
  \section{VISÃO GERAL DO PRODUTO}
  
  \subsection{Perspectiva do Produto}
  Esse software é um componente do sistema proposto de geração de água potável a partir da umidade do ar. A sua finalidade 
  principal é o monitoramento do sistema como um todo. Esse sistema está dividido em três subsistemas: 
  
  1. Subsistema de monitoramento mecânico e lógico das turbinas eólicas.
  2. Subsistema de monitoramento da qualidade da água.
  3. Subsistema de monitoramento do reservatório de distribuição.
  
  Alguns dos dados referentes ao monitoramento mecânico e lógico das turbinas que serão monitorados pelo software são: as temperaturas 
  internas, a quantidade de energia gerada por cada turbina e a velocidade de rotação das pás. 
  
  O subsistema de monitoramento da qualidade da água analisará, por exemplo, a quantidade de sais na água, o pH, a condutividade e a 
  turbidez, além de informar dados referentes ao volume de água disponível no reservatório.
  
  O subsistema de monitoramento do reservatório de distribuição deve monitorar o volume da água gerada pelas turbinas, armazenadas
  no reservatório e distribuídas para a população.
  
  Percebe-se, portanto, que o software proposto visa, basicamente, possibilitar o monitoramento de uma série de aspectos relacionados a 
  tecnologia de captação de água a partir da umidade do ar.
  
  
  \subsection{Suposições e Dependências}
  É possível que haja alterações futuras no software proposto, principalmente, relacionadas a integração com eletrônica. Alterações 
  nos componentes eletrônicos afetam diretamente o software proposto.
  
  
  \section{RECURSOS DO PRODUTO}
  Este item traz todos os recursos referentes ao produto: Os requisitos funcionais, os não funcionais e as restrições referentes 
  ao sistema. Cada um desses requisitos possui um código de identificação para facilitar a rastreabilidade, a sua descrição e 
  o seu nível de prioridade que são prioridade alta, média ou baixa e abaixo está estabelecido o significado de cada uma 
  destas prioridades.
  
  \textbf{Alta:}  O sistema só funcionará de maneira eficaz caso este requisito esteja sendo atendido.
  \textbf{Média:} O sistema pode funcionar caso este requisito não seja atendido.
  \textbf{Baixa:} O sistema funcionará 	normalmente caso este requisito não seja atendido.
  
  
  \subsection{Requisitos Funcionais}
  
  Os requisitos do sistema de monitoramento do processo de captação e distribuição da água foi dividido em 
  três subsistemas que estão listados abaixo junto a cada um de seus requisitos:
  \pagebreak
  
  \textbf{1. Monitoramento mecânico e lógico das turbinas eólicas}
  
  \begin{longtable}{|m{3.0cm}|m{7.5cm}|m{3.5cm}|}
   \hline
\textbf{Código} & \textbf{Descrição} & \textbf{Prioridade}
\hline
RF01 & O sistema deve permitir o monitoramento dos dados recebidos pelo acelerômetro (dados referentes à vibração da turbina). & Alta
\hline
RF02 & O sistema deve permitir o monitoramento dos dados recebidos pelo sensor Hall (dados referentes a velocidade de rotação das pás da turbina). & Alta
\hline
RF03 & O sistema deve monitorar os dados recebidos pelo sensor de temperatura presente no compartimento do condensador. & Alta
\hline
RF04 & O sistema deve permitir o monitoramento dos dados recebidos pelo sensor de temperatura presente no compartimento do condensador. & Alta
\hline
RF05 & O sistema deve permitir o monitoramento dos dados recebidos pelo sensor de temperatura presente no compartimento próximo ao rotor. & Alta
\hline
RF06 & O sistema deve permitir o monitoramento dos dados recebidos pelo anemômetro (dados referentes a velocidade e direção do vento). & Alta
\hline
RF07 & O sistema permitir o monitoramento da carga das baterias presentes na turbina. & Alta
\hline
RF08 & O sistema deve permitir a visualização de todos os dados recebidos acerca do funcionamento da turbina. & Alta
\hline
RF09 & O sistema deve gerar mensagens de alerta para níveis extremos de algum dado recebido. & Alta
\hline
RF10 & O sistema deve permitir o monitoramento da distribuição da água das turbinas para o reservatório de tratamento. & Alta
\hline
RF11 & O sistema deve permitir o controle da válvula do reservatório de água presente em cada turbina. & Alta
\hline
RF12 & O sistema deve permitir o monitoramento do nível de água do reservatório da turbina. & Alta
\hline
   
  \end{longtable}

  \textbf{2. Monitoramento da qualidade da água}
  
    \begin{longtable}{|m{3.0cm}|m{7.5cm}|m{3.5cm}|}
      \hline
\textbf{Código} & \textbf{Descrição} & \textbf{Prioridade}
\hline
RF13 & O sistema deve permitir o monitoramento dos dados referente ao pH da água. & Alta
\hline
RF24 & O sistema deve permitir o monitoramento dos dados referente a condutividade elétrica da água. & Alta
\hline
RF25 & O sistema deve permitir o monitoramento dos dados referente a turbidez da água. & Alta
\hline
RF26 & O sistema deve gerar mensagens de alerta para níveis de sais fora dos parâmetros. & Alta
\hline
RF27 & O sistema deve permitir a visualização de todos os dados recebidos acerca dos dados referentes aos sais. & Alta 
\hline
RF18 & O sistema deve permitir o anexo dos relatórios referentes a análise química da água (obtidos pela empresa terceirizada). & Alta
\hline
RF19 & O sistema deve permitir a consulta dos relatórios referentes a análise química da água (obtidos pela empresa terceirizada). & Alta
\hline
RF20 & O sistema deve permitir o monitoramento do nível de água do reservatório de tratamento. & Alta
\hline
RF21 & O sistema deve permitir o controle da válvula que liga o reservatório de tratamento ao reservatório de distribuição. & Alta
\hline
RF22 & O sistema deve permitir o controle da válvula que liga o reservatório à rede. & Alta
\hline
  \end{longtable}
  \pagebreak
  \textbf{3. Monitoramento do reservatório de distribuição}
  
    \begin{longtable}{|m{3.0cm}|m{7.5cm}|m{3.5cm}|}
      \hline
\textbf{Código} & \textbf{Descrição} & \textbf{Prioridade}
\hline
RF23 & O sistema deve permitir o monitoramento do nível da água no reservatório. & Alta
\hline
RF24 & O sistema deve permitir o monitoramento da quantidade de água distribuída para a população. & Alta
\hline
RF25 & O sistema deve permitir o controle da válvula de distribuição da água para a rede da cidade. & Alta
\hline
RF26 & O sistema deve permitir o controle da válvula que encaminha a água para engarrafamento. & Alta
\hline
RF27 & O sistema deve permitir o cadastro de cidadãos beneficiados pela água gerada. & Alta
\hline
RF28 & O sistema deve permitir a consulta de cidadãos beneficiados pela água gerada. & Alta
\hline
RF29 & O sistema deve gerar relatórios referente ao número de cidadãos beneficiados pela água gerada. & Alta
\hline
  \end{longtable}
  
  Além disso, o sistema possui alguns requisitos gerais:
  
    \begin{longtable}{|m{3.0cm}|m{7.5cm}|m{3.5cm}|}
      \hline
\textbf{Código} & \textbf{Descrição} & \textbf{Prioridade}
\hline
RF30 & O sistema deve possuir um controle de acesso aos dados. & Média
\hline
RF31 & O sistema deve permitir gerar relatórios com dados referentes ao monitoramento da qualidade da água. & Média
\hline
RF32 & O sistema deve permitir gerar relatórios com dados referentes ao Monitoramento mecânico e lógico das turbinas eólicas. & Média
\hline
RF33 & O sistema deve permitir gerar relatórios com dados referentes ao Monitoramento do reservatório de distribuição. & Média
\hline
RF34 & O sistema deve permitir a impressão de relatórios com dados referentes ao monitoramento da qualidade da água. & Média
\hline
RF35 & O sistema deve permitir a impressão de relatórios com dados referentes ao Monitoramento mecânico e lógico das turbinas eólicas. & Média
\hline
RF36 & O  sistema deve permitir a impressão de relatórios com dados referentes ao Monitoramento do reservatório de distribuição. & Média
\hline

  \end{longtable}
  
  \subsection{Requisitos Não Funcionais}
  Este item traz a especificação dos requisitos que dão suporte para a correta execução dos requisitos funcionais e indicam quais são as limitações 
  do sistema e do seu desenvolvimento.
  
  \subsection{Usabilidade:}
  O sistema deve obedecer às metas de usabilidade. Além disso, ele deve possuir um sistema de ajuda para responder eventuais duvidas do usuário.
  
  \subsection{Confiabilidade:}
  O sistema deverá estar disponível 24h por dia. Além disso, o sistema não deve ser tolerante em falhas relacionados aos critérios de qualidade da água.

  \subsection{Portabilidade:}
  O sistema deve manter o seu correto funcionamento nos sistemas operacionais Windows XP SP3 ou superior, nos sistemas operacionais Linux e Mac OS.
	
  \subsection{Linguagem:}
   sistema será desenvolvido em C#/C++.

  \subsection{Restrições do Sistema}
  
     \begin{longtable}{|m{4.0cm}|m{12.2cm}|}
  \hline
\textbf{Código} & \textbf{Descrição} 
  \hline
RS1 & O sistema deverá estar associado a um banco de dados local, onde ficarão armazenados os dados obtidos.
  \hline
RS2 & O software deverá funcionar continuamente, sem interrupções.
  \hline
RS3 & O software deve ser desenvolvido deverá ser desenvolvido na linguagem C# ou C++.
  \hline
RS4 & O sistema deve ser capaz de interagir com os microprocessadores definidos.
  \hline
 \end{longtable}
  
  \section{RÓTULO E EMBALAGEM}
  
  O sistema deve possuir a logo do projeto representado na figura abaixo.
  
    \begin{figure}[!h]
    \centering
    \includegraphics[scale = 0.2]{editaveis/figuras/logo}
    \label{logo}
    \caption{Logomarca do projeto}
   \end{figure}
   \FloatBarrier
 
