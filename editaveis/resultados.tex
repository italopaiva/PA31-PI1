\chapter[Resultados]{Resultados e discussão}
\addcontentsline{toc}{chapter}{Resultados}
 
  \section{Sistema de captação e transporte da água}
  
    \subsection{Projeto mecânico estrutural das unidades de captação}
   
    
    \subsection{Projeto estrutural do transporte da água}
    
  \section{Matriz energética}
  
    
  \section{Sistema de monitoramento e controle da qualidade da água}
    
    \subsection{Escolha dos sensores}
      
      Optou-se por sensores os quais não são integrados em um sistema próprio isolado que já disponibiliza a leitura das grandezas
      referentes às propriedades analisadas: turbidez, oxigênio dissolvido, etc. Tal escolha foi feita visando agrupar todas as
      leituras em uma só interface controlada por um microcontrolador capaz de processar os sinais de saída de todos os sensores.
      Além disso, percebeu-se que fazendo esse tipo de abordagem (preferência por sensores não integrados à um sistema isolado)
      é possível baratear o projeto, uma vez que os sensores integrados são bem mais caros.
      
      \subsubsection{Sensor de rotação (Efeito Hall)}
	
	O sensor escolhido foi o US1881 \textit{Hall Latch} – \textit{High Sensitivity} da Melexis
	(\textit{Microeletonic Integrated Systems}), que é um sensor
	digital disponível na loja virtual SparkFun a U\$ 0,95 (este valor não considera as taxas de envio). Devido à sua larga
	faixa de operação de tensão (VDD) e temperatura, esse sensor é adequado para a aplicação proposta \cite{melexis}.
	
	O sensor detecta a presença de polos magnéticos. Existem dois tipos de encapsulamentos, cujas lógicas de funcionamento
	são inversas e são apresentadas a seguir:
	
	\begin{figure}[!htbp]
	  \centering
	  \includegraphics[scale=0.4]{editaveis/figuras/encapsulamento_sensor_efeito_hall}
	  \caption[Tipos de encapsulamento do sensor de efeito Hall]{Tipos de encapsulamento do sensor de efeito Hall \cite{melexis}.}
	  \label{encapsulamento_sensor_efeito_hall}
	\end{figure}
	
	\begin{figure}[!htbp]
	  \centering
	  \includegraphics[scale=0.5]{editaveis/figuras/funcionamento_sensores_encapsulamento}
	  \caption[Lógica de funcionamento dos sensores considerado os encapsulamentos]
	  {Lógica de funcionamento dos sensores considerado os encapsulamentos (T = $-40^\circ\mathrm{C}$ a $150^\circ\mathrm{C}$, VDD = 3,5V a 24V) \cite{melexis}.}
	  \label{funcionamento_sensores_encapsulamento}
	\end{figure}
	
	Onde:
	
	\begin{itemize}
	 \item $B_{op}$ = Densidade de fluxo magnético aplicado no lado que tem a marca do encapsulamento que liga o condutor de saída.
	 \item $B_{rp}$ = Densidade de fluxo magnético aplicado no lado que tem a marca do encapsulamento que desliga o condutor de saída.
	\end{itemize}
	
	Abaixo é apresentada uma tabela com a os valores máximos suportados de diversos parâmetros 
	(informações necessárias para a estimação de consumo energético do sistema):
	
	\begin{figure}[!htbp]
	  \centering
	  \includegraphics[scale=0.5]{editaveis/figuras/sensor_rotacao_max_valores}
	  \caption[Máximos valores suportados pelo sensor de rotação.]
	  {Máximos valores suportados pelo sensor de rotação \cite{melexis}.}
	  \label{sensor_rotacao_max_valores}
	\end{figure}
	
	Deve-se adquirir o sensor o qual tem o sufixo $\mathrm{L}$, uma vez que esse é o tipo que suporta a maior variação de
	temperatura: $-40^\circ\mathrm{C}$ a $150^\circ\mathrm{C}$ \cite{melexis}.
	
	As tabelas com as descrições/especificações do sensor são apresentadas abaixo:
	
	\begin{figure}[!htbp]
	  \centering
	  \includegraphics[scale=0.5]{editaveis/figuras/sensor_rotacao_pinagem}
	  \caption[Pinagem e função dos pinos do sensor de rotação]
	  {Pinagem e função dos pinos do sensor de rotação \cite{melexis}.}
	  \label{sensor_rotacao_pinagem}
	\end{figure}
	
	\begin{figure}[!htbp]
	  \centering
	  \includegraphics[scale=0.4]{editaveis/figuras/sensor_rotacao_spec_eletrica}
	  \caption[Pinagem e função dos pinos do sensor de rotação]
	  {Pinagem e função dos pinos do sensor de rotação \cite{melexis}.}
	  \label{sensor_rotacao_spec_eletrica}
	\end{figure}
	
	\begin{figure}[!htbp]
	  \centering
	  \includegraphics[scale=0.5]{editaveis/figuras/sensor_rotacao_spec_mag}
	  \caption[Especificações elétricas do sensor de rotação]
	  {Especificações elétricas do sensor de rotação \cite{melexis}.}
	  \label{sensor_rotacao_spec_mag}
	\end{figure}
	
	As figuras ~\ref{sensor_rotacao_idd_vs_vdd} e ~\ref{sensor_rotacao_vdd_vs_temperatura} apresentam dois gráficos
	de performance do sensor.
	
	\begin{figure}[!htbp]
	  \centering
	  \includegraphics[scale=0.5]{editaveis/figuras/sensor_rotacao_idd_vs_vdd}
	  \caption[Sensor de rotação -  IDD vs VDD]
	  {Sensor de rotação -  IDD vs VDD \cite{melexis}.}
	  \label{sensor_rotacao_idd_vs_vdd}
	\end{figure}
	
	\begin{figure}[!htbp]
	  \centering
	  \includegraphics[scale=0.5]{editaveis/figuras/sensor_rotacao_vdd_vs_temperatura}
	  \caption[Sensor de rotação -  VDD vs Temperatura]
	  {Sensor de rotação -  VDD vs Temperatura \cite{melexis}.}
	  \label{sensor_rotacao_vdd_vs_temperatura}
	\end{figure}
	
	Para medir a velocidade de rotação da turbina, pretende-se elaborar uma montagem como a mostrada a seguir:
	 
	\begin{figure}[!htbp]
	  \centering
	  \includegraphics[scale=0.4]{editaveis/figuras/sensor_rotacao_monitoramento_velocidade}
	  \caption[Esquema simplificado de funcionamento do monitoramento da velocidade do eixo]
	  {Esquema simplificado de funcionamento do monitoramento da velocidade do eixo.}
	  \label{sensor_rotacao_monitoramento_velocidade}
	\end{figure}
	
	Uma vez que o sinal é digital, não há necessidade de filtragem do sinal, basta aplicar uma tensão VDD que seja suportada
	pelo microcontrolador (5V por exemplo).
	
	\begin{figure}[!htbp]
	  \centering
	  \includegraphics[scale=0.4]{editaveis/figuras/sensor_rotacao_conexao}
	  \caption[Esquema de conexão do sensor de rotação]
	  {Esquema de conexão do sensor de rotação. \footnotemark}
	  \label{sensor_rotacao_conexao}
	\end{figure}
	\footnotetext{Disponível em: <http://bildr.org/2011/04/various-hall-effect-sensors/>.}
	
	O algoritmo para determinar a velocidade de rotação das hélices deve, portanto, avaliar a frequência em que a saída do
	sensor apresenta-se em nível lógico alto (borda de subida) ou baixo (borda de descida) e com base nisso, efetuar o
	cálculo da velocidade (rpm, por exemplo).
	
    \vfill
    
    \pagebreak
    \subsection{Sistema de Gestão da Informação}
    
      
      \subsubsection{Interface Homem-Máquina}

    
    
    
    
    