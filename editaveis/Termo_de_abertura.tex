% \documentclass[12pt,openright,oneside,a4paper,brazil]{abntex2}
% \usepackage[utf8]{inputenc}
% \counterwithout{section}{section}
% \counterwithout{figure}{chapter}
% \counterwithout{table}{chapter}
% \setlength{\parindent}{1.3cm}
% \usepackage{indentfirst}
% \setlength{\parskip}{0.2cm}
% \usepackage[bottom=2cm,top=3cm,left=3cm,right=2cm]{geometry}
% \usepackage{graphicx}
% \graphicspath{{figuras/}}
% \usepackage{placeins}
% 
% %opening
% \title{}
% \author{}
% 
% \begin{document}

%capa

% \textual
\begin{center}
 {\large Termo de abertura}\\[0.2cm]
 {Planta de abastecimento de água potável a partir da umidade do ar}\\
 \end{center}
 
 \section{Histórico de Alterações}
\begin{table}[h]
\centering
\begin{tabular}{|c|c|p{6cm}|p{5cm}|}

Data & Versão & Descrição & Responsável\\
\hline                               
20/04/2015 & 0.0 & Criação do documento pelo grupo de integração. & Ludimila S. Ferreira,
Laís R. Carvalho, Rafael C. Bragança, Vitor Silva.\\
\hline
21/04/2015 & 1.0 & Alteração dos itens: III, IV,V,VI,VII,VIII(3), IX & Ludimila S. Ferreira,
Laís R. Carvalho, Rafael C. Bragança, Vitor Silva.\\
\hline
\end{tabular}
\end{table}

\section{Nome do projeto}
  Planta de abastecimento de água potável a partir da umidade do ar
  
  \section{Descrição do projeto}
O projeto visa desenvolver uma planta de abastecimento de água potável, por meio de um sistema de captação a partir da umidade do ar na cidade de Aracari-RN (município da Microrregião do Seridó Oriental, na região do Seridó), no bairro Vereador Tarcísio Bezerra Galvão. O bairro possui uma população de aproximadamente 900 habitantes e sofre constantemente com a falta de água. 
O projeto gira em torno de uma solução tecnológica  capaz de retirar a água do ar e por processos químicos e físicos e tornar essa água potável e própria para o consumo humano de forma renovável e menos danosa possível. Para isso aplicamos conhecimentos das demais áreas da engenharia atuando no desenvolvimento e conceituação do projeto. Nossa estimativa é de que possamos produzir aproximadamente 3 mil litros de água por dia. Essa quantidade e suficiente para atender toda a população do município. 
O projeto conta com grupos que trabalham em áreas diferentes mas interligadas e com um único propósito. Cada etapa e gerida e coordenada por representantes de forma a garantir um resultado rápido e com qualidade.


\section{Objetivos do projeto}
  
  Este trabalho tem por objetivo propor um projeto de solução que consiga suprir a demanda da população. Os esforços que serão
realizados ao longo do período de desenvolvimento, no transcorrer de todas as etapas do processo buscam a qualidade e real
eficácia do produto, ou seja, oferecer a demanda de água do bairro Vereador Tarcísio Bezerra Galvão situado em Aracarí-RN, 
água potável de qualidade.

 São objetivos específicos do projeto:
 \begin{itemize}
  \item Elaborar o projeto mecânico estrutural do sistema de captação de água;
  \item Elaborar o projeto estrutural do sistema de transporte da água para a central de armazenamento;
  \item Elaborar o projeto do sistema de monitoramento e controle da qualidade da água captada;
  \item Elaborar o projeto da matriz energética que dará o suporte para os sistemas de captação de água e monitoramento e controle;
 \end{itemize}
  
\section{Justificativa do projeto}
 \begin{enumerate}[label=\Alph*]
\item Por que deve-se pensar em projetos como esse?\\
Atualmente, cerca de 40 \% da população mundial sofre com consequências da falta de água, além da sede faltam recursos hídricos, o que gera graves implicações na economia e política. De acordo com o geólogo Sjiklomanov, do Instituto Hidrológico Estadual de São Petesburgo, Rússia, em 2000 foi previsto que: “Os países em desenvolvimento vão aumentar seu uso de água em até 200\% em 25 anos”. Em 2014 no Brasil, foi evidenciado consequências desse aumento no consumo de água juntamente com fatores climático, resultando na falta de água em cidades de Pernambuco, Minas Gerais e São Paulo.
Essa situação de falta de água não é nova, a ONU em 2003 já previa os futuros transtornos que seriam causados pela crise de água. O World Water Development Report, se destaca sobre o tema porque é um documento da ONU que também traz estudos mostrando como esse problema já afeta e mata milhares de pessoas. Este estudo prevê que 2,7 bilhões de seres humanos – 45\% da população mundial – vão ficar sem água no ano 2025.
Diante dessa situação e de previsões sobre a falta de água tão breves, devem-se tomar medidas para minimizar a situação e planejar soluções para a produção de água potável buscando outras fontes, como o ar, por exemplo. Por isso, esse projeto visa através da umidade do ar, retirar água potável e planeja um estudo de abastecimento na cidade de Acari, RN.

\item Por que  Aracarí foi a cidade escolhida?\\
Na situação de crise hídrica vivida pelo país atualmente, observa-se grandes centros urbanos sofrendo com a falta de água para consumo humano (o que antes era praticamente exclusivo para a região do semiárido nordestino). Além disso, observa-se que a seca na região nordeste vem se agravando muito nos últimos anos, especificamente na região do Seridó, que fica no semiárido do RN.
Dessa forma, soluções alternativas para o abastecimento de água potável a fim de atender o consumo humano fazem-se necessárias. Sendo assim, a região para a qual o sistema será projetado será o município de Acari – RN, pois é uma região onde há muita demanda (11303 habitantes) e tem sofrido muito com a escassez de água. A escolha dessa região baseia-se principalmente na questão social, uma vez que o projeto visa atender o consumo humano de pessoas que não tem acesso à água potável.
Além de ser uma região que apresenta necessidade de um planejamento para a amenização ou suprimento da escassez de água, é também, uma região muito quente com temperatura média anual de 33Cº, pois está localizada no polígono das Secas (local de maior concentração de seca no país). Consequentemente, o volume de água do Açude de Gargalheiras, açude este que abastece a cidade, decai consideravelmente em épocas de seca, fazendo com que haja escassez de água na cidade. A possibilidade de retirar água potável do subterrâneo, é inviável pois a água é muito salobra. Apesar de clima quente e semi-árido a umidade do ar nessa região  possui uma média anual de 60\%, o que possibilita a implantação de tecnologias que retiram a umidade do ar  e transforma em água potável. Portanto, devido a esses fatores apresentados, a cidade foi escolhida para se realizar o planejamento.

\item Por que se escolheu o bairro Vereador Tarcísio Bezerra Galvão?\\
De acordo com o censo 2010 o bairro Vereador Tarcísio Bezerra Galvão tem cerca de 900 habitantes onde a maioria, cerca de 60\%, possui entre 15 e 64 anos. Sua localização foi uma das principais motivações para a escolha do bairro, pois é um bairro muito próximo do Açude de Gargalheiras, e uma das áreas a ser planejada nesse projeto é em relação à distribuição da água, ou seja, o açude pode facilitar essa distribuição. Outro motivo seria para limitar um pouco mais o projeto a uma população menor, para assim agilizar e facilitar o estudo para planejamento.
\end{enumerate}
  
\section{Nome do gerente do projeto, suas responsabilidades e sua autoridade }
Gerente de Projeto – Adrianny Viana de Araújo Amorim: responsável pelo projeto; realizar a gestão da mudança, escopo, custo, qualidade e os recursos que serão compartilhados entre os vários setores do projeto; selecionar e adaptar os processos de gerenciamento de projetos mais apropriados para a realidade da gerência, na medida da necessidade; dirigir e liderar a equipe, almejando a realização dos objetivos e metas; acompanhar a maturidade da equipe em gerenciamento de projetos, identificando necessidades de orientação e treinamentos; aprovar o plano de cada parte do projeto e autoriza sua execução; definir documentos padrões, base de dados e ferramentas; convocar e coordenar reuniões do projeto; administrar conflitos; definir o escopo do produto.

\section{Necessidades básicas do trabalho a ser realizado }
Para a realização do projeto é necessária uma boa gestão para que todos os envolvidos tenham o máximo de aproveitamento do tempo dedicado. Organização e cumprimento de prazos bem como respeito as lideranças escolhidas. E extremamente importante que o trabalho seja realizado em um ambiente favorável ao desenvolvimento de ideias e que possua recursos que visem possibilitar e facilitar o andamento do projeto. O acesso a internet e indispensável para que possamos manter uma comunicação rápida e eficaz. A utilização de computadores ajuda no tempo de execução das tarefas, possibilitando pesquisas mais rápidas e um melhor gerenciamento das informações do projeto.
Aliado a tudo descrito a cima, a competência de toda a equipe em suas áreas de atuação.


\section{Detalhes do projeto}
\begin{enumerate}
\item Produto do projeto
\begin{itemize}
\item Projeto estrutural mecânico das unidades de captação de água da umidade do ar.
\item Projeto estrutural do sistema de transporte da água para a central de armazenamento.
\item Projeto do sistema de monitoramento e controle da qualidade da água captada.
\item Projeto da matriz energética que dará o suporte para os sistemas de captação de água e monitoramento e controle.
\end{itemize}
\item Cronograma de marcos sumarizado 
\begin{itemize}
\item Pesquisa sobre tecnologias de captação da água inicio 25/03/2015
\item Pesquisar sobre local para a implantação do projeto 25/03/2015 
\item Pesquisar informações sobre o local (clima, tempo, população...)30/03/201
\item Pesquisar sobre localidade específica  06/04/2015
\item Fazer análise do escopo (5W/2H) 13/04/2015
\item Criar Termo de abertura do projeto 20/04/2015
\item Fazer Declaração de escopo 17/04/2015
\item Fazer Relatório do projeto 17/04/2015
\item Ponto de controle 2   30/04/2015
\item Ponto de controle 3   26/05/2015
\end{itemize}
\item Estimativas iniciais de custos \ \\
	Será necessário um investimento na casa dos milhares. Foi levado em conta suporte legal e tributário, suprimentos de escritório, equipamentos, despesas postais, espaço administrativo, salário dos funcionários, entre outros fatores. Com isso temos uma necessidade inicial de aproximadamente R\$ 800 mil, previsto um tempo de implementação de 6 meses.
\end{enumerate}

\section{Administração}
\begin{enumerate}
\item Necessidade inicial de recursos\\
O projeto conta com uma distribuição recursos humanos composta da seguinte forma: 
\begin{itemize}
\item Escopo
\item Qualidade
\item Tempo
\item Integração
\item Custo
\item Comunicação
\item Risco
\item RH
\item Aquisição
\end{itemize}
Cada área possui aproximadamente 3 membros, somando um total de 25 membros na equipe. Dentre as áreas citadas temos a distribuição do pessoal de acordo afinidade e competência estabelecidos. Os membros de cada um dos grupos vem das áreas de Engenharias de Software; Energia; Automotiva; Eletrônica; e Aeroespacial. Também serão necessários profissionais das áreas de engenharia mecânica, engenharia de redes ,advocacia e química dada a necessidade no projeto de conhecimentos específicos nessas áreas.\\
Para equipamentos: computadores, acesso a internet, softwares de gerenciamento, pacote office, acesso a impressão, armazenamento e transporte de dados, local físico para reuniões, suprimentos de escritório ( caneta, papel, quadro branco, pincel para quadro branco, mesa, cadeiras, calculadoras).
\item Necessidade de suporte pela organização\\
Temos a necessidade do apoio de especialistas, na forma de consultoria, nas áreas do conhecimento requisitadas pelo projeto. Os especialistas nos ajudarão com dando uma visão mais critica e coesa sobre determinados aspectos a serem implementados ou direções a serem tomadas no andamento do projeto.

\end{enumerate}
\newpage
\section{Assinaturas}
\begin{center}
Data: \rule{0.5cm}{0.1mm}/\rule{0.5cm}{0.1mm}/\rule{1cm}{0.1mm}     \\
\rule{13cm}{0.1mm}\\
ADRIANNY VIANA DE ARAÚJO AMORIM – GERENTE DE PROJETO\\


\end{center}
% \end{document}