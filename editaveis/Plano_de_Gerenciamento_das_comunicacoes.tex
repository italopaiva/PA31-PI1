\documentclass[12pt,openright,oneside,a4paper,brazil]{abntex2}
\usepackage[utf8]{inputenc}
\counterwithout{section}{section}
\counterwithout{figure}{chapter}
\counterwithout{table}{chapter}
\setlength{\parindent}{1.3cm}
\usepackage{indentfirst}
\setlength{\parskip}{0.2cm}
\usepackage[bottom=2cm,top=3cm,left=3cm,right=2cm]{geometry}
\usepackage{graphicx}
\graphicspath{{figuras/}}
\usepackage{placeins}

%opening
\title{}
\author{}

\begin{document}


\textual
\begin{center}
 {\large Plano de gerenciamento das Comunicações}\\[0.2cm]
 {Planta de abastecimento de água potável a partir da umidade do ar}\\
 \end{center}
 
 \section{Histórico de Alterações}
\begin{table}[h]
\centering
\begin{tabular}{|c|c|p{6cm}|p{5cm}|}

Data & Versão & Descrição & Responsável\\
\hline                               
23/04/2015 & 0.0 & Criação do Plano de Gerenciamento das comunicações & Júlio César\\
\hline
24/04/2015 & 1.0 & Alteração: Incluído a descrição dos processos de gerenciamento das comunicações, eventos de comunicação, atas de reunião e administração dos plano de gerenciamento de comunicação. & 
Júlio César\\
\hline
\end{tabular}
\end{table}

\section{Objetivo}
O objetivo desse plano é estabelecer um processo de gerenciamento das comunicações do projeto.

\section{Descrição dos processos de gerenciamento das comunicações}
O processo de gerenciamento das comunicações é uma das áreas mais importantes para o gerenciamento de projetos, pois é a ligação entre as pessoas, as ideias e as informações. Além disso, a maioria dos problemas que ocorrem ao longo de um projeto é oriundo da falta de comunicação que gera o não entendimento entre as partes envolvidas cominando em um provável fracasso. O gerenciamento das comunicações do projeto inclui os processos necessários para garantir que as informações do projeto sejam geradas, coletadas, distribuídas, armazenadas, recuperadas e organizadas de forma eficiente. 

	As informações serão geradas através das diversas linhas de pesquisas, originadas em discussões e debates sobre o tema de interesse e serão tratadas em reuniões por todos da equipe. Estas informações também serão postadas, armazenadas e organizadas em nossos meios de comunicação virtual, que são realizados por meio da ferramenta de gerenciamento de projetos Trello, pela rede social Facebook e pelo serviço de armazenamento em nuvem gratuito Google Drive, além de ter todos os documentos salvos em um Pendrive.

\section{Eventos de comunicação}
O projeto terá os seguintes eventos de comunicação:
\begin{table}[h]
\centering
\begin{tabular}{|p{5cm}|p{10cm}|}
Evento & Reunião Presencial\\
\hline
Objetivo & Verificar e informar o que foi feito, o que deverá ser feito e se há ou houve algum empecilho no decorrer das atividades\\
\hline
Metodologia & Encontros semanais.\\
\hline
Responsável & Adrianny Viana de Araújo Amorim\\
\hline
Envolvidos & Grupo I de Projeto Integrador 1\\
\hline
Data e Horário e/ou Freqüência & Segunda-feira e quarta-feira\\
\hline
Duração & 1 hora e 40 minutos\\
\hline
Local & Universidade de Brasília - Campus Gama

Segunda-feira: Sala I9

Quarta-feira: Sala I5\\
\hline
Outros & Reuniões realizadas em dias úteis

Verificação de presentes

Ausência mediante justificativa plausível\\
\hline

\end{tabular}
\end{table}

\section{Atas de Reunião}
A ata de reunião é uma ferramenta que torna as reuniões mais eficientes e produtivas. Isto porque as decisões são anotadas e as atividades com seus respectivos responsáveis são acompanhadas. Informações essenciais são registadas na ata para que pessoas que não compareceram às reuniões possam saber do que se foi tratado, não havendo assim a necessidade de se perder tempo na próxima reunião repetindo fatos ocorridos. Essas informações também servem para facilitar a lembrança de temas pelos participantes, servir de bases para futuras discussões ou desentendimentos, forçar uma maior clareza sobre o que foi decidido, saber o que ainda falta fazer e poder se antecipar antes mesmo da reunião seguinte.

\section{Relatórios do projeto}
Os principais relatórios a serem publicados no sistema de informações do projeto estão apresentados em anexo.

\section{Ambiente técnico e estrutura de armazenamento e distribuição da informação (EPM)}
O ambiente técnico físico se concentra no Campus Gama da Universidade de Brasília, onde as discussões tratadas são registradas e armazenadas em computadores por meio de fotos e anotações e distribuídas virtualmente por meio de nossas ferramentas para que todos da equipe possam ter acesso e se manterem alinhados com o caminhar do projeto.

\section{Administração do plano de gerenciamento das comunicações}
\begin{enumerate}

\item Responsável pelo plano:
\begin{itemize}

\item Júlio César Tavares Primo - Gerente de Comunicação
\item Vítor Ferreira Pacífico - Subgerente de Comunicação
\item Vitor Silva Ribeiro - Coordenador de Comunicação

\end{itemize}
\item Frequência de atualização do plano de gerenciamento das comunicações

	A frequência de atualização será realizada semanalmente de acordo com as alterações e acréscimos provindas das decisões das reuniões.
\end{enumerate}

\section{Outros assuntos relacionados ao gerenciamento das comunicações do projeto não previstos nesse plano}
A comunicação interna entre os subgrupos designados para os específicos modelos de pesquisa

\section{Assinaturas}
\begin{center}
Data: \rule{0.5cm}{0.1mm}/\rule{0.5cm}{0.1mm}/\rule{1cm}{0.1mm}     \\
\rule{13cm}{0.1mm}\\
ADRIANNY VIANA DE ARAÚJO AMORIM – GERENTE DE PROJETO\\
\rule{13cm}{0.1mm}\\
JÚLIO CÉSAR TAVARES PRIMO - GERENTE DE COMUNICAÇÃO


\end{center}
\end{document}