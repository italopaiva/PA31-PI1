Considerando que nossa empresa deverá distribuir água com qualidade suficiente para consumo humano, baseamos no decreto nº7.217, de de 21 de junho de 2010, que regulamenta a lei nº 11.445, de 5 de janeiro de 2007, que estabelece diretrizes nacionais para o saneamento básico.

Art. 1º Esta portaria dispõe sobre os procedimentos de controle e de vigilância de qualidade da água para consumo humano e seu padrão de potabilidade.

Art. 2º Esta portaria se aplica à água destinada ao consumo humano proveniente, de sistema e solução alternativa de abastecimento de água.

Art. 3º Toda água destinada ao consumo humano distribuída coletivamente por meio de sistema ou solução alternativa coletiva de abastecimento de água, deve ser objeto de controle e vigilância de qualidade da água.

Art. 4º Toda água destinada ao consumo humano proveniente de solução alternativa individual de abastecimento de água independentemente da forma de acesso da população, está sujeita à vigilância de qualidade de água.

Art. 5º Para fins desta portaria, são adotadas as seguintes definições:

I - Água para consumo humano: água potável destinada à ingestão, preparação e produção de alimentos e à higiene pessoal, independentemente da sua origem.

II – Agua potável: água que atenda ao padrão de potabilidade estabelecido nesta Portaria e que não ofereça riscos à saúde.

III - Padrão de potabilidade: conjunto de valores permitidos como parâmetro da qualidade da água para consumo humano, conforme definido nesta Portaria.

IV - Padrão organoléptico: conjunto de parâmetros caracterizados por provocar estímulos sensoriais que afetam a aceitação para consumo humano, mas que não necessariamente implicam riscos à saúde.

V - Água tratada: água submetida a processos físicos, químicos ou combinação destes, visando atender ao padrão de potabilidade.

VI – Sistema de abastecimento de água para consumo humano: instalação composta por um conjunto de obras civis, materiais e equipamentos, desde a zona de captação até as ligações prediais, destinada à produção e ao fornecimento coletivo de água potável, por meio de rede de distribuição.

VII - Solução alternativa coletiva de abastecimento de água para consumo humano: modalidade de abastecimento coletivo destinada a fornecer água potável, com captação subterrânea ou superficial, com ou sem canalização e sem rede de distribuição.

VIII - Solução alternativa individual de abastecimento de água para consumo humano: modalidade de abastecimento de água para consumo humano que atenda a domicílios residenciais com uma única família, incluindo seus agregados familiares.

IX - Rede de distribuição: parte do sistema de abastecimento formada por tubulações e seus acessórios, destinados a distribuir água potável, até as ligações prediais.

X - Ligações prediais: conjunto de tubulações e peças especiais, situado entre a rede de distribuição de água e o cavalete, este incluído.

XI - Cavalete: kit formado por tubos e conexões destinados à instalação do hidrômetro para realização da ligação de água.

XII - Interrupção: situação na qual o serviço de abastecimento de água é interrompido temporariamente, de forma programada ou emergencial, em razão da necessidade de se efetuar reparos, modificações ou melhorias no respectivo sistema.

XIII - Intermitência: é a interrupção do serviço de abastecimento de água, sistemática ou não, que se repete ao longo de determinado período, com duração igual ou superior a seis horas em cada ocorrência.

XIV - Integridade do sistema de distribuição: condição de operação e manutenção do sistema de distribuição (reservatório e rede) de água potável em que a qualidade da água produzida pelos processos de tratamento seja preservada até as ligações prediais.

XV - Controle da qualidade da água para consumo humano: conjunto de atividades exercidas regularmente pelo responsável pelo sistema ou por solução alternativa coletiva de abastecimento de água, destinado a verificar se a água fornecida à população é potável, de forma a assegurar a manutenção desta condição.

XVI - Vigilância da qualidade da água para consumo humano: conjunto de ações adotadas regularmente pela autoridade de saúde pública para verificar o atendimento a esta Portaria, considerados os aspectos socioambientais e a realidade local, para avaliar se a água consumida pela população apresenta risco à saúde humana.

XVII - Garantia da qualidade: procedimento de controle da qualidade para monitorar a validade dos ensaios realizados.

XVIII - Recoleta: ação de coletar nova amostra de água para consumo humano no ponto de coleta que apresentou alteração em algum parâmetro analítico.

XIX - Passagem de fronteira terrestre: local para entrada ou saída internacional de viajantes, bagagens, cargas, contêineres, veículos rodoviários e encomendas postais.

