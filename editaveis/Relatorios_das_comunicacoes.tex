%  \documentclass[12pt,openright,oneside,a4paper,brazil]{abntex2}
%  \usepackage[utf8]{inputenc}
%  \counterwithout{section}{section}
%  \counterwithout{figure}{chapter}
%  \counterwithout{table}{chapter}
%  \setlength{\parindent}{1.3cm}
%  \usepackage{indentfirst}
%  \setlength{\parskip}{0.2cm}
%  \usepackage[bottom=2cm,top=3cm,left=3cm,right=2cm]{geometry}
%  \usepackage{graphicx}
%  \graphicspath{{figuras/}}
%  \usepackage{placeins}
% 
% % opening
% % \title{}
% % \author{}
% 
%  \begin{document}
% 
% 
%  \textual
\begin{center}
{\large Relatório de comunicação 1}
\begin{table}[h]
\begin{tabular}{|p{6cm}|p{9cm}|}\hline
Relatorio & Relatório da Primeira Semana de reuniões; Reunião realizada dia 25/03/2015.\\ \hline
Objetivo & Definir Escopo do trabalho, amadurecer conceito da Teoria dos 5w 2h.\\ \hline
Informações & Na primeira semana, com reunião única foi definida o local pra implementação do projeto, as áreas de pesquisa a respeito do local assim como o “O que pesquisar” foi definido.\\ \hline
Responsável & Vitor Silva Ribeiro\\ \hline
Destinatários & Integrantes do Projeto, Professores Coordenadores.\\ \hline
Data e Horário e/ou Freqüência & Data: 25/03/2015 com horário de inicio 17h30min com termino as 18h00min\\ \hline
Local de Armazenagem & Relatório ficara armazenado facebook, no trelo, e no Google Doc.\\ \hline
Outros & O Relatório da comunicação será realizado semanalmente.\\ \hline

\end{tabular}
\end{table}

{\large Relatório de comunicação 2}
\begin{table}[h]
\begin{tabular}{|p{6cm}|p{9cm}|}\hline
Relatorio & Relatório da Segunda Semana de reuniões; Reunião realizada dia 01/04/2015.\\ \hline
Objetivo & A reunião tinha como objetivo explicar,detalhar e exemplificar a utilização da ferramenta de controle escolhida anteriormente (Trello).Separar duplas de pesquisa pra definição do tema (“O que”).\\ \hline
Informações & Na segunda semana, depois de definido os meios de comunicação, e os integrantes do grupo terem aderido ao mesmo, foi possível dar um passo a frente em relação ao tema, duplas de pesquisa foram formadas e um dos 5hs foi escolhido como ponto de partida (What).\\ \hline
Responsável & Vitor Silva Ribeiro\\ \hline
Destinatários & Integrantes do Projeto, Professores Coordenadores.\\ \hline
Data e Horário e/ou Freqüência & Data: 01/04/2015 com horário de inicio 17h42min com termino as 18h00min\\ \hline
Local de Armazenagem & Relatório ficara armazenado facebook, no trelo, e no Google Doc.\\ \hline
Outros & O Relatório da comunicação será realizado semanalmente.\\ \hline
\end{tabular}
\end{table}
\newpage

{\large Relatório de comunicação 3}
\begin{table}[h]
\begin{tabular}{|p{6cm}|p{9cm}|}\hline
Relatorio & Relatório da Terceira Semana de reuniões; Reunião realizada dia 06/04/2015\\ \hline
Objetivo & Apresentação das Pesquisas realizadas após a definição das duplas na ultima reunião. BrainStorm com foco  no escopo, tendo uma base sólida nas pesquisas.\\ \hline
Informações & Na terceira semana de reuniões, foram apresentadas idéias sobre dois locais de possível implementação do projeto, a plataforma de petróleo, e a cidade de Aracarí-Rn. As idéias foram implementas com pros e contras para cada ponto.\\ \hline
Responsável & Vitor Silva Ribeiro\\ \hline
Destinatários & Integrantes do Projeto, Professores Coordenadores.\\ \hline
Data e Horário e/ou Freqüência &Data: 06/04/2015 com horário de inicio 17h30min com termino as 18h00min\\ \hline
Local de Armazenagem & Relatório ficara armazenado facebook, no trelo, e no Google Doc.\\ \hline
Outros & O Relatório da comunicação será realizado semanalmente.\\ \hline
\end{tabular}
\end{table}

{\large Relatório de comunicação 4}
\begin{table}[h]
\begin{tabular}{|p{6cm}|p{9cm}|}\hline
Relatorio & Relatório da Terceira Semana de reuniões; Reuniões realizadas dias 13/04/2015 e 15/04/2015.\\ \hline
Objetivo & A quarta semana teve como principal objetivo definição do escopo, e a organização e separação/escolha, dos integrantes de acordo com as áreas de conhecimento do PMBok.\\ \hline
Informações & Na quarta semana, as discussões foram voltadas para a finalização do escopo, e a separação dos 9 grupos de acordo com o PMBok. Uma vez definidos, cada grupo ficou responsável pela entrega de seu próprio relatório.
Após as reuniões dessa semana ficaram claros os problemas iniciais. e as tecnologias empregadas no desenvolvimento do projeto foram filtradas, restando apenas uma.
Surgiu também uma base para a criação do FishBone.\\ \hline
Responsável & Vitor Silva Ribeiro\\ \hline
Destinatários & Integrantes do Projeto, Professores Coordenadores.\\ \hline
Data e Horário e/ou Freqüência & Data: 13/04/2015 com horário de inicio 17h30min com termino as 18h00min e 15/04/2015 com horário de inicio 16h13min com  termino as 17h50min\\ \hline
Local de Armazenagem & Relatório ficara armazenado facebook, no trelo, e no Google Doc.\\ \hline
Outros & O Relatório da comunicação será realizado semanalmente.\\ \hline
\end{tabular}
\end{table}

\end{center}

% \end{document}