%  \documentclass[12pt,openright,oneside,a4paper,brazil]{abntex2}
%  \usepackage[utf8]{inputenc}
%  \counterwithout{section}{section}
%  \counterwithout{figure}{chapter}
%  \counterwithout{table}{chapter}
%  \setlength{\parindent}{1.3cm}
%  \usepackage{indentfirst}
%  \setlength{\parskip}{0.2cm}
%  \usepackage[bottom=2cm,top=3cm,left=3cm,right=2cm]{geometry}
%  \usepackage{graphicx}
%  \graphicspath{{figuras/}}
%  \usepackage{placeins}
% 
% % opening
% % \title{}
% % \author{}
% 
%  \begin{document}
% 
% 
%  \textual
\begin{center}
{\large Relatório de comunicação 1}
\begin{table}[h]
\begin{tabular}{|p{6cm}|p{9cm}|}\hline
Relatorio & Relatório da Primeira Semana de reuniões; Reunião realizada dia 25/03/2015.\\ \hline
Objetivo & Definir Escopo do trabalho, amadurecer conceito da Teoria dos 5w 2h.\\ \hline
Informações & Na primeira semana, com reunião única foi definida o local pra implementação do projeto, as áreas de pesquisa a respeito do local assim como o “O que pesquisar” foi definido.\\ \hline
Responsável & Vitor Silva Ribeiro\\ \hline
Destinatários & Integrantes do Projeto, Professores Coordenadores.\\ \hline
Data e Horário e/ou Freqüência & Data: 25/03/2015 com horário de inicio 17h30min com termino as 18h00min\\ \hline
Local de Armazenagem & Relatório ficara armazenado facebook, no trelo, e no Google Doc.\\ \hline
Outros & O Relatório da comunicação será realizado semanalmente.\\ \hline

\end{tabular}
\end{table}

{\large Relatório de comunicação 2}
\begin{table}[h]
\begin{tabular}{|p{6cm}|p{9cm}|}\hline
Relatorio & Relatório da Segunda Semana de reuniões; Reunião realizada dia 01/04/2015.\\ \hline
Objetivo & A reunião tinha como objetivo explicar,detalhar e exemplificar a utilização da ferramenta de controle escolhida anteriormente (Trello).Separar duplas de pesquisa pra definição do tema (“O que”).\\ \hline
Informações & Na segunda semana, depois de definido os meios de comunicação, e os integrantes do grupo terem aderido ao mesmo, foi possível dar um passo a frente em relação ao tema, duplas de pesquisa foram formadas e um dos 5hs foi escolhido como ponto de partida (What).\\ \hline
Responsável & Vitor Silva Ribeiro\\ \hline
Destinatários & Integrantes do Projeto, Professores Coordenadores.\\ \hline
Data e Horário e/ou Freqüência & Data: 01/04/2015 com horário de inicio 17h42min com termino as 18h00min\\ \hline
Local de Armazenagem & Relatório ficara armazenado facebook, no trelo, e no Google Doc.\\ \hline
Outros & O Relatório da comunicação será realizado semanalmente.\\ \hline
\end{tabular}
\end{table}
\newpage

{\large Relatório de comunicação 3}
\begin{table}[h]
\begin{tabular}{|p{6cm}|p{9cm}|}\hline
Relatorio & Relatório da Terceira Semana de reuniões; Reunião realizada dia 06/04/2015\\ \hline
Objetivo & Apresentação das Pesquisas realizadas após a definição das duplas na ultima reunião. BrainStorm com foco  no escopo, tendo uma base sólida nas pesquisas.\\ \hline
Informações & Na terceira semana de reuniões, foram apresentadas idéias sobre dois locais de possível implementação do projeto, a plataforma de petróleo, e a cidade de Aracarí-Rn. As idéias foram implementas com pros e contras para cada ponto.\\ \hline
Responsável & Vitor Silva Ribeiro\\ \hline
Destinatários & Integrantes do Projeto, Professores Coordenadores.\\ \hline
Data e Horário e/ou Freqüência &Data: 06/04/2015 com horário de inicio 17h30min com termino as 18h00min\\ \hline
Local de Armazenagem & Relatório ficara armazenado facebook, no trelo, e no Google Doc.\\ \hline
Outros & O Relatório da comunicação será realizado semanalmente.\\ \hline
\end{tabular}
\end{table}

{\large Relatório de comunicação 4}
\begin{table}[h]
\begin{tabular}{|p{6cm}|p{9cm}|}\hline
Relatorio & Relatório da Terceira Semana de reuniões; Reuniões realizadas dias 13/04/2015 e 15/04/2015.\\ \hline
Objetivo & A quarta semana teve como principal objetivo definição do escopo, e a organização e separação/escolha, dos integrantes de acordo com as áreas de conhecimento do PMBok.\\ \hline
Informações & Na quarta semana, as discussões foram voltadas para a finalização do escopo, e a separação dos 9 grupos de acordo com o PMBok. Uma vez definidos, cada grupo ficou responsável pela entrega de seu próprio relatório.
Após as reuniões dessa semana ficaram claros os problemas iniciais. e as tecnologias empregadas no desenvolvimento do projeto foram filtradas, restando apenas uma.
Surgiu também uma base para a criação do FishBone.\\ \hline
Responsável & Vitor Silva Ribeiro\\ \hline
Destinatários & Integrantes do Projeto, Professores Coordenadores.\\ \hline
Data e Horário e/ou Freqüência & Data: 13/04/2015 com horário de inicio 17h30min com termino as 18h00min e 15/04/2015 com horário de inicio 16h13min com  termino as 17h50min\\ \hline
Local de Armazenagem & Relatório ficara armazenado facebook, no trelo, e no Google Doc.\\ \hline
Outros & O Relatório da comunicação será realizado semanalmente.\\ \hline
\end{tabular}
\end{table}
\FloatBarrier

{\large Relatório de comunicação 5}
\begin{table}[h]
\begin{tabular}{|p{6cm}|p{9cm}|}\hline
Relatório&Relatório da sexta semana de reuniões; Reunião realizada dia06/05/2015.\\ \hline
Objetivo&Revisar as pendências que ficaram para o ponto de controle 2 e observações feitas pelos orientadores.\\ \hline
Informações&	Explicação sobre o Scrum; Reunião dos gerentes e sub-gerentes com os orientadores;
Definição do Release Backlog (escopo do segundo ponto de controle);\\ \hline
Responsável	&Júlio César Tavares Primo\\ \hline
Destinatários&Integrantes do Projeto, Professores Coordenadores.\\ \hline
Data e Horário e/ou Freqüência& Data: 06/05/2015 com horário de inicio 16h30min com termino as 18h00min\\ \hline
Local de Armazenagem&Relatório ficará armazenado no Google Doc.\\ \hline
Outros&	O Relatório da comunicação será realizado semanalmente.\\ \hline
\end{tabular}
\end{table}
\FloatBarrier


{\large Relatório de comunicação 6}
\begin{table}[h]
\begin{tabular}{|p{6cm}|p{9cm}|}\hline
Relatório&	Relatório da sétima semana de reuniões; Reunião realizada dia 11/05/2015.\\ \hline
Objetivo	&Definir a quantidade de turbinas, suas disposições e demanda; Verificar resultados de pesquisa de cada frente.\\ \hline
Informações&	Discussão sobre as dimensões das turbinas e da área; Definições de atividades das frentes; Reunião dos professores com gerentes e subgerentes;\\ \hline
Responsável &	Júlio César Tavares Primo\\ \hline
Destinatários&Integrantes do Projeto, Professores Coordenadores.\\ \hline
Data e Horário e/ou Freqüência	& Data: 11/05/2015 com horário de inicio 16h10min com termino as 18h10min\\ \hline
Local de Armazenagem&Relatório ficará armazenado no Google Doc.\\ \hline
Outros&O Relatório da comunicação será realizado semanalmente.\\ \hline

\end{tabular}
\end{table}
\FloatBarrier

{\large Relatório de comunicação 7}
\begin{table}[h]
\begin{tabular}{|p{6cm}|p{9cm}|}\hline
Relatório&Relatório da oitava semana de reuniões; Reunião realizada dia18/05/2015.\\ \hline
Objetivo&Definição das dimensões de todos os componentes, preenchimento do formulário de acompanhamento de projeto (avaliação individual).\\ \hline
Informações &	Equipe de software terminando o levantamento de requisitos;
Todas as frentes reunidas para verificar os feitos e pendências;
Levantamento da viabilidade das torneiras (distribuição).\\ \hline
Responsável	&Júlio César Tavares Primo\\ \hline
Destinatários	&Integrantes do Projeto, Professores Coordenadores.\\ \hline
Data e Horário e/ou Freqüência	& Data: 18/05/2015 com horário de inicio 16h15min com termino as 17h55min\\ \hline
Local de Armazenagem&Relatório ficará armazenado no Google Doc.\\ \hline
Outros&O Relatório da comunicação será realizado semanalmente.\\ \hline
\end{tabular}
\end{table}
\FloatBarrier


{\large Relatório de comunicação 8}
\begin{table}[h]
\begin{tabular}{|p{6cm}|p{9cm}|}\hline
Relatório&Relatório da nona semana de reuniões; Reunião realizada dia20/05/2015.\\ \hline
Objetivo&Definiçãodas dimensões de todos os componentes;checagem de atividades feitas e pendências para o fechamento da sprint; \\ \hline
Informações&Ponto de controle dia 01/06 e 03/06;
Sprint finaliza dia 21/05;
Quarta a Sexta 12h as 14h monitores na mocap;
Livro de mecânica encontrado na biblioteca.\\ \hline
Responsável&Júlio César Tavares Primo\\ \hline
Destinatários&Integrantes do Projeto, Professores Coordenadores.\\ \hline
Data e Horário e/ou Freqüência&Data: 20/05/2015 com horário de inicio 16h05min com termino às 18h00min\\ \hline
Local de Armazenagem&Relatório ficará armazenado no Google Doc.\\ \hline
Outros&O Relatório da comunicação será realizado semanalmente.\\ \hline

\end{tabular}
\end{table}
\FloatBarrier

{\large Relatório de comunicação 9}
\begin{table}[h]
\begin{tabular}{|p{6cm}|p{9cm}|}\hline
Relatório&Relatório da décima semana de reuniões; Reunião realizada dia25/05/2015.\\ \hline
Objetivo&Definição da sprint 3;
Elaboração de relatório com feitos e pendências;
Preenchimento da auto-avaliação;\\ \hline
Informações	& Ponto de controle dia 01/06 e 03/06;
Entrega de relatório para quinta-feira (28/05);\\ \hline
Responsável&Júlio César Tavares Primo\\ \hline
Destinatários&Integrantes do Projeto, Professores Coordenadores.\\ \hline
Data e Horário e/ou Freqüência&Data: 25/05/2015 com horário de inicio 16h05min com termino às 18h05min\\ \hline
Local de Armazenagem&Relatório ficará armazenado no Google Doc.\\ \hline
Outros&O Relatório da comunicação será realizado semanalmente.\\ \hline
\end{tabular}
\end{table}
\FloatBarrier

{\large Relatório de comunicação 10}
\begin{table}[h]
\begin{tabular}{|p{6cm}|p{9cm}|}\hline
Relatório&Relatório da décima semana de reuniões; Reunião realizada dia27/05/2015.\\ \hline
Objetivo&Diagramas de fluxos da água e sua distribuição;
Discussão de pendências com o orientador;
Atualização dos planos, cronogramas e escopo;
Dimensionamento da torre;
Encerramento das atividades da sprint 3 de cada frente;
Preenchimento da auto-avaliação.\\ \hline
Informações&Requisitados as tensões das ventoinhas;
Especificações do gerador e do compressor pedidas;\\ \hline
Responsável&Júlio César Tavares Primo\\ \hline
Destinatários&Integrantes do Projeto, Professores Coordenadores.\\ \hline
Data e Horário e/ou Freqüência&Data: 27/05/2015 com horário de inicio 16h15min com termino às 17h50min\\ \hline
Local de Armazenagem&Relatório ficará armazenado no Google Doc.\\ \hline
Outros&O Relatório da comunicação será realizado semanalmente.\\ \hline
\end{tabular}
\end{table}
\FloatBarrier

{\large Relatório de comunicação 11}
\begin{table}[h]
\begin{tabular}{|p{6cm}|p{9cm}|}\hline
Relatório&Relatório da quinta Semana de reuniões; Reuniões realizadas dias 27/04/2015 e 29/04/2015.\\ \hline
Objetivo&Finalizar as tarefas referentes ao primeiro ponto de controle e dar continuidade ao projeto.\\ \hline
Informações&	Apresentação realizada para o dia 29/04.
Requisitos do sistema levantados pelo grupo de automotiva e aeroespacial.
Referencial teórico finalizado pelo pessoal de eletrônica.
Atualização no Latex por parte da gerência.\\ \hline
Responsável&Vitor Silva Ribeiro\\ \hline
Destinatários&Integrantes do Projeto, Professores Coordenadores.\\ \hline
Data e Horário e/ou Freqüência	&Data: 27/04/2015 com horário de inicio 16h30min com termino as 18h00min e 29/04/2015 com horário de inicio h min com termino as h min\\ \hline
Local de Armazenagem&Relatório ficara armazenado no Google Doc.\\ \hline
Outros	& O Relatório da comunicação será realizado semanalmente.\\ \hline
\end{tabular}
\end{table}
\FloatBarrier

{\large Relatório de comunicação 12}
\begin{table}[h]
\begin{tabular}{|p{6cm}|p{9cm}|}\hline
Relatório&Relatório da décima primeira semana de reuniões; Reunião realizada dia 10/06/2015.\\ \hline
Objetivo&Sprint 5 (08/06-15/06);
Separar entre as frente as atividades pendentes do PCII;
Subsistemas de Software;
Alinhar o andamento das pesquisas e atividades;
Avaliação e melhoria do projeto.
Cálculo de mecânica dos sólidos aplicados.\\ \hline
Informações	& Orientações da Eneida e Edgar.\\ \hline
Responsável	& Júlio César Tavares Primo\\ \hline
Destinatários& Integrantes do Projeto, Professores Coordenadores.\\ \hline
Data e Horário e/ou Freqüência&Data: 10/06/2015 com horário de inicio 17h10min com termino as 18h00min.\\ \hline
Local de Armazenagem&Relatório ficara armazenado facebook, no trelo, e no Google Doc.\\ \hline
Outros&O Relatório da comunicação será realizado semanalmente.\\ \hline

\end{tabular}
\end{table}
\FloatBarrier

{\large Relatório de comunicação 13}
\begin{table}[h]
\begin{tabular}{|p{6cm}|p{9cm}|}\hline
Relatório&Reunião realizada dia 15/06/2015.\\ \hline
Objetivo&Verificação das atividades feitas por cada frente;
Alinhamento em relação ao LATEX;
Pendências distribuídas;\\ \hline
Informações&Análise do decorrer do projeto debatido presencialmente;
Ponto de controle adiado para 25/06/2015;
Intenso debate com os orientadores de cada frente.\\ \hline
Responsável	&Júlio César Tavares Primo\\ \hline
Destinatários	&Integrantes do Projeto, Professores Coordenadores.\\ \hline
Data e Horário e/ou Freqüência&Data: 15/06/2015 com horário de inicio 16h00min com término as 18h00min.\\ \hline
Local de Armazenagem&Relatório ficara armazenado facebook, no trello, e no Google Doc.\\ \hline
Outros	&O Relatório da comunicação será realizado semanalmente.\\ \hline

\end{tabular}
\end{table}
\FloatBarrier

\end{center}


% \end{document}