Sabendo-se que o sistema de refrigeração tem a parte de alta temperatura e o de baixa, é necessário entender quem são os intermediários entre essas duas partes. Da baixa temperatura e pressão para alta temperatura e pressão, está o compressor. Já para intermediar o líquido refrigerante quente do gás gelado, temos a válvula de expansão. 

O funcionamento de uma válvula de pressão tem por finalidade reduzir a pressão do refrigerante e controlar o fluxo de massa que entrará no evaporador. Assim sendo, ele é essencial para um funcionamento perfeito, visto que ele deve manter um superaquecimento constante para evitar que o haja a entrada de líquido no compressor. 

Para a escolha específica da válvula, deverá ser analisado todo o conjunto do sistema para uma melhor determinação das especificações do produto. Analisando a nomenclatura, já se pode dizer que será usado uma válvula de equalização de pressão externa, visto que são comumente usados para sistemas mais robustos e que demandam grande quantidade de liquido refrigerante pelo sistema\cite{compressors_for_refrigeration}. 

Além disso, a escolha de uma válvula está ligada ao compressor, visto que, se escolhermos um circuito dotado de tubos capilares, as pressões antes e depois são iguais mesmo quando o compressor é desligado, isto da chance de ter um motor com baixo torque de partida. Já um circuito com válvula de expansão terá suas pressões iguais somente quando o compressor estiver ligado, requerendo um motor com alto torque de partida\cite{embracovalvula}. 

Não definiu-se ainda o fabricante da peça, visto que há vários fornecedores estrangeiros, como a Emerson Climate Technology ou a Danfoss, ou fornecedores nacionais como a Embraco. A média de preço de uma válvula do tipo de pressão externa varia entre R\$ 100,00 a R\$ 350,00.
    