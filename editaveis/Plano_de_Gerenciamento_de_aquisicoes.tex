% \documentclass[12pt,openright,oneside,a4paper,brazil]{abntex2}
% \usepackage[utf8]{inputenc}
% \counterwithout{section}{section}
% \counterwithout{figure}{chapter}
% \counterwithout{table}{chapter}
% \setlength{\parindent}{1.3cm}
% \usepackage{indentfirst}
% \setlength{\parskip}{0.2cm}
% \usepackage[bottom=2cm,top=3cm,left=3cm,right=2cm]{geometry}
% \usepackage{graphicx}
% \graphicspath{{figuras/}}
% \usepackage{placeins}

% opening
% \title{}
% \author{}

% \begin{document}


% \textual
\begin{center}
 {\large Plano de gerenciamento das Aquisições}\\[0.2cm]
 {Planta de abastecimento de água potável a partir da umidade do ar}\\
 \end{center}
 
 \section*{Histórico de Alterações}
\begin{table}[h]
\centering
\begin{tabular}{|c|c|p{6cm}|p{5cm}|}

Data & Versão & Descrição & Responsável\\
\hline                               
19/04/2015 & 0.0 & Criação do Plano de Gerenciamento das aquisições e o preenchimento da descrição dos processos e os tipos de contratos. & Gustavo Pereira\\
\hline
20/04/2015 & 1.0 & Alterações do tópico III e o preenchimento dos tópicos IV, V, VI, VII, IX e X. & Gustavo Pereira, Yan Watanabe, Brenda Tagna\\
\hline
22/04/2015 & 2.0 & Atualização e retirada de alguns textos nos itens III e IV. Reorganização dos critérios de avaliação do item VI. & Gustavo Pereira, Yan Watanabe, Brenda Tagna\\
\hline
23/04/2015 & 2.0 & Preenchimento do item VIII de acordo com os requisitos inicias do projeto. & Gustavo Pereira, Yan Watanabe, Brenda Tagna\\
\hline
\end{tabular}
\end{table}

\section*{Objetivo}
  O objetivo desse plano é estabelecer um processo de gerenciamento das aquisições do projeto.
  
\section*{Descrição dos processos de gerenciamento das aquisições}
  Esse processo será desenvolvido por Gustavo Pereira, Yan Watanabe e Brenda Tagna. Esse plano de aquisição tem como objetivo listar todos os serviços e produtos a serem adquiridos por meio de contratos. 
A escolha de um determinado fornecedor ou prestador de serviço, será feita de uma forma em que possamos analisar itens considerados de extrema importância para um acordo contratual, como os listados no item V e VI desse plano. A partir dessa análise poderá ser feita uma escolha que melhor atenda aos requisitos do projeto. 
Todas as aquisições deverão seguir um padrão ético, sempre buscando a melhor escolha para o projeto, sem que haja benefício ilícito. 

   
\section*{Gerenciamento e tipos de contratos}
Para o projeto em questão, serão usados principalmente contratos de Preço Fixo, uma vez que esse tipo de contrato é o mais recomendado para projetos de escopo bem detalhado, como esse. 
Existem algumas variações do contrato de Preço Fixo:
\begin{itemize}

\item PFG: Preço fixo garantido
\item PFRI: PF + Remuneração de Incentivo (ex: prêmio para entrega no prazo, excelência no serviço prestado, etc)
\item PFAEP: PF com Ajuste Econômico do Preço
\end{itemize}

\section*{Critérios de avaliação de cotações e propostas}
\begin{itemize}
\item Preço;
\item Boa referência de serviços prestados
\item Certificaçao de produtos e serviços prestados
\item Forma de pagamento
\end{itemize}

\section*{Avaliação de fornecedores}
\begin{itemize}
\item Adequação aos requisitos do projeto
\item Preço
\item Prazo
\item Assistência Técnica pré (consultoria) e pós compra
\item Forma de pagamento
\item Boa referência de serviços prestados
\item Qualidade
\item Garantia do produto ou serviços
\item Certificação de produtos e serviços prestados
\end{itemize}

\section*{Frequência de avaliação dos processos de aquisição}
A frequência de verificação dependerá de cada contrato. Em um período de duas semanas antes do término do mesmo, será analisada a necessidade de renovação ou encerramento, levando em consideração o cumprimento (ou não) das cláusulas do contrato.

\section*{Alocação financeira para o gerenciamento das aquisições}
Para a alocação financeira  inicial do projeto, será necessário adquirir os seguintes itens:
\begin{table}[h]
\centering
\begin{tabular}{|p{7cm}|c|p{2cm}|p{2.5cm}|}
Itens & Tipo de Contrato & Quantidade & Valor\\
\hline
Aquisição de computadores & Preço Fixo	& 25 & R\$ 30.000,00 \\
\hline
Aquisição do Pacote Office 365 Business Premium (Semestral)	& Preço Fixo & 25 & R\$ 7.350,00\\
\hline
Impressora Multifuncional & Preço Fixo & 1 & R\$ 3.500,00\\
\hline
Plano de Internet (Semestral) & Preço Fixo & 1 & R\$ 720,00\\
\hline
 & & Total & R\$ 41.570,00
\end{tabular}
\end{table}

\section*{Administração do plano de gerenciamento das aquisições}
\begin{enumerate}
\item Responsável pelo plano:
\begin{itemize}
\item Yan Watanabe Martins – gerente de Aquisições
\item Gustavo Pereira – suplente do responsável direto pelo plano de gerenciamento de Aquisições
\end{itemize}
\item Freqüência de atualização do plano de gerenciamento das aquisições:
O Plano de Gerenciamento de Aquisições será revisado em cada reunião. As reuniões ocorrem pelo menos duas vezes por semana. Tal plano poderá ser modificado (ou não) levando em consideração as determinações e avaliações dos resultados de cada membro da equipe. 
\end{enumerate}

\section*{Outros assuntos relacionados ao gerenciamento das aquisições do projeto não previstos nesse plano}
Todas as mudanças referentes ao Plano de Gerenciamento de Aquisições deverão ser apresentadas e discutidas em reunião, sendo a equipe de Aquisição a responsável por tal avaliação. A responsabilidade de possíveis mudanças no quadro pessoal da equipe fica direcionada aos gerentes de projeto.

\section*{Assinaturas}
\begin{center}
Data: \rule{0.5cm}{0.1mm}/\rule{0.5cm}{0.1mm}/\rule{1cm}{0.1mm}     \\
\rule{13cm}{0.1mm}\\
ADRIANNY VIANA DE ARAÚJO AMORIM – GERENTE DE PROJETO\\
\rule{13cm}{0.1mm}\\
YAN WATANABE - GERENTE DE AQUISIÇÕES

\end{center}
% \end{document}