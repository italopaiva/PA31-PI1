% \documentclass[12pt,openright,oneside,a4paper,brazil]{abntex2}
% \usepackage[utf8]{inputenc}
% \counterwithout{section}{section}
% \counterwithout{figure}{chapter}
% \counterwithout{table}{chapter}
% \setlength{\parindent}{1.3cm}
% \usepackage{indentfirst}
% \setlength{\parskip}{0.2cm}
% \usepackage[bottom=2cm,top=3cm,left=3cm,right=2cm]{geometry}
% \usepackage{graphicx}
% \graphicspath{{figuras/}}
% \usepackage{placeins}

% opening
% \title{}
% \author{}

% \begin{document}


% \textual
\begin{center}
 {\large Plano de gerenciamento de Custos}\\[0.2cm]
 {Planta de abastecimento de água potável a partir da umidade do ar}\\
 \end{center}
 
 \section*{Histórico de Alterações}
\begin{table}[h]
\centering
\begin{tabular}{|c|c|p{6cm}|p{5cm}|}

Data & Versão & Descrição & Responsável\\
\hline                               
17/04/2015 & 0.0 & Criação deste documento para descrever todo planejamento de custo do projeto. & Guilherme Matias, João Gabriel, Pedro Kelvin.\\ \hline
\end{tabular}
\end{table}

\section*{Objetivo}

O objetivo desse plano é estabelecer um processo de gerenciamento de custos do projeto
  
\section*{Descrição dos processos de gerenciamento de custo}

A gerência do custo do projeto agrega os processos que envolvem planejamento, estimativa, orçamento e controle de custos que serão necessários para a conclusão do projeto a partir de uma previsão orçamentária. Onde a estimativa de custo desenvolve uma aproximação dos gastos com recursos necessários para a elaboração do projeto, quantoao orçamento, agregar os custos estimados de atividades ou de pacotes individuais de trabalho para estabeleceru ma base de custo, e o controle influência nos fatores que geram uma variação de custo e controlar as mudanças de orçamento de projeto. O gerenciamento de custos será analisado desde a fase de iniciação do projeto até o seu encerramento. O processo de gerenciamento será feito através de dados queiram dizer se a relação custo/benefício está sendo viável ou não para o projeto, analisando o custos de: tecnologia utilizada para a condensação do ar, estrutura da planta, tempo de retorno do valor investido. Será usado a técnica do PMBOK para gerir a estimativa de custo e orçamento do projeto. Serão analisados para saber em que vai influenciar do decorrer do projeto, e então corrigí-lo.

\section*{Frequência de avaliação do orçamento do projeto e das reservas gerenciais}   

O orçamento será avaliado semanalmente, durante as reuniões da equipe de custos. A exposição dos resultados para o resto da equipe será feita quinzenalmente, durante as reuniões gerais.

\section*{Reservas gerenciais}

Não se aplica.

\section*{Autonomias}

Não se aplica.

\section*{Alocação financeira das mudanças no orçamento}

As mudanças de caráter corretivo podem ser alocadas dentro das reservas gerenciais do projeto, desde que dentro da alçada do gerente de projeto.

\section*{Administração do plano de gerenciamento de custos}
\begin{itemize}

\item Responsável pelo plano:

\begin{enumerate}
\item João Gabriel da Silva Souza – Gerente de planejamento de custos
\item Guilherme Matias – Coordenador do planejamento de custos
\end{enumerate}

\item Freqüência de atualização do plano de gerenciamento de custos\\

A frequência de atualização será realizada duas vezes por semana, de acordo com as exigências apresentadas.
\end{itemize}

\section*{Outros assuntos relacionados ao gerenciamento de custos do projeto não previstos nesse plano}
Todas as alterações não previstas neste plano devem ser submetidas à reunião para aprovação. Imediatamente após sua aprovação devem ser atualizadas no plano de gerenciamento dos custos com seu devido registro de atualização.

\section*{Assinaturas}
\begin{center}
Data: \rule{0.5cm}{0.1mm}/\rule{0.5cm}{0.1mm}/\rule{1cm}{0.1mm}     \\
\rule{13cm}{0.1mm}\\
ADRIANNY VIANA DE ARAÚJO AMORIM – GERENTE DE PROJETO\\
\rule{13cm}{0.1mm}\\
JOÃO GABRIEL DA SILVA SOUZA - GERENTE DE CUSTO
\end{center}
% \end{document}