%\chapter*[Introdução]{Introdução}
%\addcontentsline{toc}{chapter}{Introdução}

\chapter{Introdução}

Este trabalho apresenta ...

\section{Contexto}

Apresentar o Contexto da falta de água no nordeste e especificamente na cidade escolhida..

\section{Justificativa}

Dizer o porquê de termos escolhido a cidade de Acarí e o porquê de termos escolhido sanar esse problema..

\section{Objetivos}

Este trabalho tem por objetivo propor uma solução ...

 \subsection{Objetivos específicos}
 
 São objetivos específicos do projeto:
 
 \begin{itemize}
  \item Elaborar o projeto mecânico estrutural do sistema de captação de água;
  \item Elaborar o projeto estrutural do sistema de transporte da água para a central de armazenamento;
  \item Elaborar o projeto do sistema de monitoramento e controle da qualidade da água captada;
  \item Elaborar o projeto da matriz energética que dará o suporte para os sistemas de captação de água e monitoramento e controle;
 \end{itemize}

 
\section{Metodologia do Projeto}

Breve descrição e justificativa da metodologia utilizada para a gerência do projeto. (Scrum e PMBoK)

Foram utilizados os seguintes planos de gerenciamento do PMBoK como documentos auxiliares:

  \begin{itemize}
  \item Plano de Gerenciamento do Projeto;
  \item Plano de Gerenciamento de Escopo;
  \item Plano de Gerenciamento de Recursos Humanos;
  \item Plano de Gerenciamento de Comunicações;
  \item Plano de Gerenciamento de Aquisições;
  \item Plano de Gerenciamento de Tempo;
  \item Plano de Gerenciamento de Qualidade;
  \item Plano de Gerenciamento de Custos;
  \item Plano de Gerenciamento de Riscos.
  \end{itemize}
  
  Colocar os planos em anexos e dizer aqui qual o número anexo de cada um.
  
  \subsection{EAP}
   
   Inserir a EAP e explicar.
   
  \subsection{Cronograma}
  
  Inserir o cronograma e explicar.



