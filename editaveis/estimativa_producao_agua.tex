\textbf{Atenção:Os cálculos foram feitos com aproximações e condições de contorno especificadas ao 
decorrer do texto. Se usou uma situação ideal geral, sem embasamento em transferencia de calor ou 
matéria.Foi feito apenas para estimar a capacidade de produção de água.}

Para a determinação da a temperatura e umidade do ar de saída, utilizaremos as equações abaixo 
para determinar, numa aproximação razoável, quantidade de água que retiraremos do ar.   

A umidade relativa do ar é a porcentagem da quantidade máxima de água que o ar (nas mesmas 
condições de pressão e temperatura) consegue carregar.
Para determinar a quantidade de água no ar, usamos a equação:

\begin{equation}
 r = \frac{0.622e}{p - e}
\end{equation}

Onde: O ar é considerado um gás perfeito constituído apenas de N,O e Ar.\\
r = razão de massa entre água e ar;\\
e = pressão parcial do vapor de água;\\
p = pressão atmosférica.

Para obtermos a quantidade de água presente no ar, precisamos da pressão atmosférica e da 
pressão parcial do vapor de água. A pressão atmosférica utilizada foi a pressão a 600m de altitude : 706 
mmHg. A pressão de vapor da água, foi obtida pela junção da segunda e terceira equação da imagem 
abaixo, adotando os coeficientes da água entre 0 e 100 graus, retirados da tabela (1).