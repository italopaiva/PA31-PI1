% \documentclass[12pt,openright,oneside,a4paper,brazil]{abntex2}
% \usepackage[utf8]{inputenc}
% \counterwithout{section}{section}
% \counterwithout{figure}{chapter}
% \counterwithout{table}{chapter}
% \setlength{\parindent}{1.3cm}
% \usepackage{indentfirst}
% \setlength{\parskip}{0.2cm}
% \usepackage[bottom=2cm,top=3cm,left=3cm,right=2cm]{geometry}
% \usepackage{graphicx}
% \graphicspath{{figuras/}}
% \usepackage{placeins}
% 
% %opening
% \title{}
% \author{}
% 
% \begin{document}

%capa

% \textual
\begin{center}
 {\large Declaração de escopo}\\[0.2cm]
 {Planta de abastecimento de água potável a partir da umidade do ar}\\
 \end{center}
 
 \section*{Histórico de Alterações}
\begin{table}[h]
\centering
\begin{tabular}{|c|c|p{6cm}|p{5cm}|}

\hline
Data & Versão & Descrição & Responsável\\
\hline                               
21/04/2015 & 0.0 & Criação do documento. & César A. Marques Jr., Jonnatas L.L. Costa\\
\hline
22/04/2015 & 1.0 & Alteração dos itens: III, VII e IX & Jonnatas L. L. Costa\\
\hline
22/04/2015 & 1.1 & Alteração dos requisitos & Jonnatas L. L. Costa\\
\hline
\end{tabular}
\end{table}

\section*{Nome do projeto}
  Planta de abastecimento de água potável a partir da umidade do ar
  
  \section*{Descrição do projeto}
% O projeto visa desenvolver uma planta de abastecimento de água potável, por meio de um sistema de captação a partir da umidade do ar na cidade de Aracari-RN (município da Microrregião do Seridó Oriental, na região do Seridó), no bairro Vereador Tarcísio Bezerra Galvão. O bairro possui uma população de aproximadamente 900 habitantes e sofre constantemente com a falta de água. 
% O projeto gira em torno de uma solução tecnológica  capaz de retirar a água do ar e por processos químicos e físicos e tornar essa água potável e própria para o consumo humano de forma renovável e menos danosa possível. Para isso aplicamos conhecimentos das demais áreas da engenharia atuando no desenvolvimento e conceituação do projeto. Nossa estimativa é de que possamos produzir aproximadamente 3 mil litros de água por dia. Essa quantidade e suficiente para atender toda a população do município. 
% O projeto conta com grupos que trabalham em áreas diferentes mas interligadas e com um único propósito. Cada etapa e gerida e coordenada por representantes de forma a garantir um resultado rápido e com qualidade.

O avanço tecno-científico possibilitou novas formas de obtenção, “aproveitamento” das fontes energética naturais.
Ao estudar tais sistemas, os stakeholders do projeto almejam elaborar uma planta de capitação de água a partir da
umidade do ar. O objetivo e desenvolver uma solução tecnológica “simples” capaz de absorver a água do ar através de
químicos e físicos, tornando-a potável e apropriada para o consumo humano.

Houve uma pesquisa de modo a  encontrar o melhor sistema para tal empreendimento. O modelo mais provável é o dispositivo
da Eolewater de gerar água por meio da energia eólica, possui uma turbina WMS1000, de potência de 30kW além de uma vida
útil de aproximadamente 20 anos. Dependendo das condições em que o motor é submetido ele pode gerar até 1200 litros de
água por dia com baixo impacto para o ambiente.

A proposta contará com a multidisciplinaridade das engenharias Aeroespacial, Automotiva, Eletrônica, Energia e Software 
para o desenvolvimento da tecnologia. 

Com base nos estudos realizados pela equipe feito através de levantamento de dados em tabelas e gráficos utilizando técnicas como
5W2H e fishbone, ainda aliados com metodologias de gerenciamento como PMBoK e Scrum, entre outros, o local da implantação será
a cidade de Acari-RN (município da Microrregião do Seridó Oriental, na região do Seridó), no bairro 
Vereador Tarcísio Bezerra Galvão. Pois alem de atender os requisitos de facilidade de locomoção até a instalação,
proximidade ao ponto de consumo da energia, espaço necessário para manutenções, não é uma área muito fria, não tem prédios,
árvores, plantações e construções elevadas (diminuem a velocidade do vento e causam turbulência). Ainda há o fato do bairro
sofrer constantemente com a falta de água pois o  açude Gargalheiras, que abastece os municípios de Acarí e Currais Novos,
na região Seridó do Rio Grande do Norte, atingiu o pior volume de água de toda a sua história.
O reservatório, que tem capacidade de aproximadamente 44 milhões de metros cúbicos de água, está com pouco mais de 8\% do total.
deixando diversas pessoas em má situação, o bairro tem a  população de aproximadamente 900 habitantes.
Assim nota-se que há uma grande necessidade de uma iniciativa como essa, pois resolveria o problema que essas pessoas tem,
acarretando numa melhoria na qualidade de vida de toda essa comunidade.

O projeto gira em torno de uma solução tecnológica  capaz de retirar a água do ar e por processos químicos e físicos
tornar essa água potável e própria para o consumo humano através de filtros de 5, 1 e 0,1micrômetros aliados a um tratamento
com radiação UV onde UV matará os microorganismos e os filtros garantirão a retenção de certas bactérias, proveniente do sol
de forma renovável e menos danosa possível. Para isso aplicamos conhecimentos das demais áreas da engenharia atuando no 
desenvolvimento e conceituação do projeto. 

No custo foi levado em conta diversos fatores que afetam a produtividade e gastos com pessoal e material por se tratar
de um projeto de grande porte o investimento inicial está na casa dos milhões, entretanto é perfeitamente viável pensando
no custo beneficio do empreendimento e nas vantagens do projeto funcionando trazendo assim um retorno imediato.

O projeto conta com grupos que trabalham em áreas diferentes, mas interligadas e com um único propósito: melhorar 
a vida das pessoas. Cada etapa é gerida e coordenada por representantes de forma a garantir um resultado rápido e com
qualidade.

\section*{Objetivos do projeto}
  
  Este trabalho tem por objetivo propor um projeto de solução que consiga suprir a demanda da população. Os esforços que serão
realizados ao longo do período de desenvolvimento, no transcorrer de todas as etapas do processo buscam a qualidade e real
eficácia do produto, ou seja, oferecer a demanda de água do bairro Vereador Tarcísio Bezerra Galvão situado em Aracarí-RN, 
água potável de qualidade.

 São objetivos específicos do projeto:
 \begin{itemize}
  \item Elaborar o projeto mecânico estrutural do sistema de captação de água;
  \item Elaborar o projeto estrutural do sistema de transporte da água para a central de armazenamento;
  \item Elaborar o projeto do sistema de monitoramento e controle da qualidade da água captada;
  \item Elaborar o projeto da matriz energética que dará o suporte para os sistemas de captação de água e monitoramento e controle;
 \end{itemize}
  
\section*{Justificativa do projeto}
 \begin{enumerate}[label=\Alph*]
\item Por que deve-se pensar em projetos como esse?\\
Atualmente, cerca de 40 \% da população mundial sofre com consequências da falta de água, além da sede faltam recursos hídricos, o que gera graves implicações na economia e política. De acordo com o geólogo Sjiklomanov, do Instituto Hidrológico Estadual de São Petesburgo, Rússia, em 2000 foi previsto que: “Os países em desenvolvimento vão aumentar seu uso de água em até 200\% em 25 anos”. Em 2014 no Brasil, foi evidenciado consequências desse aumento no consumo de água juntamente com fatores climático, resultando na falta de água em cidades de Pernambuco, Minas Gerais e São Paulo.
Essa situação de falta de água não é nova, a ONU em 2003 já previa os futuros transtornos que seriam causados pela crise de água. O World Water Development Report, se destaca sobre o tema porque é um documento da ONU que também traz estudos mostrando como esse problema já afeta e mata milhares de pessoas. Este estudo prevê que 2,7 bilhões de seres humanos – 45\% da população mundial – vão ficar sem água no ano 2025.
Diante dessa situação e de previsões sobre a falta de água tão breves, devem-se tomar medidas para minimizar a situação e planejar soluções para a produção de água potável buscando outras fontes, como o ar, por exemplo. Por isso, esse projeto visa através da umidade do ar, retirar água potável e planeja um estudo de abastecimento na cidade de Acari, RN.

\item Por que  Aracarí foi a cidade escolhida?\\
Na situação de crise hídrica vivida pelo país atualmente, observa-se grandes centros urbanos sofrendo com a falta de água para consumo humano (o que antes era praticamente exclusivo para a região do semiárido nordestino). Além disso, observa-se que a seca na região nordeste vem se agravando muito nos últimos anos, especificamente na região do Seridó, que fica no semiárido do RN.
Dessa forma, soluções alternativas para o abastecimento de água potável a fim de atender o consumo humano fazem-se necessárias. Sendo assim, a região para a qual o sistema será projetado será o município de Acari – RN, pois é uma região onde há muita demanda (11303 habitantes) e tem sofrido muito com a escassez de água. A escolha dessa região baseia-se principalmente na questão social, uma vez que o projeto visa atender o consumo humano de pessoas que não tem acesso à água potável.
Além de ser uma região que apresenta necessidade de um planejamento para a amenização ou suprimento da escassez de água, é também, uma região muito quente com temperatura média anual de 33Cº, pois está localizada no polígono das Secas (local de maior concentração de seca no país). Consequentemente, o volume de água do Açude de Gargalheiras, açude este que abastece a cidade, decai consideravelmente em épocas de seca, fazendo com que haja escassez de água na cidade. A possibilidade de retirar água potável do subterrâneo, é inviável pois a água é muito salobra. Apesar de clima quente e semi-árido a umidade do ar nessa região  possui uma média anual de 60\%, o que possibilita a implantação de tecnologias que retiram a umidade do ar  e transforma em água potável. Portanto, devido a esses fatores apresentados, a cidade foi escolhida para se realizar o planejamento.

\item Por que se escolheu o bairro Vereador Tarcísio Bezerra Galvão?\\
De acordo com o censo 2010 o bairro Vereador Tarcísio Bezerra Galvão tem cerca de 900 habitantes onde a maioria, cerca de 60\%, possui entre 15 e 64 anos. Sua localização foi uma das principais motivações para a escolha do bairro, pois é um bairro muito próximo do Açude de Gargalheiras, e uma das áreas a ser planejada nesse projeto é em relação à distribuição da água, ou seja, o açude pode facilitar essa distribuição. Outro motivo seria para limitar um pouco mais o projeto a uma população menor, para assim agilizar e facilitar o estudo para planejamento.
\end{enumerate}
  
\section*{Produtos do projeto}

  O presente trabalho fornecerá os seguintes produtos:
  
  \begin{itemize}
    \item Projeto estrutural mecânico das unidades de captação de água da umidade do ar.
    \item Projeto estrutural do sistema de transporte da água para a central de armazenamento.
    \item Projeto do sistema de monitoramento e controle da qualidade da água captada.
    \item Projeto da matriz energética que dará o suporte para os sistemas de captação de água e monitoramento e controle.
  \end{itemize}

\section*{Critérios de aceitação do produto do projeto}

  Os produtos de projeto devem atender as especificações inerentes aos requisitos funcionais
  e não-funcionais de cada produto específico.
  
\section*{Exclusões específicas e limites do projeto}

  Ficam excluídos das competências do projeto:
  
  \begin{itemize}
   \item Atividades como as de tratamento de recursos hídricos já existentes na região e que não possuem
      as especificações mínimas para consumo humano, como é o caso da água salobra.
   
   \item Todo o processo de manutenção dos componentes do sistema, visando uma durabilidade de longo prazo, não fará parte 
      do escopo deste projeto. Também não fará parte do escopo a distribuição de água para localidades vizinhas.
      
   \item O uso de tecnologias auxiliares para a obtenção da água, como é o caso osmose reversa, não serão abordadas
      pois fogem da premissa inicial.
   
   \item O tratamento da energia residual gerada.
   
   \item O tratamento da água residual que for armazenada nos tanques.
   
  \end{itemize}

\pagebreak
\section*{Estrutura Analítica do Projeto}

  \begin{figure}[h]
  \begin{center}
    \includegraphics[scale=0.3]{editaveis/figuras/EAP}
    \label{EAP}
  \end{center}
  \end{figure}
  \FloatBarrier
  
\section*{Características e requisitos do produto do projeto}

  Os produtos do projeto são partes que compõem um sistema que tem por características principais ser um sistema auto sustentável
  e que não seja muito agressivo ao meio ambiente, utilizando uma matriz energética alternativa.
  A estimativa é que o produto gere cerca de 3 mil litros de água potável por dia.
  
  O sistema conta com uma central de monitoramento para garantir a qualidade da água produzida. Possui uma série de sensores
  com funções variadas, que vão de ler a umidade do ar até a medição de dados sobre a água.
  
  Uma importante característica do sistema é que ele pode se adequar às mudanças climáticas, de forma que ele pare de
  funcionar quando a umidade do ar chegar a um nível mínimo, para que não afete a saúde da população.
  
  O projeto possui quatro frentes de requisitos:
  
    \begin{itemize}
      \item Requisitos do projeto estrutural mecânico do sistema de captação da água e do transporte para a central de armazenamento;\\
	
	\textbf{Requisitos funcionais}
	  \begin{itemize}
	   \item Retirar água da umidade do ar;
	   \item Bombear água para o reservatório;
	  \end{itemize}
	  
	  \textbf{Requisitos não-funcionais}
	  \begin{itemize}
	    \item O sistema deve atender a uma demanda de água diária;
	    \item O sistema deve bombear toda água por dia produzida pro reservatório; 
	    \item O sistema deve ser composto por materiais resistentes à água;
	    \item O sistema deve possuir um reservatório de capacidade de até 3 dias de água;
	    \item O sistema deve operar com incidências de ventos de 7 a 50 m/s, umidade a partir de 40\% e 	temperatura à partir de $26\,^{\circ}\mathrm{C}$\cite{eole}
	    \item O sistema deve aproveitar o relevo da região para o transporte da água;

	  \end{itemize}
	
      \item Requisitos do projeto dos circuitos eletrônicos que irão compor o sistema de monitoramento e controle da qualidade da água.\\
	
	\textbf{Requisitos funcionais}
	\begin{itemize}
	  \item Atuar como um sistema de controle dos elementos do sistema de modo a produzir a saída desejada (manter a água própria para o consumo humano);
	  \item Obter informações climáticas da região;
	  \item Obter dados do estado reservatório;
	  \item Ler parâmetros que definem a qualidade da água;
	  \item Efetuar conversão analógica/digital dos sinais filtrados;
	  \item Processar o sinal convertido de modo que os dados possam ser transmitidos ao usuário;
	  \item Exibir dados obtidos ao usuário;
	\end{itemize}
	
	\textbf{Requisitos não-funcionais}
	\begin{itemize}
	  \item Utilizar sensores para obtenção dos dados;
	  \item Exibir dados dos sensores ao usuário em tempo real;
	  \item Utilizar filtros analógicos para retirar eventuais ruídos que a saída do sensor possa gerar;
	\end{itemize}
	
      \item Requisitos do projeto do sistema de Gestão da Informação do monitoramento da qualidade da água;\\
	 
	 \textbf{Requisitos funcionais}
	  \begin{itemize}
	   \item Apenas o moderador poderá, modificar os dados.
	   \item O sistema registrará os dados de qualidade da agua.
	   \item O sistema deve emitir um alerta caso um parâmetro de qualidade não esteja aceitável.
	   \item O sistema deve permitir consultas dos dados armazenados de datas anteriores.
	   \item O sistema deve possuir uma interface pare exibir os dados.
	   \item O sistema deve possuir uma página de \textit{login} antes de entrar no sistema.
	   \item O sistema deve possuir um mecanismo de impressão dos dados.
	   \item O sistema possuirá um mecanismo para exportar os dados.
	  \end{itemize}
	  
	  \textbf{Requisitos não-funcionais}
	  \begin{itemize}
	   \item O sistema deve ser fácil de usar, evitando excesso de digitação, de modo a dar agilidade ao processo.
	   \item O sistema deve possuir uma interface simples.
	   \item O sistema deve funcionar no sistema operacional Windows.
	   \item O sitema deve monitorar as amostras de água a cada 30 mimutos.
	  \end{itemize}
      
      \item Requisitos do projeto da matriz energética que dará o suporte para o sistema de captação de água e o sistema de monitoramento da qualidade da água;\\
      
	\textbf{Requisitos funcionais}
	\begin{itemize}
	  \item Produzir energia elétrica através da energia eólica;
	  \item Converter energia cinética em energia elétrica;
	  \item Armazenar a energia elétrica gerada;
	  \item Fornecer energia para os componentes eletrônicos, de controle e monitoramento do produto final;
	  \item Fornecer energia para o bombeamento mecânico de água.
	\end{itemize}
	
	\textbf{Requisitos não-funcionais}
	\begin{itemize}
	  \item Utilizar uma fonte renovável de energia;
	  \item Ser autossuficiente no quesito energia gerada-consumida;
	  \item Possuir eficiência energética aceitável;
	  \item Ser estável energeticamente;
	\end{itemize}
	
    \end{itemize}
  
\section*{Assinaturas}

  \begin{center}
  Data: \rule{0.5cm}{0.1mm}/\rule{0.5cm}{0.1mm}/\rule{1cm}{0.1mm}     \\
  \rule{13cm}{0.1mm}\\
  ADRIANNY VIANA DE ARAÚJO AMORIM – GERENTE DE PROJETO\\


\end{center}
% \end{document}