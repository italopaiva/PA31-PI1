\begin{resumo}[Abstract]
 \begin{otherlanguage*}{english}
   
   Water scarcity is a common issue in several places around the globe, including some regions of Brazil.
   Drought causes a lot of difficulties for the development of these regions, among the problems found, the main are related
   to the social context. The lack of water compromises agriculture and livestock, reducing food production.
   Besides this, the desperate search for water causes the population do not worry if the water is dirty or contaminated.
   With poor nutrition and poor quality drinking water, the inhabitants of these regions fall prey to many diseases.
   To solve part of this social issue, we proposed a new technology to mitigate the impact of the lack of clean water.
   Although the issue of water shortage can not be solved completely, this technology can considerably reduce the thist
   of the residents of the region, and prevent their contamination by low-quality water intake.
   This paper proposes a new technology capable of capturing water from the humidity in the air.
   We aim to do this through the use of a condenser embedded in a wind turbine.By doing this, a self-sustaining system
   is obtained, which requires no power coming from an external source, and able to benefit the inhabitants of the region
   so it could also be considered a project of social engineering. To consolidate this idea, were debated many topics as,
   for example, in which place technology would fit the best. Among the regions analyzed, the municipality of Acari,
   located in Rio Grande do Norte, stood out. Therefore, the proposed work will be suitable to the characteristics of the
   selected region, looking for the best possible efficiency.
   
   \vspace{\onelineskip}
 
   \noindent 
   \textbf{Key-words}: Air umidity. Water. Acarí.
 \end{otherlanguage*}
\end{resumo}
